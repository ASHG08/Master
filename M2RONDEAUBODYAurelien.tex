\documentclass[12pt, a4paper, oneside]{book}

% --- Paquets de base ---
\usepackage[utf8]{inputenc}
\usepackage[T1]{fontenc}
\usepackage[top=2.5cm, bottom=2.5cm, left=3cm, right=2.5cm]{geometry} % Marges académiques
\usepackage{setspace}
\onehalfspacing % Interligne 1.5 obligatoire
\usepackage{import}

% --- Personnalisation de la numérotation ---

% 1. Pour afficher les numéros jusqu'aux \subsubsection
\setcounter{secnumdepth}{3}

% 2. Pour retirer le numéro du chapitre devant les sections
\renewcommand{\thesection}{\arabic{section}}
\renewcommand{\thesubsection}{\thesection.\arabic{subsection}}

% --- Commandes sémantiques ---
\newcommand{\latin}[1]{\textit{#1}}       % Pour les mots latins
\newcommand{\oeuvre}[1]{\textit{#1}}      % Pour les titres d'oeuvres

% --- Espacement entre les notes de bas de page ---
\setlength{\footnotesep}{1.5em} % Ajustez la valeur (1em, 1.5em, 5mm...)

% --- Personnalisation des titres ---
\usepackage{titlesec}
% --- STYLE ROMAIN ---
\titleformat{\chapter}[display]
  {\normalfont\huge\bfseries\centering} % Tout centré, police normale, gras
  {\textsc{\chaptertitlename} \thechapter} % "CHAPITRE 1" en petites capitales
  {20pt} % Espace entre "Chapitre 1" et le titre
  {\Huge} % Le titre en très gros

\titlespacing*{\chapter}{0pt}{50pt}{40pt} % Espaces (Gauche, Haut, Bas)

% --- Histoire & Langues Anciennes ---
\usepackage{libertine} % Police élégante et complète pour le latin/grec
\usepackage[polutonikogreek,french]{babel} % Pour les citations en grec
\newcommand{\gr}[1]{\foreignlanguage{greek}{#1}}
\usepackage{graphicx}
% --- Bibliographie (Style Sorbonne / École Française de Rome) ---
\usepackage{csquotes}
\usepackage[
    backend=biber,
    style=verbose-ibid,    % Style standard présent partout
    language=french,       % Force la traduction des termes (éd., vol., etc.)
    ibidtracker=constrict,
    autocite=footnote
]{biblatex}
\addbibresource{bibliographie.bib}
\usepackage{microtype}
% --- Espacement de la bibliographie ---
\setlength{\bibitemsep}{1em}

\usepackage{xcolor}
\usepackage{tabularx}
\usepackage{booktabs} % Pour les jolies lignes

\begin{document}

% --- Page de Garde ---
\begin{titlepage}
    \centering

    % --- LOGO ---
    % Remplacez 'logo-sorbonne' par le nom exact de votre fichier image
    % 'width=0.4\textwidth' ajuste la taille de l'image
    \includegraphics[width=0.5\textwidth]{logo-sorbonne}

    \vspace{1cm}

    {\Large \textsc{Sorbonne Université}} \\
    {\Large \textsc{UFR d'Histoire}} \\

    \vfill

    % --- TITRE ---
    \rule{\linewidth}{0.5mm} \\[0.4cm]
    {\huge \bfseries Ambroise de Milan et le pouvoir impérial : \\[0.4cm]
    \Large Une pensée et une autorité chrétienne au cœur de l'Empire romain.} \\
    \rule{\linewidth}{0.5mm} \\[1.5cm]

    % --- AUTEUR ---
    {\Large \textbf{Aurélien RONDEAU-BODY}} \\[2cm]

    % --- INFOS MASTER ---
        \noindent % Empêche l'indentation
        \begin{minipage}[t]{0.45\textwidth}
            \begin{flushleft}
                \large
                \textbf{Master 2 :}\\
                Mondes antiques\\
                Spécialité christianisme antique
            \end{flushleft}
        \end{minipage}
        \hfill % Espace élastique entre les deux blocs
        \begin{minipage}[t]{0.5\textwidth}
            \begin{flushright}
                \large
                \textbf{Sous la direction de :} \\[0.2cm]
                M. Jean-Marie \textsc{Salamito}\\
                M. François-Xavier \textsc{Romanacce}\\
                M. Jonathan \textsc{Cornillon}
            \end{flushright}
        \end{minipage}

    \vfill

    % --- DATE ---
    {\large Année universitaire 2025-2026}

\end{titlepage}


\tableofcontents
\chapter*{Introduction}
\addcontentsline{toc}{chapter}{Introduction}

%blabla

%\paragraph{test}

%\bigskip

%hdhdhdh

\chapter{Théoriser le gouvernement romain}
\clearpage
\input{chapitre1/C1introduction.tex}
\vspace{2cm}
\section{L’utopie d’un gouvernement républicain}

\section{L’idéal du monarque chrétien par Ambroise}

\section{Conception et construction d'un pouvoir réaliste}

\vspace{2cm}
\input{chapitre1/C1conclusion.tex}

\chapter{Une figure politique sous le regard de la foi}
\clearpage
\input{chapitre2/C2introduction.tex}
\vspace{2cm}
\section{Les fondements scripturaires dans la théorie politique ambrosienne}

\section{Un prince nécessairement religieux}

\vspace{2cm}
\input{chapitre2/C2conclusion.tex}

\chapter{Le rôle politique de l'Église et d'Ambroise}
\clearpage
\section*{Introduction}

À la fin du IV\textsuperscript{e} siècle, les relations entre l'Église et l'Empire se voient fortement évoluer : l'intervention fréquente des empereurs dans les affaires de foi depuis Constantin est de plus en plus critiquée au sein de l'Église qui peine à se détacher de l'influence du pouvoir impérial. Lors de son arrivée dans l'épiscopat de Milan, Ambroise ne tarde pas à se montrer comme un penseur autonome, prêt à défier les empereurs pour servir sa vision de la foi chrétienne et lutter contre ce qu'il considère être des hérésies, dont certaines sont soutenues par le pouvoir impérial. Si les chapitres précédents ont permis d'établir les fondements théoriques de la pensée politique d'Ambroise ainsi que sa conception d'un pouvoir chrétien idéal sous le regard de la foi, il convient désormais d'analyser comment cette réflexion s'incarne concrètement dans la réalité du gouvernement impérial. Le troisième chapitre de ce mémoire a pour objectif d'explorer la vision d'Ambroise sur les relations que le monde épiscopal doit entretenir avec le pouvoir impérial, aussi bien sur le plan institutionnel que personnel. Ce qui nous amène évidemment à regarder les diverses actions de l'évêque de Milan pour interagir de façon concrète avec les souverains, en tant que figure qui redéfinit le rôle d'évêque et de conseiller impérial.

\bigskip

Le thème central de cette étude est celui d'une nouvelle \latin{auctoritas} : celle de l'Église. C'est une idée chère à Ambroise qui, bien qu'il ne l'évoque pas de façon explicite, guide l'ensemble de ses réflexions et actions dans le champ politique. Il ne se contente plus de réclamer une liberté de culte ou une protection juridique pour les représentants de la foi chrétienne, mais il cherche à instaurer une forme d'autorité religieuse capable de s'impliquer de façon indépendante dans les affaires impériales. Cette \latin{auctoritas} doit pouvoir se faire sentir par divers domaines : le pouvoir intrinsèque de l'institution, l'autorité offerte par le fait d'être un représentant de la foi, la légitimité diplomatique à travers les compétences d'orateur, ou encore par la confiance octroyée par les empereurs grâce à la construction d'une relation personnelle. Autant de domaines théorisés et maîtrisés par Ambroise qui lui permettent d'agir de façon importante dans la politique romaine. Giuseppe Zecchini souligne particulièrement bien la réflexion d'Ambroise au sujet de l'\latin{auctoritas} dans son ouvrage sur la pensée politique romaine : «~l'évêque de Milan fut le premier à comprendre, théoriser et appliquer le nouveau rôle politique de l'Église, appelée à combler par son autorité le vide progressivement laissé par le Sénat.\autocite[???]{zecchini_pensiero_politico}~» Cette observation est capitale : l'historien suggère ici que l'\latin{auctoritas} dont veut s'emparer Ambroise est une captation d'une prérogative sénatoriale puis impériale et non pas une invention pour donner un plus grand pouvoir à l'Église.\footnote{Voir l'explication de l'\latin{auctoritas} dans les réflexions d'Ambroise dans le 1.1.1.} L'enjeu est donc de comprendre comment l'évêque de Milan met en place un travail de collaboration avec les empereurs tout en imposant une autonomie importante de l'Église dans les affaires religieuses.

\bigskip

Cette réflexion nous oblige à questionner la nature exacte de cette nouvelle autorité. Bien qu'il soit évident qu'Ambroise revendique une \latin{auctoritas} indiscutable sur les dogmes et les affaires ecclésiastiques, comment arrive-t-il à justifier l'arrivée d'une nouvelle autorité, concurrente sur certains points au pouvoir de l'empereur ? Un siècle plus tard, la séparation des sphères d'influence de l'autorité épiscopale et du pouvoir impérial semble être plus largement définie et l'\latin{auctoritas} du pouvoir spirituel atteint un tout autre niveau lorsque l'évêque de Rome Gélase écrit à l'empereur d'Orient Anastase pour lui rappeler que « la direction du monde est assurée de concert par l'\latin{auctoritas sacrata pontificum} et par la \latin{regalis potestas}\autocite{sassier_auctoritas}.~» L'Église parvient donc à consolider son autorité, y compris dans le domaine politique tout au long du V\textsuperscript{e} siècle. Mais que parvient à faire concrètement Ambroise sur ce sujet pendant son épiscopat ? Il est donc nécessaire d'observer la façon dont il construit cette nouvelle autorité, entre autonomie religieuse, nécessité du soutien impérial, conflits avec les représentants du pouvoir et enfin ambition personnelle principalement marquée par sa proximité avec les différents souverains.

\bigskip

L'ensemble de ce chapitre s'inscrit tout particulièrement dans la suite des analyses de Peter Brown développées dans son \textit{Pouvoir et Persuasion}\autocite{brown_power_1992}. L'historien y décrit l'arrivée des évêques comme une nouvelle figure d'autorité avec laquelle le pouvoir impérial est contraint de composer. Les philosophes et simples conseillers politiques sont dépassés par ces évêques dont la légitimité explose en tant que représentants d'une communauté chrétienne importante. Le dialogue entretenu diffère largement face à des interlocuteurs qui ne cherchent plus seulement à plaire mais à guider l'empereur dans un chemin de foi et de justice propre à la chrétienté. Cette figure de pouvoir et d'influence est particulièrement bien représentée par le personnage d'Ambroise qui use de sa popularité et de ses connaissances pour s'imposer comme un outil indispensable du pouvoir impérial.

\bigskip

Les différentes analyses et réflexions de ce chapitre s'appuient majoritairement sur le corpus des \latin{Epistolae} d'Ambroise plus que sur ses écrits religieux et théoriques. En effet, les traités dogmatiques ou exégétiques restent souvent au stade théorique, ce qui ne nous permet pas, dans le cadre d'un chapitre sur les relations au pouvoir et les actions politiques, de véritablement comprendre tout l'enjeu de cette \latin{auctoritas} ambrosienne et de son influence sur les directives impériales. Ses lettres, en revanche, nous permettent de faire le lien entre ses réflexions et ses actions. Elles agissent comme le reflet réaliste de son idéal, obligé de s'adapter à son interlocuteur et aux circonstances politiques pour imposer son autorité et celle de l'Église sans risquer la rupture définitive des relations. C'est donc principalement à partir de ce corpus que se perçoit la manière dont Ambroise amène l'Église à s'installer comme une institution majeure de l'Empire romain, tout en se frayant une place de conseiller impérial indispensable.

\bigskip

Ainsi, ce chapitre s'attache à démontrer comment Ambroise transforme ses réflexions théoriques en une influence politique concrète, à travers une analyse la plus complète possible de l'\latin{auctoritas} institutionnelle puis personnelle que développe Ambroise aux côtés des empereurs qu'il côtoie.

\vspace{2cm}
\section{Une nouvelle dynamique entre l'Église et le pouvoir impérial}

\subsection*{Introduction}

Au coeur de la réflexion politique d’Ambroise se présente une complexité intéressante : comment affirmer l'existence d’une autorité de l’Église chrétienne, qui soit à la fois complètement autonome vis-à-vis du pouvoir impérial et capable de jouer un rôle politique important dans la gestion de la société, sans pour autant s'emparer d’un pouvoir temporel qui entrerait en conflit avec les prérogatives de l’empereur et de son administration. Si la séparation entre les domaines du politique et du religieux existe depuis le début du christianisme, Ambroise apporte des précisions sur les sphères d’influence, faisant de l’indépendance épiscopale une nécessité pour lancer des actions publiques. L’évêque de Milan réfléchit alors à l’ensemble des relations existantes entre les institutions de l’Église et du pouvoir impérial pour développer et légitimer l’autorité de l’Église dans la société. Ces relations sont notamment mises en évidence par le corpus de correspondance entre Ambroise et les empereurs. C’est ce qui amène Giuseppe Visonà à qualifier le livre dix des Lettres d'Ambroise de « manuel de nouvelle doctrine des rapports entre l’Église et l’État par son principal théoricien.\footnote{« \latin{E in definitiva un prontuario con la nuova dottrina Chiesa-Stato a firma del suo maggiore elabitore}.». \cite[33]{pizzolato_nec_timeo}.}~»

\bigskip

Dans cette perspective de questionnement sur le rôle de l’Église, Ambroise fait revenir le concept d’\latin{auctoritas}, pour le placer au service des évêques. Il est évident que cette idée se fait dans un objectif pleinement religieux de revendication de l’autorité divine, il ne faut pas y chercher une volonté de concurrence avec l’empereur et d’appropriation des missions administratives appartenant au pouvoir impérial. Cette première partie d’un chapitre centré sur le pouvoir politique de l’Église et d’Ambroise vise donc à comprendre les nouvelles relations dont parle Visonà, et la façon dont l’évêque de Milan les influence selon une vision qui lui est propre. En laissant pour l’instant de côté les relations personnelles et le rôle direct d’Ambroise auprès des empereurs, nous nous concentrerons sur la définition de cette \latin{auctoritas} de l’Église et sur la façon dont elle s’impose comme une institution indépendante et influente, redessinant l’équilibre des pouvoirs au sein de l’Empire romain.

\subsection{Introduire une autorité chrétienne}

\subsubsection{\latin{Auctoritas}: Héritage Républicain et appropriation ambrosienne}

Bien que l’on ne puisse pas qualifier de révolution politique l’appropriation du concept d’\latin{auctoritas} par Ambroise et l’Église chrétienne, il est tout de même nécessaire de comprendre l’impact qu’a l’utilisation de ce terme par une nouvelle autorité institutionnelle, qui cherche à se détacher des souvenirs glorieux du Sénat de la période républicaine, et de l’actuelle puissance de l’empereur, détenteur officiel de cette fameuse \latin{auctoritas} depuis Auguste. L’évêque de Milan est le premier personnage clérical à utiliser, de façon conséquente, le terme \latin{auctoritas}. Cette notion est par exemple pratiquement absente des traductions de l’Ecriture par Jérôme, alors qu’Ambroise l’utilise pour définir la provenance divine d’une sanction d’ordre spirituel \autocite[214]{sassier_auctoritas}. Pour pleinement saisir la portée de l’appropriation de ce concept par Ambroise, il est nécessaire de revenir sur certaines structures de la pensée romaine traditionnelle, qui englobe une séparation juridique entre les deux notions d’\latin{auctoritas}, souvent simplement traduit par autorité, et de\latin{potestas}, pouvant signifier puissance ou pouvoir. L’évêque de Milan est un ancien gouverneur et membre de l’aristocratie romaine, sa pensée politique est donc fondamentalement ancrée dans les traditions romaines. Il connaît parfaitement l’ancienne République et donc l’origine de la notion d’\latin{auctoritas}, ce qui lui permet de l’utiliser avec plus de justesse à son profit.

\bigskip

Il est difficile de comprendre avec précision ce qu’englobe l’\latin{auctoritas} en tant que notion juridique, c’est une idée implantée dans la société romaine qui n’est pas remise en question, mais également jamais franchement définie. Il est par contre clair qu’une séparation importante existe entre l’autorité et le pouvoir, des notions pouvant être dans les mains d’une ou de plusieurs personnes. Les historiens Michel Christol, Frédéric Hurlet et Pierre Cosme en parlent comme ceci : « L’autorité telle que les Romains la concevaient était un surcroît de pouvoir reconnu à un groupe où un individu, conférant une influence et un ascendant à celui qui en était le dépositaire. Elle s’exprimait à travers les initiatives que le groupe ou l’individu prenait et qui étaient suivies d’effet : elle était cette puissance qui permettait de l’emporter dans la prise de décision, sans avoir recours à la force. ». \autocite{christol_histoire_2021} Tout au long de la période républicaine, dans les institutions romaines, l’\latin{auctoritas} est plutôt dans les mains du Sénat, et la \latin{potestas}, dans celles des magistrats. L’autorité du Sénat n’est donc pas un pouvoir exécutif, mais l’institution peut présenter des décisions « obligatoires », que doivent appliquer les magistrats. C’est bien par leur influence, par l’\latin{auctoritas}, que s’exerce le véritable pouvoir du Sénat. C’est sur cette base juridique que va s’appuyer Ambroise pour instaurer une influence grandissante et pratiquement obligatoire de l’Église sur l’empereur.

\bigskip

Un premier changement juridique intervient avec Auguste, qui fait de cette distinction entre autorité et puissance le cœur de son renouveau politique. Le premier empereur s’impose très rapidement comme le détenteur unique de l’\latin{auctoritas}, à l’insu du Sénat qui s’efface dans un rôle presque uniquement symbolique. Dans le chapitre 34 de ses \latin{Res Gestae}\footnote{« \latin{Post id tempus auctoritate omnibus praestiti, potestatis autem nihilo amplius habui quam ceteri qui mihi quoque in magistratu conlegae fuerunt.}.» {Traduction de Frédéric Hurlet : « après cette époque, je l’ai emporté sur tous par mon \latin{auctoritas}, mais je n’ai pas eu plus de \latin{potestas} que tous les autres qui ont été mes collègues dans chaque magistrature. »}.}, Auguste décrit clairement sa supériorité par son autorité, tout en précisant qu’il possède un pouvoir identique à ceux des autres magistrats. Auguste est alors princeps, celui qui propose et influence, par son \latin{auctoritas}, et magistrat, celui qui exécute, par sa \latin{potestas}\autocite{magdelain_auctoritas}. Le pouvoir moral de l'\latin{auctoritas} fondé sur l’influence personnelle de l’empereur se développe tout au long de l’Empire pour devenir un statut juridique affirmé chez le pouvoir impérial. En étant pleinement conscient de cette culture politique, Ambroise de Milan se retrouve devant le double défi de faire renaître une autorité institutionnelle à travers l’Église, tout en supplantant l’\latin{auctoritas} impérial, à la fois par son statut d’évêque et de membre de l’Église que je développe dans cette partie, ainsi que par sa proximité avec les empereurs et ses relations personnelles, que je développerai dans une seconde partie du chapitre.

\bigskip

Ambroise ne rejette pas dans ses écrits le schéma juridique de l’\latin{auctoritas} et de la \latin{potestas}, au contraire, il s’en empare pleinement et cherche à donner à l’Église et une autorité et une influence capable de travailler à égalité avec le pouvoir impérial dans la gestion de la société romaine. Et pour légitimer le nouveau rôle qu’il attribue à l’Église chrétienne, Ambroise apporte un aspect divin à l’\latin{auctoritas}. Dans sa thèse sur l’utilisation du terme \latin{auctoritas} par Cyprien, Tertullien et Ambroise\autocite{ring_auctoritas}, Thomas Gerhard Ring démontre que l’évêque de Milan parle peu, explicitement, d’\latin{auctoritas}, et qu’il limite son utilisation aux formes d’autorités religieuses.

\textcolor{red}{Le terme d’auctoritas dans les textes d’Ambroise :
Utilisation de l’idée d’auctoritas dans le domaine religieux, il l’utilise en De officiis III 129-132 ?
L’auctoritas apparait comme une qualité permanente, une désignation de la dignité. Il en parle en Ep 37 8, en parlant d’isaac qui place jacob au dessu d Esau
Il utilise ce terme de facon assez large : qqn ayant la foi intérieur, qqn ayant le droi d’agir etc, le sujet de l’auctoritas peut même être un comporterment de qqn.
p. 122, lors de l’éloge funèbre à Valentinien II, il loue la iustitia et l’auctoritas de l’empereur défunt, mais plutot comme une qualité personnelle que un point de vu politique.} Je développerai cet aspect tout au long du chapitre, mais il n’y a pas chez Ambroise de volonté de soustraire à l’empereur une partie de son pouvoir temporel. Ainsi, il trouve les fondements de l’autorité épiscopale dans les volontés divines ou dans les apôtres, des formes d’autorité incontestable pour un empereur chrétien. C’est cette exemple qu’utilise également Yves Sassier que j’ai cité précédemment :

\bigskip

\begin{quote}
    « Tu vois qu’il condamne cet homme par l’autorité apostolique [...] et pourtant il ne lui a pas ôté l’espérance, lui qu'il a invité à la pénitence\footnote{« \latin{Vides quod hunc apostolica auctoritate condemnet [...] et tamen non interclusit ei spem, quem invitavit ad poenitentiam.}~». Ambroise, De poenitentia, II, 4, 40. Traduction personnelle.}~»
\end{quote}

\bigskip

L’utilisation de la notion d’\latin{auctoritas} avec celle d’apostolica permet de définir une sanction d’ordre spirituel. Et c’est à partir de ce fondement chrétien que l’évêque développe sa réflexion : l’autorité divine est immuable et évidente, par conséquent l’Église en tant que corps du Christ et l’évêque en tant que successeur des apôtres se retrouvent en possession de cette nouvelle forme d’autorité. L’\latin{auctoritas} divine apparaît dans la société romaine christianisée comme une puissance absolue et contraignante pour la conscience, la rendant bien supérieure à l'influence politique traditionnelle, ce qui amène une forme d’autorité innée chez les évêques, c’est une idée que je développe dans le point suivant. Thomas Gerhard Ring pousse les comparaisons, et la compréhension l'utilisation du terme par Ambroise, un peu plus loin. L’évêque de Milan s’appuie très nettement sur la pensée de Tertullien qui définit l’\latin{auctoritas} divine comme la volonté révélée de Dieu qui ne peut donc qu’être acceptée et suivie par les Hommes ayant la foi chrétienne. Il y a donc une rupture avec l’origine première de l’\latin{auctoritas} présent chez le Sénat ou l’empereur. L’historien allemand la décrit comme fonctionnant de manière horizontale, d’humains à humains, grâce à un consensus social qui reconnaît les qualités d’un individu ou d’une institution. Avec Ambroise, la nouvelle vision de l’\latin{auctoritas} est verticale, elle provient du divin et descend de Dieu vers l’Église pour créer une \latin{auctoritas} épiscopale qui influence les décisions impériales\autocite[À partir de la page 163]{ring_auctoritas}.

\bigskip

La réflexion ambrosienne sur la nouvelle autorité de l’Église, largement décrite par T.G. Ring, ne s’arrête pourtant pas à un phénomène de théologie abstrait. Ambroise évoque à plusieurs reprises la réalisation concrète, et bien connue dans l’empire, d’une autorité institutionnelle et juridique dans l’Église chrétienne : l’audientia episcopalis. Cette instance est une forme de tribunal épiscopal, utilisé depuis plusieurs siècles par les communautés chrétiennes, puis rendu légal et officiel par Constantin en 318, qui permet aux évêques de rendre la justice sur des préoccupations civiles, à partir du moment où l'une des deux parties souhaite y avoir recours. Dans le cadre que j’ai posé tout au long de cette sous-partie, il est possible de concevoir cette fonction de l’évêque comme la réalisation de cette \latin{auctoritas}. La justice de l’évêque est en effet demandée car son jugement est vu comme rapide, non corrompu comme peuvent l’être certaines institutions romaines, mais surtout fondé sur une influence divine, bien moins contestable que le droit civil.

\bigskip

\begin{quote}
    « Quelle autorité a-t-il pour dénoncer la fraude celui qui a pu saisir l’appât si laid des plaisirs ? Celui en effet qui accuse autrui de péché doit être lui-même exempt de péché. Je n’en appellerai donc pas à des balivernes de ce genre pour appuyer sur ce point l’autorité de la censure de l’Église [...]\footnote{«~\latin{Quam hic redarguendi habet auctoritatem doli, qui tam turpe captarit aucupium deliciarum. Qui enim alterum peccati arguit, ipse a peccato debet alienus esse. Non ergo huiusmodi nugas ego in hanc ecclesiasticae censionis auctoritatem uocabo [...]}~». \cite[\nopp \latin{De Officiis} III, 72]{ambroise_devoirs_2}.}~»
\end{quote}

\bigskip

Bien qu’adressé à son clergé, cet extrait de son traité \latin{De Officiis} nous permet de réaliser le recours récurrent, et même nécessaire pour l’empire, à cette forme d’autorité et de législation. On peut tout de même noter ici qu’Ambroise met également en évidence les limites de la juridiction cléricale : le tribunal repose sur les notions de péchés, de foi et de morale, il y est donc impossible d’y juger des affaires purement civiles si les deux parties sont considérés comme étant dans l’erreur et le péché, comme c’est le cas dans cet exemple autour de l’achat d’une propriété. Dès lors, on retrouve cette importance chez Ambroise de développer une nouvelle autorité sans entrer en concurrence avec les formes de pouvoirs temporels existantes. L’\latin{auctoritas} que l’évêque implante dans l’Église chrétienne ne se fonde pas sur la seule motivation de contester le pouvoir absolu du système impérial et des institutions qui en sont issues. Au contraire, il cherche à travailler en parallèle de l’empereur et profite d’une foi chrétienne et d’une autorité divine très largement diffusée et respectée pour développer considérablement le rôle de l’Église et des évêques dans divers domaines de la société.

\bigskip

\subsubsection{Imposer les évêques au coeur du gouvernement impérial}

Le postulat d’Ambroise est donc que l’\latin{auctoritas} sur laquelle doit s’appuyer l’Église chrétienne émane du divin, ce qui légitime l’institution à exercer un pouvoir sur la société sans en référer à l’empereur. En suivant le raisonnement, cette \latin{auctoritas} permet également aux évêques de s’imposer comme des membres à part entière du système administratif impérial, et donc comme conseillers ou juges dans les affaires religieuses. Grâce à l’aspect divin derrière ce changement, découle une autorité innée des évêques, qui ne gagnent pas leur influence par une concession ou un soutien impérial, mais qui provient simplement de sa charge. L’évêque, par le simple fait d’être évêque, est la seule source légitime d’autorité pour toute question relative à la foi ou la morale, créant une forme d'expertise religieuse que le pouvoir se doit de reconnaître et consulter. Ambroise pousse cette idée sur le principe simple de la compétence professionnelle : il est nécessaire d’être établi et reconnu comme autorité religieuse pour pouvoir juger la foi, et ce peu importe son importance civile. Dans le cadre de ce qui est aujourd’hui appelé le conflit des basiliques milanaises\footnote{Crise religieuse entre 385 et 386 opposant la communauté arienne de Milan, derrière le personnage d’Auxence et soutenu par la mère de l’empereur, Justine, à Ambroise. Il est question du don d’une basilique aux ariens, ce à quoi l’évêque de Milan s’oppose fermement, allant jusqu’à s’écarter de l’avis de Valentinien II, au nom de la foi chrétienne.}, l’empereur Valentinien II convoque Ambroise au consistoire afin qu'ait lieu un débat contre Auxence, évêque arien fidèle au dogme de Rimini. Une affaire religieuse, en l'occurrence un conflit entre deux pensées chrétiennes opposées, doit donc être réglée par l’empereur et chez l’empereur. Evidemment, Ambroise repousse fermement cette idée et évoque, pour se défendre et éviter de se faire qualifier de « rebelle », l’expertise religieuse des évêques et la nécessité de les consulter :

\bigskip

\begin{quote}
    « À cette convocation je fais, je pense, la réponse qui convient. Et personne ne saurait me juger rebelle quand je soutiens ce que ton père, d’auguste mémoire, a non seulement prescrit oralement mais aussi sanctionné par ses lois : « Dans une affaire concernant la foi ou quelque dignité ecclésiastique, doit juger celui qui n’est pas d’un rang inférieur et n’a pas un statut juridique différent. » Ce sont les mots du rescrit\footnote{Ces textes de lois ne nous sont pas parvenus.}, autrement dit, il a voulu que ce soient des évêques qui jugent des évêques.\footnote{« \latin{Cui rei respondeo, ut arbitror, competenter. Nec quisquam contumacem iudicre me debet, cum hoc asseram quod augustae memoriae pater tuus non solum sermone respondit sed etiam legibus suis sanxit : « In causa fidei uel ecclesiastici alicuius ordinis eum iudicare debere qui nec munere impar sit nec iure dissimilis. » Haec enim uerba rescripti sunt hoc es sacerdotibus uoluit iudicare.}~». \cite[\nopp 75,2]{ambroise_lettres}.}~»
\end{quote}

\bigskip

Bien qu’il soit important de noter que ce conflit ne nous est connu que par la vision d’Ambroise et de ses proches, tel que Paulin, et qu’il est donc impossible de connaître les véritables objectifs et volontés de Valentinien ou Auxence et leur réaction après les écrits d’Ambroise, cette citation permet de constater la mise en place du respect de l’autorité ecclésiastique, tel que le veut notre évêque. Bien sûr, Ambroise sait se plier aux exigences du pouvoir impérial, il précise d’ailleurs dans un autre passage qu’il se rendra au consistoire si telle est la volonté de l’empereur, en revanche il cherche à aborder la question du rôle des évêques avec plus de précision et de justesse pour toucher directement Valentinien II et apporter une grande compréhension de la situation. Ici, en citant Valentinien Ier, il place l’actuel empereur face à une autorité plus personnelle et respectée dans l’empire, et affirme ainsi la position de l’Église par une légitimité juridique de longue date. L’application concrète dans cette lettre du recours, désormais obligatoire, aux évêques reflète parfaitement l’objectif théorique d’Ambroise : faire en sorte que l’Église ne soit plus une simple administration religieuse mais une institution incontournable dans la gestion de l’empire romain. À partir de cette perspective, Ambroise refuse que les débats de foi se tiennent dans le consistoire, lieu de la \latin{potestas} impériale par excellence, afin d'éviter de voir le déplacement de l’autorité épiscopale vers  l’empereur :

\bigskip

\begin{quote}
    « Je me rendrai volontiers au palais de l’empereur si c’était en accord avec mon devoir d’évêque, pour mener ce débat dans le palais plutôt que dans l’église. Mais, dans le consistoire, l’usage veut que le Christ soit non pas un accusé, mais un juge. Une question de foi c’est dans l’Église qu’elle doit être plaidé, qui peut dire le contraire ?\footnote{« \latin{Ad palatium imperatoris irem libenter, si hoc congrueret sacerdotis officio, ut in palatio magis certarem quam in ecclesia. Sed in consistorio non reus solet Christus esse sed iudex. Causam fidei in ecclesia agendam quis abnuat ?}~». \cite[\nopp 75A,3]{ambroise_lettres}.}~»
\end{quote}

\bigskip

L’autorité de l’Église apparaît ainsi comme une réalité avec laquelle l’empereur se doit de travailler. Il ne s’agit pas d’une concurrence mais bien d’un travail en commun où chaque institution contrôle et agit dans son domaine d’expertise. Il n’est donc pas possible pour des affaires de foi de se régler auprès d’une justice impériale, car celà mettrait les évêques directement sous les ordres de l’empereur. Ambroise semble pousser les empereurs, et en particulier Valentinien II avec qui il partage une plus forte relation, à intégrer l’institution ecclésiale dans son processus de prise de décision, comme c’est le cas pour les autres instances du gouvernement :

\bigskip

\begin{quote}
    « Du reste, si on ne me fait pas assez confiance, fais venir les évêques que tu voudras, qu’on débatte, empereur, des mesures à prendre dans le respect de la foi. Si sur des affaires financières, tu prends l’avis de tes comtes, combien est-il plus équitable que, sur une affaire qui concerne la religion, tu prennes l’avis des prêtres du Seigneur.\footnote{« \latin{Certe si mihi parum fidei defertur, iube adesse quos putaueris episcopos, tractetur, imperator, quid salua fide agi debeat. Si de causis pecuniariis comites tuos consulis, quanto magis in causa religionis sacerdotes Domini aequum est consulas.}~». \cite[\nopp 74,27]{ambroise_lettres}.}~»
\end{quote}

\bigskip

La comparaison qui est faite avec les cours financières n’est pas anodine, Ambroise évoque l’un des aspects les plus importants du pouvoir impérial, à la fois pour placer l’Église comme une institution de premier plan dans la gestion de l’empire, mais également pour rappeler à l’empereur qu’il possède toujours la mainmise dans la plupart des secteurs du pouvoir, les questions économiques et militaires entre autres. L’idée, et je continuerai de la développer tout au long de ce chapitre, est vraiment d’imposer l’autorité ecclésiastique sans « effrayer » le pouvoir impérial qui pourrait avoir le sentiment d’une perte sèche de capacité politique. De la même façon que les magistrats en charge des impôts ou du budget sont indispensables dans la gestion de la société, il n’est plus possible de diriger un empire chrétien sans s’appuyer sur l’autorité des évêques. Ambroise établit véritablement l’Église comme une institution indispensable du gouvernement impérial. La relation entre les évêques et l’empereur se forme donc avec Ambroise dans un esprit de collaboration nécessaire, avec tout de même l’idée, et l’évêque de Milan n’y manque pas, de revendiquer une pleine et entière autonomie de l’Église.

\bigskip

\subsubsection{La défense d'une indépendance cléricale}

En se revendiquant seule experte des questions de la foi, l’Église se doit d'administrer et de contrôler ses interventions elle-même, sans ingérence impériale, c’est ce qui constitue le cœur de la défense de l’indépendance épiscopale. L’important corpus de lettres que nous possédons, d’Ambroise s’adressant aux empereurs, permet de pousser la compréhension de l’objectif d’Ambroise vis-à-vis du développement d’une \latin{auctoritas} des « prêtres du Seigneur » qui se rapproche de l’influence que possédait le Sénat dans la période Républicaine. Ainsi, après avoir démontré le droit des évêques d’intervenir auprès de l’empereur lors des questions religieuses, Ambroise cherche à clarifier l’idée d’autonomie qui doit émaner de l’Église chrétienne. Comme indiqué précédemment, l’\latin{auctoritas}  de l’Église provient directement du divin, donc une autorité intrinsèque et qui ne peut pas se soumettre au pouvoir impérial. Cette autonomie est en effet nécessaire pour que puisse s’exercer sans contestation la nouvelle autorité spirituelle. Charles Norris Cochrane s’arrête un moment sur les détails de l’indépendance que vise Ambroise. On peut notamment comprendre qu’elle englobe à la fois « l'autodétermination de l’organisme et la liberté des ministres, en tant que représentant, de s’exprimer et d’agir comme ils l’entendent.\autocite[347]{cochrane_christianity}~». L’idée d’Ambroise n’est en aucun cas d’interdire l’empereur d’intervenir dans toutes questions concernant la foi d’un citoyen ou la gestion interne de l’Église, mais bien de limiter ces interventions à des cas demandés par les membres de l’Église\footnote{Je reviens longuement sur les droits et devoirs des interventions impériales dans le domaine religieux en 1.2.1 et 1.2.2.}. Mais pour en arriver à délimiter les interventions impériales, il est nécessaire d’imposer l’indépendance de l’Église dans les relations politiques. Et évidemment ce que l’on peut immédiatement citer est de nouveau un extrait de la lettre 75 adressée à Valentinien II, qui met de nouveau en avant l’autorité ecclésiastique dans les affaires religieuses et rappelle le rôle du souverain en matière de foi :

\bigskip

\begin{quote}
    « Quand as-tu entendu dire, empereur très clément, que, dans une affaire touchant la foi, des laïcs aient jugé le cas d’un évêque ? [...] En tout cas si nous passons en revue les livres des divines Ecritures et les temps anciens, quelle est la personne qui peut contester que, dans une affaire touchant la foi, je dis bien une affaire touchant la foi, ce sont les évêques qui jugent habituellement la conduite des empereurs chrétiens, et non pas les empereurs celle des évêques ?\footnote{« \latin{}~». \cite[\nopp 75,4]{ambroise_lettres}.}~»
\end{quote}

\bigskip

Ce passage se révèle particulièrement intéressant pour comprendre la pensée d’Ambroise au sujet de l’autonomie épiscopale. Une chose frappe ou dérange dès la première lecture : l’appuie qui est fait sur la notion « d’affaire touchant la foi », « \latin{in causa fidei} ». Tout comme je l’ai évoqué lors de la précédente partie, Ambroise veut accentuer l’idée que l’\latin{auctoritas} de l’Église ne s’impose que dans un seul domaine : la foi. L’objectif derrière est double, premièrement rassurer le pouvoir impérial sur le fait que le développement d’une nouvelle forme d’autorité ne vient pas mettre en péril la \latin{potestas} du prince, et deuxièmement le fait de recentrer les volontés de l’Église autour d’un seul thème garantit un contrôle absolu de cet aspect, et donc une plus grande autonomie. Un autre point intéressant de la citation est le recours de deux autorités incontestable pour un aristocrate chrétien : les Écritures et les temps anciens. Si cette idée peut paraître un peu vague dans ce à quoi elle fait référence, il faut savoir que l’expression « livres ou textes des Saintes Écritures » est très fréquente dans ses travaux. Gérard Nauroy dénombre 16 occurrences et désigne ce terme comme « la succession des divers livres qui constituent le corpus des Écritures.\footnote{Gérard Nauroy, note 5, page 503. \cite[\nopp 77,3]{ambroise_lettres}.}~». Ambroise en appel donc à une mémoire chrétienne plus que romaine, faisant confiance à la culture religieuse de l’empereur, pour soumettre la légitimité de sa vision derrière une autorité textuelle capable de s’imposer sur les décisions impériales.

\bigskip

Mais l’indépendance de l’épiscopat apparaît par cette lettre et par ce conflit comme une lutte « de son temps » plutôt qu’un combat théorique et politique plus important. Dans le même contexte, une autre lettre nous permet de constater que ce combat pour l’autonomie mené par Ambroise n’est pas un simple opportunisme d’occasion mais bien le cœur de ses objectifs longs termes pour l’Église. Il s’agit de la lettre 76 adressée à sa sœur, Marcelina, qui semble, selon ses dires, se préoccuper de l’avenir de l’Église et de l'épiscopat de Milan. Cette source est donc bien moins « officielle » qu’un texte qui doit être lu par un empereur, elle est donc à même de refléter avec plus de justesse sa pensée. Et même s’il est possible que l’évêque de Milan mente sur certains sujets, il est difficile de contrôler la véracité de l’ensemble des faits,  il n’exprime finalement ici que son idéal théorique à propos des relations entre l’Église et le pouvoir impérial. Il n’a plus besoin de camoufler le fond de sa pensée derrière des formules de politesse. Le ton est alors bien plus direct, ne laissant pas de doute à sa volonté d'indépendance de l’\latin{auctoritas} de l’Église :

\bigskip

\begin{quote}
    « Bref, on me fait dire : « Livre la basilique ». Je réponds : « Il ne m’est pas permis de la livrer et il n’est pas dans ton intérêt, empereur, de la recevoir. Alors que tu ne peux en droit mettre la main sur la maison d’un particulier, penses-tu que tu peux te saisir de la maison de Dieu ? » On prétend que tout est permis à l’empereur car tout lui appartient. Je ŕeponds : « Ne charge pas ta conscience , empereur, en pensant que tu disposes de quelque droit impérial sur les choses divines. [...]\footnote{« \latin{Mandatur denique : « Trade basilicam. » Respondeo : « Nec mihi fas est tradere nec tibi accipere, imperator, expedit. Domum priuati nullo potes iure temerare, domum Dei existimas auferendam ? » Allegatur imperatori licere omnia ipsius esse uniuersa. Respondeo : « Noli te grauare imperator, ut putes te in ea quae diuina sunt imperiale aliquod ius habere.}~». \cite[\nopp 76,19]{ambroise_lettres}.}~»
\end{quote}

\bigskip

Ambroise met ici en place l’argument d’une autorité ecclésial fondée sur son espace public. L’évêque de Milan est conscient de la nécessité d’englober toutes les formes d’autorités possibles pour légitimer la nouvelle place de l’Église chrétienne. Replacer la figure de Dieu derrière la symbolique des bâtiments religieux comme les églises ou les basiliques fait partie de l’ensemble du processus permettant une reconnaissance politique par le pouvoir impérial. La séparation bien connue, aussi bien dans la pensée chrétienne que romaine, entre les biens publics et les biens sacrés permet de continuer dans le sens de la distinction nette des domaines d’actions des deux autorités, définissant toujours plus la relation entre l’Église et le l’empereur comme deux autorités complémentaires pour l’empire. La dernière phrase du passage que j’ai choisi de citer permet de clarifier une fois pour toute le fond de la pensée ambrosienne à travers l’autonomie cléricale et la liberté politiques des figures comme Ambroise.

\bigskip

Il est donc important de retenir qu’il n’y a dans cette pensée aucune soumission d’une autorité envers une autre. Ambroise ne cherche pas à mettre l’empereur sous la puissance de l'Église, mais ne veut pas voir non plus l’empereur interférer de son plein gré dans les décisions épiscopales. L’indépendance recherchée devient un prérequis nécessaire à l’entretien de l’\latin{auctoritas} des évêques et à leur place dans le gouvernement impérial, comme je l’ai abordé précédemment. Il est désormais important de comprendre la façon dont l’évêque de Milan veut que l’empereur se comporte devant cette autorité, en regardant notamment les droits d’interventions que lui laisse Ambroise, ou les priorités politiques qui lui sont attribuées.

\bigskip

\subsection{L'empereur face à cette \latin{auctoritas}}

\subsubsection{La priorité du christianisme sur la société romaine}

Ambroise s’efforce, notamment à travers ses communications avec les empereurs, de construire une indépendance de l’autorité et des champs d’action de l’Église, dans l’objectif de se constituer une défense contre les empiètements du pouvoir impérial. Mais derrière cette action politique ne se cache pas l’envie de rendre les domaines de la religion inaccessibles, bien au contraire l’évêque de Milan exige que la \latin{potestas} de l’empereur se mette activement, et de façon prioritaire, au service de la foi. Ambroise théorise assez clairement une hiérarchie des domaines d’action où la cause de la religion doit systématiquement l’emporter sur la rigueur du droit civil et même sur le maintien de l'ordre public.

\bigskip

Ce principe de priorité est tout d’abord évoqué dans ses lettres à Valentinien II en 384, lors d’un moment de tension au sujet de la restitution de l’autel de la Victoire dans la curie sénatoriale. Le jeune empereur fait ici face à un dilemme à la fois moral et politique qui rentre exactement dans le questionnement politique d’Ambroise. La \latin{relatio} de Symmaque, porte-parole de la frange païenne du Sénat qui vise le retour de la Victoire, démontre à Valentinien l'importance des traditions romaines et de la stabilité de cette institution ayant des membres toujours majoritairement païens. L’intervention d’Ambroise, d'abord en prévention avec l’\latin{Epistolae} 72, puis de nouveau et avec plus de précision après la lecture de la \latin{relatio}, à travers l’\latin{Epistolae} 73, a pour objectif de convaincre l’empereur que le choix en faveur de la foi chrétienne, et donc du rejet de la délégation du Sénat, est toujours le choix juste et même le seul choix valable :

\bigskip

\begin{quote}
    « Quand il s’agit de décider d’une affaire militaire, il convient de prendre l’avis d’un homme exercé aux combats et d’adopter son conseil ; quand il s’agit de religion, songe à Dieu. On ne fait aucun tort à qui on préfère le Dieu tout-puissant. Cet homme a son opinion, vous ne le contraignez pas à pratiquer malgré lui un culte dont il ne veut pas. Que le même droit vous soit accordé, empereur, [...]\footnote{« \latin{Si de re militari est consulendum, debet exercitati in proeliis uiri expectari sententia, consilium comprobari ; quando de religione tractatus est, Deum cogita. Nullius iniuria est cui Deus omnipotens antefertur. Habet ille sententiam suam. Inuitum non cogitis colere quod nolit. Hoc idem uobis liceat, imperator, [...]}~». \cite[\nopp 72, 7]{ambroise_lettres}.}~»
\end{quote}

\bigskip

Le parallèle entre l’expertise militaire et les questions religieuses a pour but de placer Dieu au-dessus de toute réflexion politique. Ambroise amène cette idée presque sous la forme d’un rappel à l’ordre : songe à Dieu. L’ensemble de la politique dans l’Empire doit se faire derrière le regard de Dieu et en ayant cette foi comme seul véritable chemin de justice. Ambroise amène Valentinien II à ce constat par un détour intéressant : il transforme la foi personnelle du souverain en un acte politique, faisant de la préférence du Dieu unique une obligation à l’échelle de l’Empire. De la même manière que Symmaque, puisque c’est de lui dont il est question dans le « cet homme », a le droit d’apporter son avis sur la présence de l’autel de la Victoire dans le Sénat, l'empereur a le droit, et donc ici même le devoir, de préférer Dieu dans sa prise de décision. La précision « on ne fait aucun tort » ne s’applique pas seulement pour le sénateur ni pour une personne seule, mais bien pour l'ensemble des décisions impériales. La fidélité à Dieu empêche toute accusation d’injustice envers le souverain, du moins selon la vision ambrosienne. L’évêque, par ce passage, introduit l’idée d’une priorité de la religion sur les autres questions : il légitime ici, par avance, toutes les décisions pouvant aller en faveur de l’Église. Une fois encore, Ambroise amène le jeune empereur dans une direction qui lui est favorable, en faisant appel à l’autorité de Dieu et de la foi chrétienne.

\bigskip

Cette approche politique, nouvellement mise en valeur par l’évêque de Milan, trouve son application la plus concrète et la plus stricte au cœur de l'\latin{Epistolae} 74 sur l’affaire de Callinicum, qui cette fois concerne l’empereur Théodose. Dans cet événement dont nous avons déjà parlé précédemment\footnote{Voir ???}, la logique administrative et impériale de l’Empire romain semble respectée à travers la décision prise par Théodose. Celui-ci décide de punir l’évêque de Callinicum responsable du mouvement ayant incendié la synagogue. Ainsi, même s’il s’agit d’un acte allant contre l’Église chrétienne, il permet un maintien de l’ordre cohérent, et donc la continuité dans l’ordre public. Mais Ambroise ne voit pas cette décision comme une aide à la société romaine, mais bien comme un affront à la foi et à Dieu. Bien qu’il agisse après la décision impériale, il se permet, dans sa lettre envoyée au souverain, des paroles virulentes ne laissant aucun doute sur la position de l’évêque. Dans sa thèse sur Ambroise, Jean-Rémy Palanque émet même l’hypothèse qu’il puisse s’agir d’une façon pour Ambroise « d’humilier l’empereur d’Orient dès son arrivée en Italie\autocite[374]{palanque_ambroise}.~». Bien qu’il soit impossible de connaître toute l’ampleur de la réflexion d’Ambroise derrière sa lettre 74, il est évident qu’elle vise en partie à imposer, dans la politique impériale, un principe de priorité religieuse qui donne à la foi chrétienne une position de domination constante face aux autres préoccupations de l’Empire. L'aboutissement de cette pensée se fait ressentir par cette phrase :

\bigskip

\begin{quote}
    « Mais c’est la considération de l’ordre public qui te préoccupe, empereur. Qu’est-ce qui a le plus d’importance, le prétexte de l’ordre public ou la cause de la religion ? La répression doit céder devant la dévotion.\footnote{« \latin{Sed disciplinae te ratio, imperator, mouet. Quid igitur est amplius, disciplinae species an causa religionis ? Cedat oportet censura deuotioni.}~». \cite[\nopp 74, 11]{ambroise_lettres}.}~»
\end{quote}

\bigskip

Le vocabulaire employé a son importance : en qualifiant le maintien de l'ordre public de « prétexte », Ambroise soumet l'un des concepts clés du pouvoir impérial à Dieu. L'ordre dans la société n'est, selon l'évêque, plus légitime d'être maintenu s'il va à l'encontre du christianisme, peu importe sa forme. Dans cet extrait, on peut constater une hiérarchie stricte, que cherche à installer Ambroise, qui place « la cause de la religion », donc le fait d'agir en faveur de la foi chrétienne, au-dessus de toute autre considération politique. Le terme « prétexte » dévalorise l'argument politique de l'empereur, soumis à sa religion et principalement ici à l'autorité d'Ambroise. L'opposition entre répression et dévotion insiste sur cette forme de soumission de la moralité civile à la foi chrétienne. Théodose se voit accusé d'avoir utilisé son droit fondamental de punition contre la dévotion. Ainsi, avec Ambroise, la suprématie chrétienne est prête à entacher la paix sociale, en favorisant la lutte religieuse à l'entente dans l'Empire.

\bigskip

Il ne semble donc, dans la pensée ambrosienne, n'y avoir pas plus important que la sauvegarde du christianisme et de tout ce qui en découle. Plusieurs historiens se sont penchés sur cette exigence qui amène une nouvelle conception de la fonction impériale. C'est en particulier le cas de Michael Müller dans son article sur le traitement des affaires politiques et religieuses par Ambroise, dans lequel il s'attarde sur le rôle de l'empereur : « Cet argument est justifié par l’ébauche d'une conception du pouvoir dans laquelle incombe à l’empereur, parce qu’il est à la tête de l’Empire romain, une obligation particulière, celle, en tant que premier serviteur du Dieu des chrétiens, de protéger et de propager la foi de ce même Dieu dans le monde\autocite[???]{muller_conflits}.~». Ce qui rejoint fortement l'idée de Thomas Ring qui note qu'Ambroise ne tolère aucune distinction entre la confession privée et l'action politique\autocite[124]{ring_auctoritas}. C'est-à-dire que l'empereur, par le simple fait d'être chrétien, se doit d'intervenir politiquement en faveur du christianisme et de l'Église. Il ne peut pas se cacher derrière une relation privée à la foi, ni agir de façon impartiale sur les questions religieuses en se cachant derrière son rôle politique. Pour Ambroise, l'empereur est un représentant de Dieu, qui doit agir comme tel. Enfin, Jean-Rémy Palanque résume parfaitement cette fusion des devoirs : l'empereur, soumis aux lois morales, conserve sa \latin{potestas}, mais doit l'utiliser pour servir la cause de Dieu en protégeant notamment l'Église. L'historien précise lui aussi qu'Ambroise ne distingue pas l'empereur et la fonction impériale, faisant de sa foi personnelle, une arme de politique générale\autocite[355]{palanque_ambroise}.

\bigskip

L'analyse de la théorie politique ambrosienne démontre donc que l'autonomie de l'Église et la montée en puissance d'une autorité parallèle ne signifie pas pour le pouvoir impérial un relâchement au sujet des questions religieuses. Au contraire, non seulement celles-ci continuent de se montrer majeures, mais elles sont plus que jamais résolues par la domination de la foi chrétienne, prioritaire sur toutes raisons morales. Charles Norris Cochrane le visualise particulièrement bien dans son ouvrage sur la pensée chrétienne, il explique notamment que la défense de la liberté par Ambroise est cadrée par une «~note d'autoritarisme typiquement romaine\autocite[355]{cochrane_christianity}.~». On peut alors penser que la liberté qu'il veut est celle de l'évêque plutôt que celle du peuple. Un paradoxe semble tout de même apparaître : comment valoriser l'indépendance de l'Église tout en ayant besoin d'une sollicitation de l'empereur sur les questions religieuses ? Ambroise le résout par la nécessité d'un encadrement rigoureux des interventions impériales, ce que nous allons désormais examiner.

\bigskip

\subsubsection{L'encadrement des interventions impériales}

L'un des points majeurs de la réflexion politique d'Ambroise est la nécessité de collaboration constante entre le pouvoir de l'Église et celui de l'empereur. L'évêque de Milan, comme présenté plus tôt, instaure l'\latin{auctoritas} épiscopale au cœur du fonctionnement de l'Empire, tout en ayant conscience de l'importance de la \latin{potestas} impériale pour faire fonctionner la «~machine~» administrative chrétienne. Dans ce cadre, le plus important est de mettre en place des limites strictes du pouvoir de l'empereur dans les questions religieuses, sans pour autant interdire les interventions. Ambroise continue de marquer par son réalisme politique, et possède une véritable conscience des limites de l'autorité spirituelle des évêques, qui peut se retrouver démunie face aux agitations publiques, par exemple des groupes considérés par la foi nicéenne comme hérétiques. La paix prônée par l'Église chrétienne, ainsi que l'ambition d'hégémonie, doit donc se reposer en partie sur le pouvoir du souverain.

\bigskip

Cette reconnaissance de l'utilité du pouvoir civil est notamment exprimée dans sa correspondance avec Gratien, dans laquelle il ne se contente pas de féliciter l'empereur pour ses actions religieuses, mais où il valorise l'usage de la force contre les ennemis de la foi :

\bigskip

\begin{quote}
    « En effet, vous m'avez rendu le repos de l'Église, vous avez fermé la bouche, puisque vous ne pouviez pas fermer le cœur, des perfides, et cela, vous l'avez fait par une autorité qui ne doit pas moins à votre foi qu'à votre pouvoir.\footnote{« \latin{Reddidisti enim mihi quietem ecclesiae, perfidorum ora atque utinam et corda clausisti ; et hoc non minore fidei quam potestatis auctoritate fecisti.}~». \cite[\nopp ec. 12, 2]{ambroise_csel_82_3}. Traduction personnelle.}~»
\end{quote}

\bigskip

Cette citation est révélatrice du pragmatisme d'Ambroise : il admet les limites de son propre pouvoir face à ses opposants et semble faire appel à une forme de répression plus directe. L'évêque de Milan évoque son échec, ainsi que celui de Gratien, de « fermer les cœurs », c'est-à-dire de changer la foi des hérétiques. En revanche, il le remercie pour avoir « fermé la bouche », donc très certainement d'avoir mis en place une interdiction de parole publique et peut-être de rassemblement, aspect réalisable uniquement par la \latin{potestas}. Bien qu'il soit difficile de savoir par ce simple extrait les détails de ce qu'évoque Ambroise, il est tout de même possible de constater la satisfaction par l'évêque de l'utilisation du pouvoir temporel du souverain au service de l'Église catholique. L'action impériale n'est donc pas toujours vue comme une ingérence mais plutôt comme un complément indispensable à la mission épiscopale.

\bigskip

Pourtant, je l'ai pleinement démontré dans les premières parties de ce chapitre, Ambroise fait régulièrement face aux empereurs, et principalement quand il s'agit d'ingérence dans un domaine qui se doit d'être réglé par l'Église. C'est le cas de l'affaire de l'autel de la Victoire, ou également du conflit concernant le don d'une basilique de Milan aux ariens. Cette opposition provient notamment de sa connaissance de l'histoire récente de la politique religieuse de l'Empire, et donc de sa conscience de l'importance de séparer les sphères de pouvoirs et de pleinement contrôler la capacité d'intervention des empereurs dans les questions de foi. En effet, comme le rappelle notamment l'ouvrage \latin{The Cambridge history of Greek and Roman political thought}, à partir de Constantin, l'Église se retrouve dans des disputes théologiques qui poussent les empereurs, malgré la volonté d’union, à prendre position, notamment avec Constance. Les différents camps de la foi chrétienne étaient dans un questionnement constant entre faire appel à l’empereur pour les aider, ou au contraire, revendiquer une indépendance cléricale quand l’empereur était contre eux\autocite[658]{rowe_history}. La mise en place par Constance II du concile de Rimini en 359 est une forme d'apogée du pouvoir impérial sur le religieux. Ce concile définit un dogme différent de celui de Nicée qui est ensuite utilisé par les ariens comme preuve de la légitimité de leur foi. C'est donc entre autres par cet événement qu'Ambroise et les évêques nicéens se retrouvent à s'opposer fermement aux ingérences impériales provenant de leur volonté propre. Une règle se dessine alors dans la seconde moitié du IV\textsuperscript{e} siècle : l'action de l'empereur ne peut pas venir de sa seule initiative, mais seulement d'une demande de l'Église catholique.

\bigskip

Cette dynamique d'appel et d'exécution se retrouve notamment dans les lettres d'Ambroise faisant état du concile d'Aquilée\autocite[ec 4, 5 et 6]{ambroise_csel_82_3} aux trois empereurs Gratien, Valentinien et Théodose. Ambroise fait ici intervenir la lutte contre les partisans de Photin, évêque de Sirmium sans doute entre 344 et 351 puis ayant bénéficié du règne de Julien pour revenir rapidement vers 361, qui forment un groupe toujours considéré comme hérétique par les évêques catholiques. Ambroise en appelle à l'application des lois et au pouvoir civil des empereurs :

\bigskip

\begin{quote}
    « Quant aux Photiniens, dont vous avez déjà décidé par une loi précédente qu'ils ne devaient tenir aucune assemblée [...], nous demandons que, puisque nous avons appris qu'ils tentent encore de s'assembler dans la ville de Sirmium, votre Clémence ordonne d'interdire leurs réunions de nouveau. Cela permettra de marquer du respect d'abord à l'Église catholique, et ensuite à vos propres lois.\footnote{« \latin{Fotinianos quoque quos et superiore lege censuistis nullos facere debere conventus [...], petimus ut quoniam in Sirmiensi oppido adhuc conventus temptare cognovimus, clementia vestra interdicta etiam nunc coitione reverentiam primum ecclesiae catholicae, deinde etiam legibus vestris deferre iubeatis.}~». \cite[\nopp ec. 4, 12]{ambroise_csel_82_3}. Traduction personnelle.}~»
\end{quote}

\bigskip

Par le « nous demandons », on comprend qu'il s'agit d'une initiative ecclésiastique, sans doute de l'ensemble du concile qui cherche à activer un levier juridique pour résoudre un problème de foi. Cette requête s'avère très intéressante pour comprendre toute la profondeur de la pensée d'Ambroise sur ce sujet. Il n'est pas demandé aux empereurs de trancher sur le fond de l'hérésie photinienne, mais bien d'appliquer une sanction civile qui semble nécessaire au maintien de l'ordre. L'intervention impériale est alors légitimée par la demande épiscopale, mais également par le fait qu'elle ne vise pas la résolution d'un problème dogmatique par une autorité non compétente, ce qui avait été critiqué dans le concile de Rimini notamment. Il est intéressant de relever le parallèle entre l'initiative ambrosienne et la question sur ce qu'est le maintien de l'ordre. Nous l'avons vu, Ambroise fait passer la réussite de la foi chrétienne au-dessus de la paix civile, comme pour l'affaire de Callinicum. En revanche, il semble ici que les divergences dogmatiques à Sirmium aillent, selon l'évêque, contre l'ordre dans l'Empire. D'une façon générale, Ambroise impose donc sa façon de penser dans la politique impériale, avec plus ou moins de succès bien sûr.

\bigskip

Ambroise veut donc mettre entre les mains des empereurs un rôle de facilitateur religieux, également dans les questions que l'on pourrait qualifier d'administratives ou de logistiques. Palanque parle de l'Empire comme le « bras séculier de l'Église » et démontre qu'Ambroise veut une ratification aveugle et docile des décisions du clergé et non pas un jugement moral ou juridique de la part de l'empereur\autocite[372]{palanque_ambroise}. Le résultat du concile de Rimini et ses conséquences se trouvent de nouveau être le parfait exemple de la nécessité de séparer les sphères d'intervention. Mais nous le comprenons bien en lisant Ambroise, le pouvoir impérial doit répondre aux exigences de l'Église :

\bigskip

\begin{quote}
    « C'est pourquoi nous vous demandons, très cléments et chrétiens princes, de décider qu'un concile de tous les évêques catholiques se tienne également à Alexandrie, afin qu'ils traitent entre eux plus amplement et définissent à qui la communion doit être accordée et pour qui elle doit être réservée.\footnote{« \latin{Ideoque petimus vos, clementissimi et Christiani principes, ut et Alexandriae sacerdotum catholicorum omnium concilium fieri censeatis, qui inter se plenius tractent atque definiant quibus impertienda communio quibusve servanda sit.}~». \cite[\nopp ec. 6, 5]{ambroise_csel_82_3}. Traduction personnelle.}~»
\end{quote}

\bigskip

Les évêques catholiques cherchent à utiliser la puissance administrative de l'Empire pour faciliter la gestion de leurs problématiques. La répartition des rôles apparaît très nettement : les empereurs fournissent une sorte de cadre matériel civil en convoquant le concile, tandis que l'objet du débat reste hors de leur portée. Ambroise ne fait qu'utiliser la puissance «~logistique~» de l'Empire.

\bigskip

Enfin, comme à son habitude, Ambroise n'hésite pas à adapter son idéal à la réalité politique et sociale qu'il confronte. Ainsi, bien qu'il défende en théorie l'élection des évêques par le clergé et le peuple, pour éviter des problèmes comme l'affaire d'Athanase, il n'hésite pas à faire appel au pouvoir impérial pour résoudre des complications d'arbitrages dans l'élection entre deux candidats. C'est le cas par exemple dans l'\latin{Epistolae extra collectionem} 9 qu'il adresse à Théodose pour lui demander son soutien à la nomination de l'évêque d'Antioche. Ainsi, on constate que l'indépendance que recherche Ambroise n'est pas un dogme aveugle de toute considération politique et pratique. Tant que l'intervention impériale se fait dans le cadre d'une demande épiscopale, elle se légitime en elle-même.

\bigskip

La réflexion politique d'Ambroise se forme donc autour d'un équilibre en constant mouvement, entre véritable autonomie et besoin de collaboration. Il utilise volontiers le pouvoir impérial pour débloquer des situations morales ou dogmatiques difficiles mais ne tolère aucunement toute prise d'initiative personnelle. La \latin{potestas} de l'empereur chrétien semble devoir se mettre à la disposition de l'Église et n'agir, dans la religion, que sous l'angle d'une vision cléricale.

\bigskip

\subsubsection{Une limitation de la \latin{potestas} impériale ?}

L'évêque de Milan instaure tout au long de son épiscopat une position politique de l'Église de plus en plus imposante. Que ce soit par l'avènement d'une nouvelle autorité morale et juridique dans l'Empire, ou encore plus nettement par l'influence de cette Église et d'Ambroise lui-même vis-à-vis de l'empereur et des décisions impériales. L'ensemble de l'appareil gouvernemental de l'Empire se retrouve donc modifié. Évidemment ce changement n'est pas à relier uniquement à l'évêque de Milan, mais celui-ci joue un rôle crucial dans la théorisation et l'application d'une forme de théologie politique des acteurs de la foi chrétienne. Pour autant, Ambroise ne remet jamais en cause la légitimité du pouvoir politique. Nous avons pu le voir tout au long de ce mémoire, bien que réfléchissant longuement sur le gouvernement et le souverain, il ne se place nullement comme véritable opposant politique ni théoricien « révolutionnaire », mais bien comme une source d'influence dont il est conscient et qu'il utilise pour tirer le pouvoir civil dans la direction qu'il souhaite. Concernant l'instauration d'une nouvelle source d'\latin{auctoritas}, nous avons pu le constater principalement par des extraits tirés de ses lettres aux empereurs, Ambroise s'applique à établir une distinction nette des domaines d'action. Ainsi, semble-t-il que la limitation qu'il impose aux empereurs ne soit pas une attaque contre son pouvoir ni son statut, mais une définition précise des frontières que chaque institution ne doit pas franchir pour l'équilibre de l'Empire.

\bigskip

Ambroise n'hésite d'ailleurs pas à s'exprimer sur ce sujet, et ce pour rappeler sa place et le fait qu'il n'outrepasse pas les droits obtenus par l'Église. L'évêque de Milan est donc conscient de la difficulté d'imposer à tous sa vision politique de la gestion des affaires religieuses, et est prêt à faire face aux accusations. Le meilleur exemple de cette défense d'Ambroise de ses droits se trouve dans sa correspondance avec sa sœur Marcelline. De nouveau, il fait part des événements dans la lutte contre l'arianisme et expose sa position, que l'on peut croire sincère puisque la lettre n'est pas à destination d'un empereur. Alors qu'il est accusé de se comporter en « tyran » en défendant sa basilique contre les ordres impériaux, il se défend en rappelant la nature historique et spirituelle du sacerdoce et en appelle aux autorités de l'Ancien et du Nouveau Testament :

\bigskip

\begin{quote}
    « Dans l’ancien droit les prêtres conféraient les pouvoirs, ils ne se les arrogeaient pas, et on disait habituellement que les empereurs désiraient davantage le sacerdoce que les prêtres le pouvoir impérial, le Christ a fui pour ne pas devenir roi\footnote{« Et Jésus, sachant qu'ils allaient venir l'enlever pour le faire roi, se retira de nouveau sur la montagne, lui seul. » Jean 6, 15.}. [...] J’ai ajouté que les prêtres n’ont jamais été des tyrans, mais ont eu souvent à souffrir des tyrans.\footnote{« \latin{Veteri iure a sacerdotibus donata imperia, non usurpata, et uulgo dici quod sacerdotes, Christus fugit, ne rex fieret. [...] Addidi quia numquam sacerdotes tyranni fuerunt sed tyrannos saepe sunt passi.}~». \cite[\nopp 76, 23]{ambroise_lettres}.}~»
\end{quote}

\bigskip

En évoquant « l'ancien droit », Ambroise fait référence aux royautés de l'Ancien Testament, où les figures de Saül, David et d'autres se sont faits rois par la légitimité apportée par les prêtres et prophètes. Par cette référence, l'évêque de Milan rappelle qu'un lien a toujours existé entre les représentants de la foi et le pouvoir civil, légitimant ainsi l'\latin{auctoritas} épiscopale comme une forme d'autorité ayant toujours existé. Puis, en plus d'exposer ses droits en tant qu'évêque, il se défend de toute ingérence cléricale envers le pouvoir impérial. Il s'appuie cette fois sur l'autorité du Christ : si même lui a fui le pouvoir temporel, alors pourquoi les évêques agiraient différemment ? Ce passage est une démonstration de l'ensemble de la réflexion ambrosienne sur ce sujet. Il maîtrise son sujet et sait se défendre, ce qui lui permet une plus grande liberté d'action. Une fois encore, il utilise sa théorie politique et religieuse pour l'exploiter dans des cas d'actions concrets. Il se permet même, du moins c'est ce qu'il pense puisqu'il l'exprime à Marcelline, une critique du pouvoir impérial et des interventions dans le monde religieux par leur propre initiative, rappelant ici certainement l'exemple d'Achab, souvent présenté comme un tyran impie pour avoir honoré Baal et avoir persécuté les représentants du Dieu unique de la chrétienté et du judaïsme. L'idée des prêtres souffrant des tyrans vise également sûrement à rappeler les multiples persécutions des empereurs païens envers les communautés chrétiennes. Ainsi, Ambroise place pratiquement l'Église comme une victime potentielle du pouvoir impérial plutôt que comme un opposant politique.

\bigskip

Dans le même contexte de l'affaire de 386 sur les basiliques de Milan, Ambroise défend avec virulence les positions catholiques contre les demandes impériales de restitution d'une église aux ariens. Cette position l'amène entre autres à être critiqué par les représentants de l'empereur comme un opposant à l'empereur, et donc le signe d'une Église dans la confrontation de la \latin{potestas} impériale. Ce à quoi il répond fermement dans son \latin{Sermo contra Auxentium} :

\bigskip

\begin{quote}
    « Que je me sois exprimé avec la déférence due à l'empereur, personne ne peut le nier. Comment montrer plus de déférence qu'en disant que l'empereur est le fils de l'Église ? En disant cela, ce n'est pas une faute de le dire, nous lui sommes loyaux. L'empereur, en effet, est au-dedans de l'Église, il n'est pas au-dessus de l'Église ; un bon empereur, en effet, demande le secours de l'Église, il ne le rejette pas. Tout cela, si nous le disons avec humilité, nous le déclarons avec fermeté.\footnote{« \latin{Quod cum honorificentia imperatoris dictum nemo potest negare. Quid enim honorificentius quam ut imperator ecclesiae filius esse dicatur ? Quod cum dicitur sine peccato dicitur, cum gratia dicitur. Imperator enim intra ecclesiam non supra ecclesiam est ; bonus enim imperator quaerit auxilium ecclesiae, non refutat. Haec ut humiliter dicimus ita constanter exponimus.}~». \cite[\nopp 75A, 36]{ambroise_lettres}.}~»
\end{quote}

\bigskip

De nouveau, Ambroise ne se prive pas de rappeler l'interdiction des ingérences impériales dans les sphères d'action de l'Église par cette injonction « L'empereur, en effet, est au-dedans de l'Église, il n'est pas au-dessus de l'Église. » Cette façon de présenter le pouvoir impérial est d'ailleurs une définition récurrente de l'empereur chrétien selon Ambroise. Dans sa traduction de l'\latin{Epistolae}, Gérard Nauroy fait d'ailleurs référence à l'une des premières rencontres entre Ambroise et Théodose, rapportée par Sozomène, où « l'évêque lui interdit de prendre place dans le chœur de l'Église, lui assignant une place au premier rang des fidèles.\footnote{Gérard Nauroy, note 1, page 432. \cite[\nopp 75A, 36]{ambroise_lettres}.}~». Il y a toujours, au sujet de la foi, cette délimitation complexe du pouvoir de chacun. L'empereur ne peut pas user de sa \latin{potestas} sur la religion, mais les évêques ne sont pas pour autant supérieurs au pouvoir impérial. Mais plus important encore, cette citation met en avant deux aspects cruciaux dans la relation d'Ambroise au pouvoir impérial. Premièrement la mise en avant constante d'une collaboration entre les deux institutions : l'idée que l'empereur doive s'appuyer sur l'autorité et la connaissance de l'Église pour consolider son pouvoir et sa gestion de l'Empire, et inversement que l'Église a besoin d'une autorité armée pour faciliter le contrôle de la foi chrétienne. Deuxièmement l'importance de la loyauté des évêques envers les empereurs chrétiens. Ce principe évite toute possibilité d'opposition brutale de la part du monde épiscopal : limiter le pouvoir de l'empereur dans le sacré n'est pas symbole d'une attaque envers son rôle politique. Finalement, il est possible de voir la création d'une \latin{auctoritas} par Ambroise comme la continuité de la politique de Gratien, qui décide en 382 de renoncer au titre de \latin{Pontifex Maximus}, et par là même à bon nombre de ses droits sur la sphère religieuse. La restriction des compétences des empereurs concernant les décisions religieuses, que nous avons pu longuement constater, n'est pas une soumission juridique à l'Église ni à Ambroise, mais une forme de désacralisation de la fonction impériale, qui se voit limitée par les règles de sa propre foi.

\bigskip

\subsection*{Conclusion}

Au terme de cette analyse de la séparation des institutions, il apparaît assez clairement que la « nouvelle doctrine » évoquée par Giuseppe Visonà n'est pas une simple rhétorique liée à un besoin d'action dans des circonstances précises, mais bien une véritable définition politique et théologique de la société romaine, et notamment de son gouvernement. Ambroise réussit à théoriser puis faire appliquer ses réflexions, principalement par le biais de ses communications aux empereurs, et réussit ainsi à placer l'\latin{auctoritas} impériale et sénatoriale entre les mains de l'Église chrétienne. Le prestige moral et juridique de l'\latin{auctoritas} permet à l'Église de consolider son influence sur le pouvoir sans pour autant développer de véritables compétences politiques.

\bigskip

L'apport principal d'Ambroise réside dans la création d'une expertise exclusive aux évêques sur les questions de foi. La \latin{potestas} impériale se place, dans ce cadre, derrière une frontière clairement définie empêchant l'empereur d'intervenir sans tomber dans la tyrannie ou le sacrilège. En tant qu'« homme de l'Église », le souverain se retrouve dans l'Église chrétienne, sans privilège lié à son statut civil. Il n'est alors plus en légitimité d'intervenir dans la sphère religieuse sur sa propre initiative, plaçant ainsi le monde épiscopal dans une véritable position autonome. Le pouvoir impérial n'est donc pas véritablement réduit, mais sa souveraineté est limitée par les exigences de la foi. Une logique de collaboration s'installe entre les protagonistes, mais toujours sous le regard de l'Église. Ambroise est bien évidemment conscient du rôle majeur joué par les empereurs dans le maintien et le renforcement de la foi chrétienne, mais il cherche à valoriser cette \latin{auctoritas} et vise ainsi à contrôler les interventions impériales.

\bigskip

La dynamique entre l'Église et le pouvoir impérial théorisée par Ambroise s'inscrit, en plus de l'aspect institutionnel, dans les relations plus personnelles entre les évêques et les empereurs. La figure d'Ambroise, en tant qu'évêque de la capitale occidentale de l'Empire, prend donc une place toujours plus importante auprès des empereurs, mettant au premier plan la question de l'\latin{auctoritas} personnelle. Ce qui nous amène à observer la mise en œuvre concrète du rôle politique que nous avons développé, à travers l'action des évêques, et plus particulièrement la relation personnelle que développe Ambroise avec les souverains contemporains.

\section{Construction et utilisation d'une \latin{auctoritas} personnelle}

\subsection*{Introduction}

Si la définition des domaines de compétence et le développement d'une autorité épiscopale permettent à l'Église de s'affirmer comme une institution autonome au sein du gouvernement impérial, ce n'est pas pour autant suffisant pour garantir l'influence des évêques sur les décisions du souverain. Pour peser réellement dans la politique de l'Empire, l'évêque ne peut se contenter des affaires religieuses : il doit s'imposer comme un interlocuteur indispensable au cœur même du pouvoir. C'est ce qu'Ambroise a particulièrement bien compris, en cherchant à développer son influence pour mettre en œuvre ses différentes réflexions et théories sur la pratique du pouvoir. Cette deuxième partie du chapitre s'attache donc à analyser la construction d'une \latin{auctoritas} personnelle par Ambroise, une autorité qui dépasse la simple fonction cléricale pour devenir une composante essentielle du pouvoir impérial.

\bigskip

Chez Ambroise, chacune de ses interventions est calculée selon sa réflexion personnelle. Ainsi, il questionne et théorise toujours sa façon d'agir, jusqu'à son dialogue avec les empereurs. Il cherche à donner une dimension politique à ses interventions religieuses et, inversement, un aspect clérical à ses actions diplomatiques. Par cette approche, il parvient à instaurer une norme dans la pratique du pouvoir : le conseil régulier de l'évêque doit être vu comme obligatoire au sein d'un pouvoir chrétien et non plus comme une ingérence extérieure. De fait, Ambroise se présente lui-même en tête de file de ce modèle d'un évêque conseiller politique, lui permettant ainsi une place majeure aux côtés des souverains.

\bigskip

Dans l'objectif d'analyser toutes les formes de l'\latin{auctoritas} personnelle développée par l'évêque de Milan, nous étudierons d'abord les outils méthodologiques et rhétoriques qu'il utilise pour conforter son propos à travers le rappel de la \latin{libertas dicendi} et l'utilisation de sa popularité pour s'assurer une proximité avec les empereurs chrétiens. Dans un second temps, nous observerons la mise à l'épreuve concrète de cette \latin{auctoritas} à travers un rôle politique grandissant pour Ambroise ou encore l'aboutissement d'un contrôle de la politique impériale par l'exemplarité morale définie par l'idéal chrétien d'Ambroise.

\bigskip

\subsection{L'instauration du rapport de force : entre obligation morale et liens personnels}

\subsubsection{La « \latin{libertas dicendi} » : entretenir le dialogue avec les empereurs}

L'étude ambrosienne sur l'\latin{auctoritas} nous a permis de comprendre les fondements institutionnels de l'autorité épiscopale ainsi que la mise en place d'une nouvelle relation entre le pouvoir impérial et l'Église. Mais l'ensemble de ce cadre politique ne peut s'exercer qu'à travers un dialogue appronfondi entre les évêques et l'empereur. Le coeur de cette dynamique se trouve dans ce qu'Ambroise nomme la « \latin{libertas dicendi} », ce droit à la parole ancré dans la tradition romaine et réutilisé par Ambroise comme un principe chrétien. Pour l'évêque de Milan, cette liberté de parole n'est pas un simple privilège que l'Église revendique mais bien une obligation qui permet de mettre clairement l'évêque dans une position de conseiller légitime face au pouvoir. Cet aspect se place dans la continuité du développement d'une autonomie cléricale dans laquelle Ambroise théorise les droits et devoirs des évêques pour définir leur rôle et leur fournir un poid important dans la balance politique. C'est ce qu'évoque Müller dans cet extrait : Au fur et à mesure de ses lettres, il travaille à l’élaboration d’une conception chrétienne du pouvoir qui confère au directeur de conscience la tâche, tant dans la défense de l’intérêt personnel de son souverain que dans celle de la prospérité de l’État, de guider l’empereur de ses conseils dans les décisions que celui-ci doit prendre, de lui rappeler ses devoirs de bon chrétien et, le cas échéant, de le rappeler à l’ordre\autocite[219]{muller_conflits}.

\bigskip

L'objectif principal de cette réflexion politique pour l'évêque de Milan est d'augmenter l'autorité et l'influence de l'Église, et par la même occasion la sienne pour se retrouver avec les mains de plus en plus libre dans l'Empire. Sa lutte doit alors se concentrer sur un point précis : éviter la censure que peuvent imposer les empereurs aux différents évêques. C'est pour cette raison qu'il est question d'obligation dans la liberté de parole : l'intervention de l'évêque, comme conseiller impérial, se développe comme un point à part entière de son rôle dans l'Empire. L'ensemble de cette théorie prend évidemment naissance dans l'action d'Ambroise, notamment lors de son intervention dans l'affaire de Callinicum, risquée puisque allant à l'encontre du choix de l'empereur. Deux passages de sa lettre à Théodose nous permettent d'établir avec précision sa pensée sur le sujet du dialogue entre évêques et empereurs. Tout d'abord avec cette idée qui semble relier la parole des évêques à la légitimité de l'empereur d'exercer son pouvoir :

\bigskip

\begin{quote}
    «~Mais il ne convient ni à un empereur de refuser la liberté de parler ni à un évêque de ne pas dire ce qu’il pense. [...] Chez un évêque aussi, rien n’est plus dangereux devant Dieu, ni plus honteux devant les hommes, que de ne pas déclarer librement ce qu’il pense.\footnote{« \latin{Sed neque imperiale est libertatem dicendi negare neque sacerdotale quod sentiat non dicere. [...] Nihil etiam in sacerdote tam periculosum apud Deum, tam turpe apud homines quam quod sentiat non libere denuntiare.}~». \cite[\nopp 74,2]{ambroise_lettres}.}~»
\end{quote}

\bigskip

Et quelques lignes plus loin l'explication de ce à quoi doit aboutir la liberté de parole :

\bigskip

\begin{quote}
    «~Je préfère donc, empereur, partager avec toi le bien plutôt que le mal, c’est pourquoi le silence de l’évêque doit déplaire à ta Clémence, mais sa franchise lui plaire. [...] Je ne me mêle pas en importun de ce qui ne me regarde pas, je ne m’ingère pas dans les affaires d’autrui, mais j’obéis à mon devoir, je me soumets aux commandements de notre Dieu.\footnote{« \latin{Malo igitur, imperator, bonorum mihi esse te cum quam malorum consortium et ideo clementiae tuae displicere debet sacerdotis silentium libertas placere. [...] Non ergo importunus indebitis me intersero, alienis ingero, sed debitis obtempero, mandatis Dei nostri oboedio.}~». \cite[\nopp 74,3]{ambroise_lettres}.}~»
\end{quote}

\bigskip

Plusieurs points majeurs de la pensée ambrosienne sont à relever dans ces citations. Les écrits de l'évêque de Milan oscillent en permanence entre réflexion politique large et rhétorique utile dans un cadre bien précis, les deux se mélangeant souvent dans ses correspondances. Une chose est certaine, il ancre en permanence ses besoins utilitaires dans des théories politiques précises qu'il cherche à appliquer à grande échelle. Ainsi dans cette lettre ce n'est pas seulement avec sa personne, pourtant proche des empereurs et évêque particulièrement important, qu'il essaie d'introduire l'idée de la contestation épiscopale, mais bien avec l'ensemble des ministres de la foi. Il légitime par ce biais l'ensemble de l'\latin{auctoritas} de l'Église et en appel à ce droit de parole. Dieu est invoqué comme source d'autorité incontestable : si il est permis par Dieu de parler librement, alors l'empereur n'a pas le pouvoir de s'y opposer. Mais surtout, Ambroise légitime sa prise de parole par des références constantes aux Écritures. Nous l'avons déjà évoqué, l'évêque de Milan utilise les textes de la Bible comme un traité politique qui peut et doit servir d'exemple à la société chrétienne romaine. Ainsi, dans les deux passages cités, il s'appuie sur des références scripturaires précises, telles que le Psaume 118 : « Devant les rois je parlerai de ton témoignage et je n'aurai nulle honte.\footnote{Psaume 118, 46.} » ou la deuxième lettre à Timothée : « Prêche la parole, insiste en toute occasion, favorable ou non, reprends, censure, exhorte, avec toute douceur et en instruisant.\footnote{2 Timothée 4, 2.}~». Le silence se présente donc comme une faute morale et un manquement aux exigences du rôle d'évêque.

\bigskip

Mais cette « \latin{libertas dicendi} ne doit pas pour autant être vu comme de la contestation politique ou de l'ingérence. Tout comme le développement de l'\latin{auctoritas} épiscopale, la parole et le conseil de l'évêque ne dépasse pas chez Ambroise la sphère d'action de l'Église, et donc les affaires religieuses. La mission principale d'un évêque, et c'est ce que nous allons étudier tout au long du reste de ce chapitre, reste d'empêcher autrui de péché, et à plus forte raison si cet évêque est proche de l'empereur. La deuxième citation de l'\latin{Epistolae} 74 permet à Ambroise de rassurer Théodose sur ses intentions. Il utilise cette fois la première personne pour montrer qu'il n'intervient que dans un cadre légitime, en phase avec ses droits. En réalité, le fait même qu'il soit obligé de théoriser son intervention et de la replacer dans une pensée politique plus large faisant appel à la liberté de parole des évêques, montre bien qu'il n'est pas rassuré et donc qu'il se place aux limites de son autorité. Bien qu'il s'agisse d'une réflexion politique pleinement aboutie, elle est ici et comme souvent utilisée comme une arme d'action immédiate.

\bigskip

Justemment, pour sortir du cadre « utile » de la pensée ambrosienne, il est intéressant de piocher une citation issue de son commentaire sur certains Psaumes. Ambroise apparaît comme conscient de la nécessité de modérer cette liberté des évêques, afin qu'elle se fasse dans le cadre le plus utile et juste possible :

\bigskip

\begin{quote}
    «~Tu vois donc qu'il ne faut pas que les prophètes de Dieu ou les prêtres fassent injure aux rois à la légère, s'il n'y a pas de fautes assez graves pour qu'ils doivent être repris. Mais là où les fautes sont plus graves, il ne semble pas que le prêtre doive les épargner, afin qu'ils soient corrigés par de justes réprimandes.\footnote{« \latin{Vides ergo, quia regibus non temere vel a prophetis Dei vel a sacerdotibus facienda iniuria sit, si nulla sint graviora peccata in quibus debeant argui. Ubi autem peccata graviora sunt, ibi non videtur a sacerdote parcendum, ut iustis increpationibus corrigantur.} » \cite[\latin{Enarratio in psalmum} 37,43. Page 172]{ambroise_csel_64}.}~»
\end{quote}

\bigskip

La figure de David, puisqu'Ambroise l'utilise comme auteur du Psaume 37, permet de parler aussi bien aux empereurs qu'aux évêques. Il tire ainsi des concepts s'adressant aux deux institutions et utilise son commentaire comme outil de réflexion politique qui doit s'appliquer dans l'Empire. D'une façon générale, ses homélies sur les Psaumes, « tout en étant attentive à marquer leur sens messianique, sont soucieuses d'interprétation morale liée concrètement à la situation eccleśiale et politique du moment.\autocite[223]{quasten_initiation}~». Par cette citation, il limite concrètement les raisons pouvant pousser aux interventions épiscopales, ce qui a pour effet de valoriser considérablement ces fameuses interventions qu'il qualifie lui même de « nécessaire réprimande ». Il est difficile de dater avec précision les homélies sur douze des Psaumes de David, mais Palanque semble indiquer que ce commentaire est sans doute prononcé autour de 389. Ce qui signifie qu'Ambroise écrit ce passage après l'affaire de Callinicum. Bien qu'il ne s'adresse pas aux empereurs, il accentue donc la théorie politique de liberté de parole en placant l'intervention d'Ambroise auprès de Théodose comme une correction liée à une « faute grave ».

\bigskip

Dans ses textes destinés à son clergé, Ambroise appuie tout de même bien sur les règles qui découlent du rôle de conseiller. Il joue donc grandement sur le fait d'encadrer les façons d'intervenir auprès des empereurs, afin de rendre ces moments plus fort et impactant, comme ici dans le deuxième volume de son \latin{De Officiis} :

\bigskip

\begin{quote}
    «~Aussi celui qui veut donner conseil à autrui doit être tel qu’il fournisse en lui-même, aux autres, un modèle pour « l’exemple des bonnes oeuvres, dans sa doctrine, dans sa chasteté, dans son sérieux » que sa conversation soit salutaire et irréprochable, son conseil utile, sa vie belle et son avis convenable.\footnote{«~\latin{???}.~». \cite[\nopp II, 86]{ambroise_devoirs_2}.}~»
\end{quote}

\bigskip

Par ce texte ou celui sur les Psaumes, on peut donc constater qu'il n'y a pas chez Ambroise de liberté sans restriction. Le rôle politique qu'il façonne chez les évêques doit suivre une réussite personnelle dans la foi, la morale et la justice afin d'apporter aux détenteurs du pouvoir impérial les conseils les plus à même d'aider à la prospérité de l'Empire. Bien que l'évêque de Milan semble tout faire pour développer son propre pouvoir et autorité au sein du gouvernement impérial\footnote{C'est principalement le coeur de la suite de ce chapitre}. Il ne cherche pas à rendre l'Église chrétienne toute puissante, mais simplement à créer une autorité religieuse capable d'agir concrètement pour le bien de l'Empire.Ainsi, la revendication d'une \latin{libertas dicendi} marque un point central dans la pensée politique ambrosienne.

\bigskip

Ces droits à la parole et aux conseils peuvent également servir à encadre les actions impériales pour les guider vers une justice commune. Dans son court passage sur l'évêque de Milan, Giuseppe Zecchini analyse la réflexion d'Ambroise sur le pouvoir impérial et se penche notamment sur le risque de voir le souverain devenir un tyran. Face à ce risque, la \latin{libertas dicendi} intervient comme la « première forme et expression de toute liberté politique\autocite[165]{zecchini_pensiero_politico}~». Avec Ambroise, la parole de l'évêque se place comme un outil de contrôle du pouvoir impérial. Zecchini note que, tout comme l'\latin{auctoritas}, cette fonction semble passer des mains des sénateurs aux « \latin{sacerdoti di Cristo} », les prêtres du Christ. Le Sénat, affaibli par la toute puissance de l'empereur, ne s'affiche plus comme le garant de la liberté face au prince. L'évêque, par cette fondamentale liberté de parole utilisée par Ambroise, obtient donc ce devoir religieux d'imposer une limitation morale aux volontés impériales.

\bigskip

Ainsi, la position de l'évêque auprès de l'empereur se définit entre autres par son devoir d'interpellation qui ne peut être supprimé. Le dialogue se doit donc d'être entretenu pour respecter le droit à la liberté de parole, qui fonde en partie le statut politique de l'Église. Le pouvoir romain chrétien est même légitimé par ce droit que possèdent les évêques à exprimer de « justes remontrance », qui permet au souverain de rester dans le chemin de la foi et d'éviter les péchés qui guettent le pouvoir.

\bigskip

\subsubsection{La proximité d'Ambroise avec les princes chrétiens}

Si la \latin{libertas dicendi} offre aux évêques le droit théorique de s'exprimer, tous n'ont pas pour autant la même autorité ou influence face au pouvoir impérial. Bien qu'ils puissent être reçus et entendus, seule la qualité de la relation tissée avec le souverain garantit une véritable écoute et prise en compte de cette parole. Ambroise est évidemment conscient des limites de l'\latin{auctoritas} institutionnelle de l'Église face à la réalité du pouvoir et s'attache fortement à développer un lien personnel important, dépassant le statut politique, avec les empereurs pour conforter son autorité au sein de la cour. L'idée est simple et résume parfaitement la vision qu'a Ambroise de s'impliquer dans la vie politique : si chacune de ses actions est encadrée par un cadre théorique poussé, il n'en oublie jamais les réalités immuables de la société, transformant alors son poste d'évêque étranger à la cour en un personnage intimement lié aux souverains qui ne peut s'en passer. Cette proximité humaine avec les détenteurs du pouvoir ne signifie pas pour autant un asservissement total envers eux. Au contraire, Ambroise a confiance en son rôle et en ses capacités et use de ses relations pour exprimer pleinement et librement ses pensées.

\bigskip

Cette dynamique s'observe dès le début de son épiscopat, à travers ses premiers rapports avec Gratien. La relation entre les deux débute par une demande de l'empereur qui souhaite avoir un exposé sur la foi trinitaire afin de mieux comprendre les tensions existantes entre les courants chrétiens. Et déjà, les réponses de l'évêque se font attendre, la correspondance prend du temps et la rédaction du \latin{De Fide} n'est pas précipitée par Ambroise qui ne semble pas réagir à la pression d'une demande impériale. Comme le note Yves-Marie Duval, «~Ambroise n’est pas le courtisan zélé qui se hâte de satisfaire la demande d’un puissant ou qui l’encombre de ses productions.\autocite[223]{duval_lettres_gratien}~». Ce refus de la course à la production le place dans une position de force : c'est bien l'empereur qui doit relancer l'évêque dans sa requête d'apprentissage de la foi. L'état de cette relation est un signe de confiance absolue de Gratien envers Ambroise dans sa capacité à répondre aux problèmes religieux puisque l'empereur ne cherche pas de réponse auprès d'autres évêques, s'appuyant sur l'autorité certaine d'Ambroise en Italie. Avec Gratien, Ambroise nourrit une relation de respect et de dépendance intellectuelle, valorisant son statut à la cour impériale.

\bigskip

Avec Valentinien II, empereur beaucoup plus jeune, Ambroise ajoute à sa posture d'autorité religieuse celle du père et maître à penser pour réussir à forger une influence durable. En revanche, cette relation met du temps à se mettre en place à cause de l'influence de la mère du prince, Justine, fervente défenseuse de la foi arienne. Ambroise se retrouve à intervenir à plusieurs reprises pour défendre la foi chrétienne et même nicéenne, lors des affaires de l'autel de la Victoire en 384 ou des basiliques de Milan en 386. Et pourtant, après la mort du jeune empereur par suicide ou assassinat en 392, Ambroise insiste dans son \latin{De Obitu Valentiniani} sur la tristesse qui l'envahit, signe d'une proximité certaine bien que sûrement en partie exagérée. Ce changement de comportement est notamment à relier à la mort de Justine en 388 et aux «~pressions discrètes de Théodose~» pour arriver à ce respect filial et cette affection paternelle\autocite[Introduction]{ambroise_mort_valentinien}. Cette proximité avec Valentinien est notamment exprimée dans la Lettre 25 à destination de Théodose, dans laquelle il revient sur la mort de l'empereur, pour rappeler son rôle auprès du pouvoir impérial en Occident :

\bigskip

\begin{quote} «~Il professait qu'il devait son éducation à moi, il me désirait comme un père attentif, et lorsque certains prétendaient avoir reçu des nouvelles de mon arrivée, il les anticipait avec impatience. D'ailleurs, pendant ces jours mêmes de deuil public, bien qu'il eût sous la main des évêques saints et éminents dans les limites de la Gaule, il jugea néanmoins bon de m'écrire pour que je lui confère le Sacrement du Baptême. Par cette demande, sinon raisonnablement, du moins affectueusement, il témoignait de son amour envers moi.\footnote{« \latin{Ille se a me nutritum praeferebat, ille ut sedulum patrem desiderabat, ille simulato a quibusdam adventus mei nuntio inpatienter praestolabatur. Quin etiam illis ipsis publici doloris diebus, cum sanctos et summos sacerdotes domini intra Gallias haberet, ut a me tamen sacramentis baptismatis initiaretur, scribendum arbitratus est ; quod etsi non rationabiliter, amabiliter tamen erga me suum studium testificatus est.}~». \cite[\nopp 25, 2.]{ambroise_csel_82_1}. {Traduction personnelle.}}~» \end{quote}

\bigskip

Le terme de « père attentif » est le plus intéressant, Ambroise n'est pas un simple évêque remplaçable au sein de la cour, mais bien celui qui offre aussi bien l'éducation que le Salut aux empereurs. Comme toujours, il est difficile de savoir la part d'exagération dans les propos d'Ambroise, surtout que nous ne possédons pas la demande de baptême par Valentinien, mais une intimité, bien que régulièrement tendue, existe, ou du moins est recherchée par l'évêque. En effet, plus que de savoir la réalité de la relation qu'il entretient avec l'empereur, il est intéressant de comprendre qu'Ambroise insiste sur le fait qu'il a toujours été proche, afin de développer une autorité auprès de tous qui dépasse le cadre institutionnel. Et c'est cette position qui lui permet d'agir aussi fermement dans l'affaire de l'autel de la Victoire. Il cherche à se différencier de son opposant Symmaque en adoptant un ton personnel dans ses lettres, comme pour démontrer une forme d'intimité dans la relation, que les arguments politico-religieux du sénateur ne peuvent dépasser. Dans son article sur la gestion des conflits par Ambroise, Gernot Michael Muller se permet même d'aller plus loin dans l'importance de la démonstration de l'intimité chez Ambroise : « C’est donc moins l’exercice réel du pouvoir sur son destinataire qui importait à Ambroise, que le fait de marquer implicitement la position de force qui était la sienne et qui dérivait d’une place privilégiée par rapport à ce même destinataire, à savoir celle du directeur de conscience.\autocite{muller_conflits}~». L'idée d'un «~directeur de conscience~» regroupe parfaitement les deux aspects de l'éducation et de la foi, et c'est ce qu'il vise avec chacun des empereurs qu'il côtoie. La relation personnelle apparaît ainsi comme un levier d'action politique, qu'Ambroise utilise en parallèle de ses rôles religieux et politiques intrinsèques à sa position d'évêque de Milan.

\bigskip

En revanche, avec Théodose, construire une relation personnelle forte se révèle être un défi plus important. Général victorieux, empereur en Orient, régulièrement en concurrence avec les pouvoirs impériaux de Gratien ou Valentinien II, Théodose ne propose pas le même profil capable d'écouter et de suivre facilement les conseils de l'évêque de Milan, et pourtant Ambroise se permet tout autant d'intervenir et de parler au nom du Salut de l'empereur. L'idée cette fois n'est plus de se mettre dans une position de père ou d'ami, mais d'égal et de conseiller, capable d'apporter un jugement moral lucide et juste. La seule position d'évêque ne lui suffit plus : si en 388, après sa lettre pour réprimander la décision de Théodose au sujet de la synagogue de Callinicum, il réussit à obtenir de Théodose le pardon à un évêque, il connaît un échec virulent en début d'année 390 alors que l'empereur refuse de le consulter suite au massacre de Thessalonique\footnote{Voir plus de détail sur Ambroise et l'affaire de Thessalonique dans le 2.2.3.}. Ambroise ne peut donc plus simplement se reposer sur l'autorité innée de son rôle d'évêque. C'est ce qu'analyse avec justesse Peter Brown à partir de la page 152 de son livre \latin{Pouvoir et persuasion dans l'Antiquité tardive} en montrant qu'Ambroise doit relancer sa relation avec Théodose avec « le courage d'un philosophe. » Brown parle ici des deux approches ambrosiennes envers le pouvoir, fondé sur l'image de la Grèce antique du philosophe affrontant l'autorité politique : « Ambroise se présentait comme l’exemple chrétien de l’ancienne \latin{karteria}, l’obstination inspirée avec laquelle les philosophes affrontaient le puissant. [...] Avec Théodose, le temps était venu de la \latin{parrhésia}, du franc-parler.\autocite[155]{brown_power_1992}~». Ambroise se montre comme le conseiller éclairant l'empereur dans le chemin de la foi.

\bigskip

Évidemment, même avec Théodose, l'évêque de Milan cherche à jouer de sa proximité avec le pouvoir impérial pour mieux se faire entendre, comme le montrent les premiers mots de sa lettre pour demander à l'empereur de se repentir : « Le souvenir de notre ancienne amitié m'est doux.\footnote{« \latin{Et veteris amicitiae dulcis mihi recordatio est}~». \cite[\nopp ec. 11, 1]{ambroise_csel_82_3}.}~». Il est même question ici d'un temps long, comme pour justifier le fait que ses propos ne sont pas là pour entraver la vie de l'empereur. Les différentes lettres aux empereurs nous permettent donc de saisir avec précision les relations qu'Ambroise a su tisser avec le pouvoir impérial tout au long de son épiscopat, qui se révèlent tout aussi importantes, si ce n'est plus, que l'aspect institutionnel de l'épiscopat, la foi et l'Église dans son ensemble.

\bigskip

Finalement, le meilleur exemple dans la vie d'Ambroise de l'importance de l'entretien des liens intimes avec les souverains de l'Empire se situe dans son échec auprès d'Honorius, et plus précisément de Stilicon. À la mort de Théodose en 395, la structure politique de l'Empire est modifiée. Alors qu'il s'était fait maître de tout le territoire romain suite à la mort de Valentinien II et surtout suite à la défaite d'Eugène en 394, Théodose laisse son pouvoir dans les mains de ses deux fils : Arcadius âgé de 18 ans qui s'empare de l'Orient, et Honorius, seulement âgé de 10 ans, se retrouve avec l'Occident. Honorius est d'ailleurs présent à Milan lors de l'enterrement de Théodose, point important puisqu'une partie de l'Oraison Funèbre d'Ambroise est dédiée au très jeune héritier. Pourtant, Ambroise ne parvient pas à reproduire le schéma de proximité paternelle qu'il avait employé avec Valentinien II. L'évêque doit à ce moment faire face à un nouvel adversaire politique : Stilicon, général romain proche de Théodose qui affirme avoir reçu la régence de l'Empire. L'Orient échappe tout de même rapidement à son contrôle à cause de l'opposition d'Arcadius plus grand et donc apte à se présenter comme empereur, et du préfet du prétoire Rufin. En revanche, Honorius passe bien sous la tutelle de Stilicon, ce qui lui permet d'imposer son pouvoir dans les provinces d'Occident, reléguant le jeune prince à un rôle pratiquement protocolaire.

\bigskip

Alors que sur la fin du règne de Théodose l'influence ambrosienne semble au plus haut au sein du pouvoir impérial, son \latin{auctoritas} est contestée par Stilicon. En effet, celui-ci se légitime par son mariage avec la nièce de Théodose, son rôle de commandant militaire ainsi que le soi-disant legs de l'autorité impériale par l'empereur défunt. Stilicon ne semble pas avoir besoin d'une légitimité religieuse que peut apporter l'évêque de Milan, et choisit de l'écarter progressivement de son rôle politique en se montrant comme le véritable tuteur d'Honorius\autocite[à partir de la page 298]{palanque_ambroise}. Ambroise se voit donc forcé de revenir à un rôle purement clérical pour les derniers mois de sa vie. Le signe le plus flagrant de cette perte d'influence et de rôle politique est l'absence de lettre adressée à Stilicon ou Honorius pendant la durée de leur règne. Comme l'explique Neil McLynn, ce silence d'Ambroise, qui ne rejette ni n'adhère à la politique menée par le régent, ne doit pas être simplement perçu comme un manque de source, mais bien comme une preuve d'une fissure entre l'\latin{auctoritas} d'Ambroise et la \latin{potestas} de l'empereur\autocite[366]{mclynn_ambrose}.

\bigskip

En ayant conscience de cet aspect de la fin de vie d'Ambroise, il est intéressant de relire le \latin{De Obitu Theodosii}, non pas comme une simple louange à Théodose mais bien, comme le dit McLynn, comme une tentative politique de récupération par Ambroise de la tutelle d'Honorius et donc de se placer comme son mentor à la place de Stilicon\autocite[358-360]{mclynn_ambrose}. L'appui dans son discours d'une continuité entre Théodose et ses fils, aussi bien dans la politique impériale que dans la foi\footnote{\cite[\nopp 6-7]{ambroise_mort_theodose}.} a pour objectif, en plus d'établir une stabilité politique, de faire apparaître la tutelle d'Ambroise comme évidente. Puisque l'évêque était un proche conseiller de Théodose, il est normal qu'il le reste pour ses fils. Les volontés d'instructions et de préoccupations morales d'Ambroise envers les deux jeunes souverains sont donc certainement un simple outil politique pour garantir son rôle à la cour, ce qui s'avère être un échec. Stilicon s'empare pleinement du pouvoir en Occident et fait disparaître l'\latin{auctoritas} ambrosienne.

\bigskip

L'ensemble des relations d'Ambroise avec les empereurs de la fin du IVème siècle agit comme un révélateur des limites de l'\latin{auctoritas} épiscopale. Aussi importante soit-elle dans le discours d'Ambroise, le rôle politico-religieux de l'Église au sein de l'Empire reste principalement dépendant du bon vouloir du souverain, et donc de la réussite ou non dans la création d'un lien intime entre un évêque et son empereur. Cette compréhension par Ambroise des réalités politiques lui permet de mettre en place ses théories et réflexions et de perdurer, sous différentes postures, au coeur du jeu politique de l'Empire.

\bigskip

\subsubsection{La maîtrise des armes populaires et liturgiques}

Pour autant, lorsque les relations personnelles ne suffisent pas à s'imposer politiquement, et que les ordres impériaux viennent confronter les idées sociétales ou religieuses d'Ambroise, l'évêque de Milan ne se retrouve pas démuni et sait user d'armes variées pour arriver à ses fins. Il délaisse son \latin{auctoritas} politique pour revenir à un rôle d'évêque charismatique de la capitale occidentale. Au cours de la crise arienne de 386, le pouvoir impérial cherche de plus en plus à utiliser la force pour soutenir les objectifs de Justine et d'Auxence. Ambroise fait alors appel à un outil d'influence et d'autorité qui lui est propre : la maîtrise de la ferveur populaire et de l'espace liturgique. L'idée étant d'opposer à l'empereur les symboles du peuple et de la foi pour rendre impossible toute intervention.

\bigskip

Dans sa description de l'affrontement de 386, Jean-Rémy Palanque illustre cet appui sur la force populaire pour lutter contre les mesures juridiques que parvient à mettre en place Justine en faveur de la foi arienne, notamment une loi de janvier 386 accordant la liberté de culte aux « tenants de la foi de Rimini\autocite[150]{palanque_ambroise}.~». Il est difficile de savoir si Valentinien est allé jusqu'à demander à Ambroise de s'exiler, mais une chose est certaine, l'évêque de Milan préfère l'affrontement direct avec le pouvoir impérial, plutôt que la défaite de ses convictions religieuses, comme on peut le constater dans sa lettre 75 à Valentinien :

\bigskip

\begin{quote}
    «~À présent les évêques me disent qu'il n'y a pas grande différence entre quitter volontairement l'autel du Christ et le livrer, car le quitter, c'est le livrer.\footnote{«~\latin{Nunc mihi sacerdotibus dicitur non multum interesse utrum uolens relinquas an tradas altare Christi, cum enim reliqueris trades.}~». \cite[\nopp 75, 18]{ambroise_lettres}.}~»
\end{quote}

\bigskip

Selon Gérard Nauroy, Valentinien n'aurait pas pu le condamner à l'exil alors qu'il lui demande de venir débattre au Consistoire, ça serait contradictoire. Ainsi, la demande d'exil dont il est question dans les écrits de Paulin sur la vie d'Ambroise provient sûrement de la non prise en compte du moment précis de la requête. Dans sa lettre, Ambroise feint de regretter que la demande d'exil arrive trop tard, et qu'il aurait été enclin à ne pas résister plus tôt, mais qu'il est désormais trop tard : «~Tu aurais dû m'envoyer où tu voulais, car je m'offrais moi-même à tous.\footnote{«~\latin{Debuisti me quo uolueras destinare, quem ipse omnibus.}~». \cite[\nopp 75, 18]{ambroise_lettres}.}~»

\bigskip

Mis à mal par Justine et Valentinien, Ambroise choisit de s'enfermer dans la basilique, soutenu par la foule milanaise très attachée à sa figure. C'est dans ce contexte qu'il met en place une nouvelle arme pour parvenir à imposer ses idées : la cohésion par la liturgie. Augustin, proche des événements notamment par le biais de sa mère, raconte dans ses \latin{Confessions} comment Ambroise instaure le chant des hymnes et des psaumes « comme cela se fait en Orient », pour éviter que la force populaire ne se fasse emporter par la peur ou l'ennui. Il ne s'agit donc pas seulement d'un caprice dans la pratique religieuse, mais bien d'une tactique politique en faveur de l'évêque de Milan. Augustin écrit sur cette ferveur comme signe de la toute-puissance ambrosienne à Milan : «~La foule des pieux fidèles passait les nuits dans l'Église, prête à mourir avec son évêque, votre serviteur.\footnote{«~\latin{...}~». \cite[IX, 7]{augustin_confessions}.}~» Le pouvoir impérial se retrouve paralysé par la simple présence de la population, amenant à une nouvelle démonstration de force d'Ambroise et de son \latin{auctoritas}.

\bigskip

Cependant, pour dépasser le rejet qu'a Valentinien II de la \latin{libertas dicendi} et de la relation personnelle, et s'imposer définitivement comme la source d'autorité à Milan, Ambroise doit chercher plus loin que la simple résistance « passive » avec le soutien du peuple. Ainsi, arrive au «~bon moment~» la découverte des corps des martyrs Gervais et Protais en juin 386. C'est un passage que regarde avec beaucoup de critique Neil McLynn, qui souligne qu'Ambroise, politiquement isolé, avait besoin d'une forme de validation divine pour montrer que sa position était celle à suivre dans la foi chrétienne, et donc que son opposition dangereuse au pouvoir impérial était juste et légitime. Cette découverte d'importance pour la religion chrétienne agit comme une source d'autorité majeure, tout en rassemblant d'autant plus la population milanaise derrière ses idées et la foi catholique, contre l'objectif impérial : «~Le thème de l'unité, propre à Ambroise, était de toute façon bien choisi pour séduire ceux qui étaient soucieux de rétablir la concorde à Milan. L'événement constituait tout autant une démonstration opportune de l'engagement de l'évêque envers cette cause qu'un déploiement triomphal de la force de son parti. C'est là que réside l'explication de sa réussite à briser la frénésie de la persécution.\autocite[215]{mclynn_ambrose}~»

\bigskip

Évidemment, Ambroise défend la véracité de cette découverte face aux accusations de tromperie émanant du camp arien, mais dans une de ses lettres à sa sœur Marcelline, il ne se cache pas pour autant d'utiliser les figures des martyrs comme une arme politique contre la force armée :

\bigskip

\begin{quote}
    «~Grâce te soient rendues, Seigneur Jésus, de nous avoir suscité une telle puissance spirituelle des saints martyrs en ce moment où ton Église ressent le besoin de plus grandes protections. [...] Les uns se glorifient de leurs chars et les autres de leurs chevaux, mais nous, nous nous glorifierons du nom de notre Seigneur Dieu.\footnote{«~\latin{Gratias tibi, Domine Iesu, quod hoc tempore tales nobis sanctorum martyrum spiritus excitasti, quo ecclesia tua praesidia maiora desiderat. [...] Hi in curribus et hi in equis, nos autem in nomine domini dei nostri magnificabimur.}~». \cite[\nopp 77, 10]{ambroise_lettres}.}~»
\end{quote}

\bigskip

Par l'opposition entre les «~chars et chevaux~» et la «~puissance spirituelle~», Ambroise retire à Valentinien la légitimité de l'usage de la force publique : Dieu s'est prononcé en faveur de la foi de Nicée. On peut également noter que dans cette lettre, de nouveau, l'évêque de Milan se rattache à un passage des Écritures pour renforcer sa position : la victoire de Ghiézi contre les Syriens, grâce au soutien de Dieu et des «~soldats du Christ\footnote{\cite[\nopp 77, 11]{ambroise_lettres}.}~».

\bigskip

Cet épisode nous permet de prendre en compte un point essentiel de l'\latin{auctoritas} ambrosienne : la capacité à mobiliser différentes armes politiques, liturgiques, rhétoriques, populaires, pour arriver à ses fins et faire en sorte que ses réflexions politiques se rapprochent de la réalité de l'Empire. Comme le note Peter Brown, Ambroise possède un atout que les philosophes antiques ne possèdent pas, et qui lui permet d'outrepasser la plupart des situations : il est le «~maître de la basilique~» et donc la plus grande autorité d'un espace aussi important pour le peuple que pour l'empereur en tant qu'outil de mise en scène du pouvoir impérial\autocite[156]{brown_power_1992}. Le contrôle de la foi couplé à l'appui populaire qu'il possède à Milan, fait d'Ambroise une autorité difficilement contestable, même lorsque la \latin{libertas dicendi} ne suffit pas.

\subsection{L'aboutissement de l'\latin{auctoritas} personnelle}

\subsubsection{L'évêque au service de la stabilité impériale}

Ce qui est intéressant au sujet de cette \latin{auctoritas}, c'est qu'elle permet à Ambroise d'atteindre des sommets de l'influence dans la partie occidentale de l'Empire, que ce soit par des actions pour la défense de la foi, par la démocratisation d'un statut de conseiller, ou bien par des événements véritablement politiques demandés par l'empereur lui-même. Comme le souligne Peter Brown tout au long de son ouvrage \latin{Pouvoir et Persuasion}, la fin du IV\textsuperscript{e} siècle marque le début d'un tournant dans l'exercice de l'autorité, et Ambroise en est un des instigateurs majeurs. L'autorité morale et locale qu'apportent les évêques tend à remplacer les philosophes et conseillers politiques laïcs, signe d'un changement important dans la perception de l'influence\autocite{brown_power_1992}. Et c'est cet aspect de l'\latin{auctoritas} qui amène Ambroise à endosser à deux reprises, en 383 et en 384 ou 386, le costume d'ambassadeur diplomatique agissant au nom de l'empereur Valentinien II. Bien que la relation entre le pouvoir impérial et l'évêque de Milan soit plutôt dans un moment de tension entre 383 et 387, c'est bien vers la figure populaire et intellectuelle d'Ambroise que Valentinien se tourne pour se rendre à Trèves et s'assurer du maintien de la paix avec l'usurpateur Maxime, signe d'une reconnaissance de ses capacités politiques et oratoires, ainsi que de sa maîtrise des codes sociaux de l'aristocratie romaine dont il est issu.

\bigskip

La première mission diplomatique qui lui est confiée date de 383 et, bien que l'on ne possède pas de lettre ou de texte d'Ambroise sur cet événement, semble être une réussite. Alors que l'empereur Gratien vient d'être assassiné par les troupes de Maxime, celui-ci se revendique empereur en s'installant à Trèves et s'empare des provinces les plus à l'ouest de l'Empire. Alors que l'usurpateur parvient rapidement à se faire reconnaître comme Auguste par Théodose\footnote{Hervé Savon explique cette reconnaissance par le fait que Théodose ne regrettait pas particulièrement la mort de Gratien. Celui-ci, malgré son jeune âge, se montrait régulièrement comme le supérieur et n'hésitait pas à intervenir dans les affaires de l'épiscopat oriental en menant une politique religieuse parfois opposée. Reconnaître Maxime comme empereur légitime lui permettait donc de se débarrasser d'une menace importante. \cite[181]{savon_ambroise}.}, le jeune empereur à Milan envoie Ambroise pour s'assurer du maintien de la paix et éviter que Maxime ne jette son dévolu sur l'Italie. C'est une des premières fois qu'un évêque joue officiellement un rôle diplomatique au service de l'Empire en mettant ses capacités pour protéger la paix entre Valentinien II et Maxime.

\bigskip

Mais notre intérêt pour comprendre l'aboutissement de l'\latin{auctoritas} ambrosienne doit se faire en se concentrant principalement sur la deuxième ambassade d'Ambroise, puisqu'il s'agit de la plus documentée, notamment grâce à l'\latin{Epistolae} 30 où Ambroise apporte à Valentinien un compte-rendu de sa mission. L'objectif, en plus de continuer à s'assurer du maintien de la paix et d'un \latin{statu quo} sur les possessions territoriales, est de récupérer la dépouille de Gratien afin de lui rendre les honneurs liés à son statut. Il est important de noter qu'un débat historiographique existe au sujet de la datation précise de cette ambassade. La lettre d'Ambroise n'est pas datée, et il est donc nécessaire de s'appuyer sur d'autres sources, extérieures ou non à l'évêque de Milan. Jean-Rémy Palanque, qui s'appuie sur la chronologie des écrits de Paulin de Milan, la situe en fin d'année 386 ou début 387, après la résolution du conflit contre les ariens. Hervé Savon, en revanche, propose une datation plus précoce, autour de 384, donc peu de temps après la première ambassade. Il défend sa théorie dans sa biographie d'Ambroise en faisant appel à un extrait du \latin{De Obitu Valentiniani} :

\bigskip

\begin{quote} «~Il a été doux pour moi de s'acquitter de cette fonction, la première fois pour vous sauver, la seconde pour la paix et la piété avec laquelle vous demandiez les restes de votre frère : vous n'étiez pas encore en sécurité, et déjà vous vous souciiez de donner à votre frère les honneurs de la sépulture\footnote{« \latin{Dulce mihi officium fuit, quod pro salute tua primum suscepi, deinde pro pace, et pietate, qua fraternas reliquias postulabas: necdum pro te securus, et iam pro fraterni funeris honore sollicitus.}~». \cite[\nopp 28]{ambroise_mort_valentinien}.}.~» \end{quote}

\bigskip

Pour Savon, l'utilisation du terme «~déjà~» dans ce passage où il évoque les deux missions qui lui ont été confiées, constituerait une « ironie déplacée » si Valentinien avait attendu trois ans pour réclamer le corps de Gratien\autocite[200]{savon_ambroise}. Il est difficile d'avoir un avis tranché sur ce sujet, bien que l'échec d'Ambroise sur cette ambassade soit certainement l'un des éléments ayant précipité la campagne de Maxime contre Valentinien en Italie, ce qui pousserait à une datation tardive, en début 387, malgré les tensions toujours existantes entre l'évêque et le pouvoir impérial.

\bigskip

C'est donc par le compte-rendu de son ambassade dans la lettre 30, qu'il a certainement rédigée par peur d'être vu comme un traître aux yeux de Valentinien après son échec à récupérer la dépouille, qu'Ambroise théorise le plus en profondeur sa conception du service de la politique impériale. L'évêque de Milan se montre ici comme un véritable membre de la cour impériale, prêt à mettre son statut épiscopal de côté pour remplir sa mission. C'est notamment le cas de son passage où il accepte de rencontrer Maxime dans le Consistoire, lieu où il refuse pourtant d'aller en 386 alors qu'il est convoqué par Valentinien tant il représente le pouvoir impérial, signe d'un transfert complet de son autorité vers un rôle d'ambassadeur :

\bigskip

\begin{quote} « Je dis que cette démarche était incompatible avec la fonction que j'occupais, mais que je n'hésiterais pas à remplir le devoir que j'avais entrepris, et que surtout, dans le service de Votre Majesté, et puisque c'était réellement pour soutenir votre affection fraternelle, je me réjouissais de m'humilier.\footnote{«~\latin{Dicens incongruum id esse muneri, quod gerebam; sed tamen non refugere me officium, quod suscepissem: et praesertim in servitio Vestrae Pietatis, et cum vere fraterno foveretis affectu, gratulari me, quod pro vobis humiliarer.}~». \cite[\nopp 30, 2]{ambroise_lettres}.}~» \end{quote}

\bigskip

Ambroise se sert de la lettre comme d'un outil pour expliciter et légitimer ses actions en tant qu'ambassadeur, et ainsi s'assurer que son autorité d'évêque et de conseiller impérial ne soit pas atteinte par le résultat de sa mission diplomatique. Cette précision de son dévouement pour l'empereur légitime lui permet de s'assurer d'avoir le soutien de Valentinien à son retour, tout en justifiant ses prises de positions virulentes envers Maxime, qu'il retransmet également dans la lettre. En effet, son rôle diplomatique ne semble pas l'empêcher d'utiliser l'ensemble de son \latin{auctoritas} pour s'adresser à Maxime et obtenir ce qu'il souhaite, usant à la fois de sa \latin{libertas dicendi} et de la \latin{karteria}, la position du philosophe s'opposant au pouvoir. Cette posture qu'Ambroise construit tout au long de son épiscopat l'amène à un discours parfois brutal envers son interlocuteur :

\bigskip

\begin{quote} « J'ai fait périr mon ennemi, dit-il [Maxime]. Ce n'était pas lui qui était votre ennemi ; c'est vous qui étiez le sien. [...] Si je ne me trompe, c'est l'usurpateur qui commence la guerre ; l'empereur ne fait que défendre son droit.\footnote{«~\latin{Hostem, inquit, occidi. Non ille tuus hostis, sed tu illius. [...] Nisi fallor, usurpatorem bellum inferre, imperatorem ius suum tueri.}~». \cite[\nopp 30, 10]{ambroise_lettres}.}~» \end{quote}

\bigskip

Évidemment, comme il s'agit d'un récit fait par la main même d'Ambroise, il ne faut pas mettre de côté la possible exagération de sa propre rhétorique. Mais le simple fait qu'Ambroise veuille se montrer comme un personnage sévère envers ses ennemis est en soi une démonstration d'un aboutissement politique de son \latin{auctoritas}. L'utilisation du terme « usurpateur » va dans ce sens d'un évêque ayant des positions tranchées, capable de rassembler ses qualités épiscopales et politiques pour réaliser ce qui lui semble juste pour l'Empire, comme ici entrer dans un esprit de contestation vis-à-vis de Maxime. Paradoxalement, le lancement d'une campagne militaire par Maxime et la prise de l'Italie en 387 est le signe de l'influence d'Ambroise dans la diplomatie impériale\footnote{Valentinien choisit de se réfugier auprès de Théodose, qui hésite à l'idée de se lancer dans une guerre civile. Finalement, à l'été 388 les deux empereurs lancent une campagne contre Maxime qui est battu puis exécuté, ce qui permet à Théodose de mettre la main sur l'Italie, repoussant le jeune Valentinien dans les provinces les plus à l'ouest.}. Alors que la première ambassade trouve sa réussite dans l'instauration d'un \latin{statu quo} et d'une reconnaissance officieuse de l'autorité de Maxime, l'échec de la deuxième mission et la tension instaurée aboutissent à un conflit armé entre les empereurs. L'\latin{auctoritas} d'Ambroise ne se résume donc pas seulement à sa popularité au cœur de Milan, ni à sa capacité à influencer la foi des souverains, mais aussi à impacter l'ensemble de la diplomatie impériale, faisant de ses réussites et échecs des tournants politiques impossibles à éclipser.

\bigskip

\subsubsection{La posture du prophète}

À l'opposé du rôle d'ambassadeur, qui s'inscrit dans une logique de service diplomatique pour l'Empire, et pourtant émanant tout aussi directement de l'\latin{auctoritas} personnelle, Ambroise développe et entretient longuement une autre facette de son rôle d'évêque : le positionnement en tant que nouveau prophète aux côtés des détenteurs du pouvoir. L'idée derrière cette posture est de trouver le meilleur moyen de modeler la conscience de l'empereur et de changer à sa guise la direction des décisions impériales lorsque celles-ci heurtent la morale chrétienne. Palanque s'y attarde avec précision, en montrant que le développement d'un avis politique d'Ambroise n'est que l'aboutissement de son autorité sous toutes ses formes : l'obligation religieuse liée à son rôle d'évêque, le développement de son statut de conseiller impérial et bien évidemment la mise en valeur et l'éloge du franc-parler. Autant d'outils utilisés par Nathan pour s'écarter du risque d'un prince mal éclairé pouvant nuire à la politique impériale\autocite[208-210]{palanque_ambroise}.

\bigskip

Cette position de prophète ne provient pas de nulle part, il l'institutionnalise et la légitime en s'emparant d'une typologie biblique bien connue et respectée : celle du roi David et du prophète Nathan. Lorsqu'Ambroise intervient pour reprendre moralement Théodose en 388 puis en 390, il cite immédiatement dans ses remontrances et demandes la figure de Nathan, qui n'est pas étrangère à l'éducation religieuse de l'empereur. Par ce simple fait d'invoquer l'autorité prophétique de Nathan qui impose sa vision sur David, au sein d'un événement contemporain similaire, Ambroise s'approprie la place de Nathan comme un intermédiaire indispensable entre Dieu et le souverain, y compris pour juger des actions non religieuses mais où la conduite morale est attaquée. L'\latin{auctoritas} personnelle de l'évêque prend donc une place importante dans la conduite du Salut du prince, et si ce chemin interfère avec les actions politiques, alors Ambroise n'hésite pas à imposer sa volonté.

\bigskip

Comme évoqué, c'est donc déjà le cas dès l'affaire de Callinicum, où il rappelle à Théodose que ses succès militaires, notamment sa victoire contre l'usurpateur Maxime, n'ont été permis que par l'aide de Dieu. L'autorité d'Ambroise vise donc également à s'assurer que la politique impériale reste dans une reconnaissance envers la foi chrétienne, qui s'explique par les paroles de Nathan :

\bigskip

\begin{quote}
    « Et que te dira le Christ après cela ? Ne te souviens tu pas de ce qu'il a fait dire au saint David par le prophète Nathan ?\footnote{«~\latin{Et quid te cum posthac Christus loquetur ? Non recordaris quid David sancto per Nathan prophetam dauerit ?}~». \cite[\nopp 40, 22]{ambroise_lettres}.}~»
\end{quote}

\bigskip

Mais plutôt que simplement notifier le rôle divin dans la réussite de Théodose, ce passage sert à rappeler que les souverains des Écritures ont toujours eu des figures religieuses à leur côté, comme Nathan devant David\footnote{2 Samuel 7.}, ou Élie face à Achab\footnote{1 Rois 21.}. Ainsi, la parole d'Ambroise n'est pas soumise aux conflits contemporains puisqu'elle n'est que la retransmission de la parole divine. Et cette parole a également comme objectif d'éviter au souverain de s'enfermer dans le péché qui apparaît comme inévitable par sa fonction dans l'Empire. En effet, la morale chrétienne prônée par les écrits d'Ambroise s'oppose à l'arbitraire monarchique. Il est quasiment impossible de suivre parfaitement le chemin de la foi chrétienne en résolvant les problèmes politiques et sociétaux de l'Empire. Et c'est notamment le cas chez Ambroise, du moins c'est ce que semble dire l'évêque dans sa lettre \latin{extra collectionem} 11 :

\bigskip

\begin{quote}
    «~Daignez me permettre, empereur gracieux. Vous avez un zèle pour la foi, je l’avoue, vous avez la crainte de Dieu, je le confesse : mais vous avez une véhémence de tempérament, qui, si elle est apaisée, peut aisément se changer en compassion, mais si elle est enflammée, devient si violente qu’il vous est à peine possible de la maîtriser.\footnote{«~\latin{Permitte, quaeso, imperator auguste. Habes zelum fidei, fateor, habes timorem dei, confiteor: sed habes naturae impetum, quem si quis leniat, cito uertes ad misericordiam; si quis stimulet, in maius suscitatur ut eum reuocare uix possis.}~». \cite[\nopp ec 11, 4]{ambroise_csel_82_3}. Traduction personnelle.}~»
\end{quote}

\bigskip

L'évêque de Milan se permet ici de tenir des propos audacieux à l'encontre de Théodose, signe de la confiance de sa posture prophétique, soutenue entre autres par l'image de Nathan. Ce passage sur la possible violence de l'empereur est à relier à certains extraits de l'\latin{Apologie de David} qu'Ambroise rédige peu de temps avant et qui s'attarde sur les risques plus grands qu'ont les détenteurs du pouvoir de fauter : «~Il a péché : c'est la marque de sa condition ; il s'est prosterné : c'est la marque de son amendement. Sa faute, c'est le lot commun ; mais sa confession, c'est son mérite distinctif. Ainsi être tombé dans le péché, c'est le propre de la nature, mais avoir lavé sa faute, c'est le propre de la vertu.\footnote{«~\latin{Peccavit, quod solet regiae esse potestatis; lapsus est, sed poenitentiam gessit, flevit, ingemuit. [...] Ergo quod lapsus est, naturae fuit; quod emendavit, virtutis.}~». \cite[\nopp IV, 15]{ambroise_apologie_david}.}~» Ambroise possède bien une certaine cohérence politique entre ses écrits et réflexions, ne délaissant pas ses idées dogmatiques lorsqu'il s'agit de confronter le pouvoir impérial. Pour Ambroise, il ne s'agit pas d'inculper Théodose de ses crimes, mais bien de lui rappeler la difficulté de son rôle, y compris par des comparaisons avec les récits du Livre de Samuel et des Rois, afin de le contenir dans la morale chrétienne et de le pousser à se repentir pour faciliter ses objectifs politiques. La vision d'un Ambroise prophète a donc pour objectif de faire rentrer l'idéal chrétien dans le quotidien politique du pouvoir.

\bigskip

Ainsi, l'identification de Nathan comme source d'autorité et d'influence sur le pouvoir permet à Ambroise de théoriser une nouvelle forme de souveraineté chrétienne : pour être un empereur victorieux, il faut être un empereur pénitent. Ambroise n'use pas de son autorité pour discréditer l'homme, mais pour sauver le prince. C'est en partie ce que nous avons déjà évoqué au sujet de la \latin{libertas dicendi} : Ambroise se veut proche mais ne cherche pas à humilier l'empereur. Son rôle de « nouveau Nathan » se fait dans un cadre intime et non pas comme un orateur public cherchant le scandale. Il privilégie, du moins dans un premier temps, la voie privée, celle de la direction de conscience :

\bigskip

\begin{quote}
    «~J’ai préféré recommander secrètement cette véhémence à votre réflexion, plutôt que de courir le risque de l’enflammer publiquement par mes actes.\footnote{«~\latin{Hunc ego impetum malui occulte tuae committere considerationi quam meis factis publice fortasse mouere.}~». \cite[\nopp ec 11, 5]{ambroise_csel_82_3}. Traduction personnelle.}~»
\end{quote}

\bigskip

Ambroise apparaît de nouveau conscient de son influence et de l'importance d'amener une remontrance privée, même s'il est possible de voir cette citation comme un avertissement envers l'empereur du fait que l'évêque est toujours en mesure d'utiliser son autorité contre la notoriété de l'empereur. C'est dans cette lettre, écrite à la suite du massacre de Thessalonique sur lequel je reviens en détail dans la fin de ce chapitre, qu'Ambroise relie définitivement son rôle auprès de Théodose à celui de Nathan. Il ne laisse plus de doute quant à sa volonté d'imposer cette grille de lecture à Théodose, utilisant les mots de l'Écriture pour convaincre la pénitence impériale :

\bigskip

\begin{quote}
    «~Que Votre Majesté ne se montre donc pas impatiente d’entendre ce qu’on lui dit, comme le prophète le dit à David, car si vous l’écoutez obéissamment et dites : « J’ai péché contre le Seigneur », si vous employez ces mots du roi Prophète, « Venez, adorons et prosternons-nous, agenouillons-nous devant le Seigneur notre Créateur », à vous aussi il sera dit : «~Parce que tu te repens, le Seigneur a mis de côté ton péché, tu ne mourras pas.~»\footnote{«~\latin{Non ergo impatiens sit imperator, si dicatur ei quod dixit propheta Dauid regi: 'in me peccatum est'; si audiat digne et dicat: 'peccaui domino'; si dicat regium illud propheticum: 'uenite adoremus et procidamus ante dominum et ploremus ante dominum qui fecit nos', dicatur et tibi: 'quoniam paenitet te, dimittit tibi dominus peccatum tuum et non morieris'.}~». \cite[\nopp ec 11, 7]{ambroise_csel_82_3}. Traduction personnelle.}~»
\end{quote}

\bigskip

La précision « à vous aussi » est la plus importante : elle relie aussi bien Théodose à David qu'Ambroise à Nathan et permet à l'évêque de citer directement les paroles de l'Ancien Testament comme un signe de similitude des événements et donc de résolution du problème. En acceptant cette réprimande, Théodose accepte définitivement l'\latin{auctoritas} d'Ambroise. Que ce soit pour Callinicum ou pour Thessalonique, il accepte son autorité dans l'idée d'éviter de se mettre à dos une population milanaise acquise à la cause de son évêque. C'est le signe ultime de l'autorité politique d'Ambroise qui est bel et bien renforcée par sa position de prophète aux côtés d'un souverain chrétien.

\bigskip

\subsubsection{Le devoir d'exemplarité morale : l'affaire de Thessalonique}

La théorie du prophète et du roi pénitent que nous venons d'exposer trouve son application concrète et éclatante lors de l'affaire de Thessalonique en 390. C'est notamment cet événement qui permet à Ambroise de transformer son \latin{auctoritas} spirituelle en une véritable contrainte politique, obligeant l'empereur à suivre son autorité, non seulement pour sa foi personnelle, mais également par nécessité d'État.

\bigskip

Le massacre de Thessalonique est un ordre impérial déclenché par Théodose au printemps 390. Cet ordre fait suite au soulèvement de la population contre Buthéric, chef des armées locales ayant fait emprisonner un cocher populaire de la ville, qui est tué. La réaction de l'empereur est immédiate et violente : il prend la décision de rassembler les citoyens dans le cirque de la ville et déclenche un massacre faisant près de sept mille victimes. Cet ordre prend place alors que les relations entre l'empereur et l'évêque de Milan sont difficiles, c'est une période que Jean-Rémy Palanque nomme la «~froideur de 389\autocite[223]{palanque_ambroise}.~». Après le conflit lié à l'affaire de Callinicum, les défiances sont nombreuses et les membres du Consistoire ont l'interdiction de parler à Ambroise pour éviter qu'il s'immisce dans les affaires politiques. Ambroise prend connaissance du massacre tardivement et décide d'intervenir, mais en évitant la confrontation directe. Il fait alors le choix de lui adresser une lettre privée qui cherche à pousser Théodose à la pénitence en liant le Salut personnel de l'empereur à la réussite de son pouvoir et donc de l'Empire. Cette pénitence ne vient évidemment pas de nulle part, elle fait écho aux péchés des rois David et Salomon et de leur longue recherche du pardon à travers la pénitence. Ambroise écrit ainsi à Théodose :

\bigskip

\begin{quote}
    «~Ce que j’écris, ce n’est pas pour vous confondre, mais pour que ces exemples royaux vous incitent à écarter ce péché de votre royaume ; ce que vous ferez en vous humiliant devant Dieu. Vous êtes un homme ; la tentation est tombée sur vous ; vainquez-la. Le péché ne se lave que par les larmes et la pénitence.\footnote{« \latin{Non ut confundam te, haec scribo; sed ut regum exempla provocent te ad tollendum hoc peccatum de regno tuo; tolles autem humiliando te Deo. Homo es, tentatio te apprehendit; vince eam. Peccatum non tollitur, nisi lacrymis et poenitentia.}~». \cite[\nopp ec 11, 11]{ambroise_csel_82_3}. Traduction personnelle.}~»
\end{quote}

\bigskip

Pour convaincre Théodose que cette pénitence n'est pas une faiblesse politique mais au contraire un signe de réussite future pour l'Empire, Ambroise joint à sa lettre son \latin{Apologie de David}. L'objectif est double : valoriser son propos par l'exemple d'une figure biblique majeure et rassurer Théodose sur les risques de faire pénitence alors qu'il détient le pouvoir. Ce texte est très certainement dans un premier temps un sermon lu devant la population de Milan en automne 388, avant qu'il ne soit modifié pour se rattacher à la défaite de Maxime et au massacre de Thessalonique. L'\latin{Apologie de David} permet ainsi à Ambroise d'illustrer ses propos sur le «~nouveau David~» et de fournir une sorte de manuel de l'exemplarité royale dans la foi. Plus que seulement le Salut du souverain, Ambroise essaie de pousser Théodose à la pénitence en le mettant face au risque d'une répercussion de ses péchés à l'échelle de l'Empire. Cette fameuse exemplarité morale chez l'empereur est donc dans le champ d'action de l'\latin{auctoritas} de l'évêque puisqu'elle impacte l'ensemble de la politique impériale :

\bigskip

\begin{quote}
    «~Il ne pouvait nier son péché, mais en tant que coupable, il l’avouait avec douleur, sachant qu’il était tenu par les liens d'autant plus étroits que plus grandes étaient ses obligations : on exige en effet davantage de celui à qui l’on a confié davantage.\footnote{« \latin{Peccatum suum negare non poterat: sed quasi reus cum dolore fatebatur; eo magis se obligatum sciens, quo majora debet. Cui enim plus committitur, plus ab eo exigitur.}~». \cite[\nopp 51]{ambroise_apologie_david}.}.~»
\end{quote}

\bigskip

Le message d'Ambroise est donc que son intervention est légitimée par le fait que la pénitence n'est pas une option privée, mais un acte nécessaire au Salut de l'Empire. C'est ainsi que l'évêque de Milan réussit à justifier son implication dans une affaire qui ne semble pourtant pas se référer au champ d'action de l'\latin{auctoritas} épiscopale que nous avons décrit tout au long de ce chapitre.

\bigskip

La victoire de l'influence ambrosienne se fait ressentir à travers deux aspects. Premièrement, Théodose finit par accepter de se plier aux conseils d'Ambroise et fait pénitence en se rendant à la basilique de Milan sans les insignes impériaux, à égalité devant Dieu. L'autorité religieuse de l'évêque atteint bien définitivement la sphère politique en réussissant à interférer avec les actions impériales, signe d'une victoire de l'influence d'Ambroise. Il est tout de même important de noter que cette pénitence n'est pas un signe de soumission de Théodose à l'Église chrétienne mais plutôt une action lui permettant de redorer son image aux yeux de tous en se montrant comme un empereur pieux. Deuxièmement, l'empereur décide dans le même temps d'introduire une constitution sur les condamnations à mort, obligeant un délai de trente jours entre la condamnation et son exécution. Bien que cette loi émane de Théodose et de ses conseillers, il ne fait aucun doute qu'elle fait écho à la pression morale exercée par l'ensemble de l'\latin{auctoritas} d'Ambroise :

\bigskip

\begin{quote}
    « Si, par un mouvement de notre puissance royale, nous ordonnons de sévir contre certains plus sévèrement que de coutume, nous ne voulons pas qu'ils subissent la peine ou reçoivent la sentence immédiatement, mais que leur sort et leur fortune soient suspendus pendant trente jours.\footnote{« \latin{Si vindicari in aliquos severius contra nostram consuetudinem pro motu regiae potestatis iusserimus, nolumus statim eos aut subire poenam aut excipere sententiam, sed per dies triginta super statu eorum sors et fortuna suspensa sit.}~». \cite[IX, 40, 13]{code_theodosien}. Bien que les premières sources datent cette constitution de 382, il s'agit très certainement d'une erreur de copiste car la constitution est clairement issue des conséquences du massacre de Thessalonique. La critique historique, à la suite de Mommsen et Palanque, s'accorde à la restituer à 390.}~»
\end{quote}

\bigskip

La mise en place de la constitution ainsi que la pénitence de Théodose sont les preuves que l'action d'Ambroise dépasse le cadre épiscopal pour finalement affirmer la direction de conscience et son impact sur la sphère impériale. L'\latin{auctoritas} d'Ambroise trouve son aboutissement dans l'encadrement moral où ses théories politiques et religieuses sur l'idéal de la royauté chrétienne rejoignent des réalités concrètes mêlant évêques et empereurs.

\subsection*{Conclusion}

L'étude autour de la construction d'une \latin{auctoritas} personnelle chez Ambroise nous permet de comprendre que l'évêque de Milan conçoit son rôle dans l'Empire comme celui d'un conseiller diplomatique tout autant que comme un régulateur moral indispensable au souverain et donc à l'Empire. Après avoir montré comment l'Église, sous l'impulsion de la réflexion ambrosienne, s'est positionnée en institution indépendante, la seconde partie du chapitre s'est concentrée sur la mise en lumière des méthodes permettant à Ambroise d'agir au plus près du pouvoir et d'accroître son influence sur les décisions impériales.

\bigskip

Ambroise se révèle capable d'user de l'ensemble des outils à sa disposition pour construire à sa manière son influence. L'objectif de l'évêque semble toujours osciller entre son ambition personnelle de se placer comme une figure importante du pouvoir impérial et sa volonté d'agir pour le Salut de l'empereur et donc de l'ensemble de l'Empire. L'aspect le plus marquant de cette autorité reste sans aucun doute la critique morale et politique qu'il n'hésite pas à faire à l'égard des souverains lorsqu'ils s'écartent de sa vision des événements. C'est le cas de la querelle religieuse en 386 contre le soutien qu'apporte Valentinien II aux ariens, mais également lors des moments où Ambroise juge que l'empereur s'écarte de la foi chrétienne et met à mal la paix et la justice dans l'Empire. Critique notamment possible grâce à sa création d'un lien intime avec chacun des empereurs lui donnant accès à une véritable liberté de parole, plus grande encore que la \latin{libertas dicendi} qu'il réclame pour l'ensemble du monde épiscopal. Son autorité est donc régulièrement déployée sur le terrain diplomatique, à travers ses missions d'ambassadeur à Trèves ou son rappel à l'ordre contre Théodose après la punition proclamée contre l'évêque de Callinicum.

\bigskip

L'affaire de Thessalonique marque l'aboutissement de cette réussite ambrosienne alors qu'il parvient à faire de son \latin{auctoritas} un outil de l'idéal du pouvoir chrétien qu'il forge dans ses écrits. L'utilisation de la typologie biblique de David et Nathan pour s'adresser à Théodose lui permet de se placer non pas comme un moralisateur et opposant, mais bien comme un proche conseiller agissant pour son bien et celui de l'Empire, n'hésitant pas à relier le Salut de l'âme du prince à la prospérité du territoire romain.

\vspace{2cm}
\section*{Conclusion}

Au terme de cette analyse de l'arrivée d'une nouvelle dynamique entre l'autorité épiscopale et le pouvoir impérial, il apparaît assez clairement qu'Ambroise de Milan a joué un rôle majeur dans la redéfinition des rapports de force à la fin du IV\textsuperscript{e} siècle. En articulant la revendication d'une autonomie institutionnelle de l'Église avec la pratique personnelle de l'influence des évêques sur les empereurs, il fait intervenir les fondements du modèle politique des territoires chrétiens, qui donne une place politique de premier plan aux autorités spirituelles. L'empereur, autrefois maître incontesté du domaine du sacré, se retrouve soumis à la juridiction morale de l'Église, de la même manière que n'importe quel autre fidèle, voire même plus à cause de ses responsabilités.

\bigskip

Ce chapitre nous permet d'explorer une autre facette des écrits d'Ambroise. En effet, la pensée politique d'Ambroise ne se présente pas comme une simple réflexion théorique parfaitement cohérente ; elle doit au contraire composer avec l'instantanéité des actions et événements, ce qui amène une grande partie de ses pensées à être exprimées par les correspondances entretenues avec les évêques, les empereurs ou sa sœur. Qu'il s'agisse de l'affaire de l'autel de la Victoire, des missions diplomatiques à Trèves ou du massacre de Thessalonique, la doctrine ambrosienne se démarque par une réponse réaliste à un besoin concret. Sa théorie lui permet de suivre une ligne directrice claire, mais qui demande une constante adaptation. Cette dimension explique les nombreuses variations d'approche et de stratégie que l'on observe à travers ses écrits : il adapte son discours à la relation qu'il entretient avec son interlocuteur, qu'il soit dans une phase d'éducation morale et politique ou au contraire simplement à la recherche d'une plus grande légitimité et popularité par la religion. Le fait qu'une grande partie de la relation qu'Ambroise établit et entretient entre l'Église et le pouvoir nous soit principalement connue par des sources du cœur de l'action, et non pas extérieures à celle-ci, nous permet de comparer avec précision l'idéal d'Ambroise, dans sa construction d'un gouvernement du peuple et d'un empereur chrétien modèle, avec la réalité qu'il impose aux souverains contemporains. S'il n'essaie pas de s'interposer comme un opposant à la politique impériale, il n'hésite pas pour autant à faire connaître sa position pour imposer des façons d'agir afin de guider le pouvoir politique dans la direction qu'il souhaite. La récupération du concept d'\latin{auctoritas} pour l'Église et surtout pour lui-même se fait ressentir régulièrement dans ses lettres, lorsqu'il insiste sur l'indépendance des évêques, la séparation des domaines d'action et surtout le positionnement de l'empereur dans l'Église et non pas au-dessus de celle-ci.

\bigskip

Il est toutefois nécessaire de poser les limites de cette « victoire » ambrosienne en rappelant les contestations impériales contre son \latin{auctoritas} : on est encore loin d'une évidence pour les souverains. Les lettres, ou plutôt dans ce cadre l'absence de lettre, nous démontrent des résistances régulières contre lesquelles l'ensemble des stratégies d'influence d'Ambroise ne peuvent rien. C'est notamment le cas dans la relation avec Théodose entre les deux affaires de Callinicum et de Thessalonique : bien que pleinement chrétien, l'empereur n'accepte pas l'intervention d'Ambroise en 388 et choisit de l'écarter des informations et décisions politiques sur les deux années suivantes. Cette tension rappelle que l'autonomie et l'autorité de l'Église restent, à cette époque, dépendantes du bon vouloir impérial qui peut être rapidement et facilement remis en cause. L'exemple précédemment évoqué des dernières années de vie de l'évêque de Milan reste la démonstration la plus marquante de cette limite de l'autorité. Avec le refus total de collaboration du général romain Stilicon, tuteur d'Honorius, Ambroise est relégué à son statut d'évêque. Bien qu'Ambroise s'impose sans aucun doute durant son épiscopat comme la figure spirituelle la plus influente de son temps, capable d'intervenir de façon régulière et respectée dans les affaires impériales, il ne met pas pour autant en place un statut officiel des évêques auprès des empereurs romains.

\bigskip

En définitive, ce troisième chapitre nous permet une meilleure compréhension de l'impact d'Ambroise dans l'évolution des rapports entre l'Église et les empereurs. L'évêque de Milan est bien loin d'instaurer une théocratie ni même de contester l'autonomie du pouvoir politique, mais il parvient à travers ses réflexions théoriques et ses communications avec les détenteurs du pouvoir à développer une autorité épiscopale suffisante pour obliger les empereurs à se référer aux évêques lors des questions de foi, en leur interdisant notamment de prendre des initiatives. L'\latin{auctoritas} de l'Église n'est donc pas encore une institution politique officielle, mais commence à apparaître comme une réalité incontournable de la gestion de l'Empire. Bien qu'il s'adresse au nom de l'ensemble des représentants chrétiens, c'est principalement par ses propres relations, connaissances diplomatiques et popularité qu'Ambroise s'impose en tant que conseiller impérial influent, représentant définitivement cette autorité épiscopale qu'il recherche et expose.


\include{conclusion/conclusion}
\nocite{*}
% --- Bibliographie Divisée ---

% Titre général de la section
\chapter*{Bibliographie}
\addcontentsline{toc}{chapter}{Bibliographie}

% 1. Les Sources
\printbibliography[keyword=source, title={Sources}, heading=subbibliography]

% 2. Les Dictionnaires
\printbibliography[keyword=outil, title={Dictionnaires et Outils de recherche}, heading=subbibliography]

% 3. Les Études
\printbibliography[keyword=etude, title={Études}, heading=subbibliography]

\end{document}
