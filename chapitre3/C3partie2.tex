\section{Construction et utilisation d'une \latin{auctoritas} personnelle}

\subsection*{Introduction}

Si la définition des domaines de compétence et le développement d'une autorité épiscopale permettent à l'Église de s'affirmer comme une institution autonome au sein du gouvernement impérial, ce n'est pas pour autant suffisant pour garantir l'influence des évêques sur les décisions du souverain. Pour peser réellement dans la politique de l'Empire, l'évêque ne peut se contenter des affaires religieuses : il doit s'imposer comme un interlocuteur indispensable au cœur même du pouvoir. C'est ce qu'Ambroise a particulièrement bien compris, en cherchant à développer son influence pour mettre en œuvre ses différentes réflexions et théories sur la pratique du pouvoir. Cette deuxième partie du chapitre s'attache donc à analyser la construction d'une \latin{auctoritas} personnelle par Ambroise, une autorité qui dépasse la simple fonction cléricale pour devenir une composante essentielle du pouvoir impérial.

\bigskip

Chez Ambroise, chacune de ses interventions est calculée selon sa réflexion personnelle. Ainsi, il questionne et théorise toujours sa façon d'agir, jusqu'à son dialogue avec les empereurs. Il cherche à donner une dimension politique à ses interventions religieuses et, inversement, un aspect clérical à ses actions diplomatiques. Par cette approche, il parvient à instaurer une norme dans la pratique du pouvoir : le conseil régulier de l'évêque doit être vu comme obligatoire au sein d'un pouvoir chrétien et non plus comme une ingérence extérieure. De fait, Ambroise se présente lui-même en tête de file de ce modèle d'un évêque conseiller politique, lui permettant ainsi une place majeure aux côtés des souverains.

\bigskip

Dans l'objectif d'analyser toutes les formes de l'\latin{auctoritas} personnelle développée par l'évêque de Milan, nous étudierons d'abord les outils méthodologiques et rhétoriques qu'il utilise pour conforter son propos à travers le rappel de la \latin{libertas dicendi} et l'utilisation de sa popularité pour s'assurer une proximité avec les empereurs chrétiens. Dans un second temps, nous observerons la mise à l'épreuve concrète de cette \latin{auctoritas} à travers un rôle politique grandissant pour Ambroise ou encore l'aboutissement d'un contrôle de la politique impériale par l'exemplarité morale définie par l'idéal chrétien d'Ambroise.

\bigskip

\subsection{L'instauration du rapport de force : entre obligation morale et liens personnels}

\subsubsection{La « \latin{libertas dicendi} » : entretenir le dialogue avec les empereurs}

L'étude ambrosienne sur l'\latin{auctoritas} nous a permis de comprendre les fondements institutionnels de l'autorité épiscopale ainsi que la mise en place d'une nouvelle relation entre le pouvoir impérial et l'Église. Mais l'ensemble de ce cadre politique ne peut s'exercer qu'à travers un dialogue appronfondi entre les évêques et l'empereur. Le coeur de cette dynamique se trouve dans ce qu'Ambroise nomme la « \latin{libertas dicendi} », ce droit à la parole ancré dans la tradition romaine et réutilisé par Ambroise comme un principe chrétien. Pour l'évêque de Milan, cette liberté de parole n'est pas un simple privilège que l'Église revendique mais bien une obligation qui permet de mettre clairement l'évêque dans une position de conseiller légitime face au pouvoir. Cet aspect se place dans la continuité du développement d'une autonomie cléricale dans laquelle Ambroise théorise les droits et devoirs des évêques pour définir leur rôle et leur fournir un poid important dans la balance politique. C'est ce qu'évoque Müller dans cet extrait : Au fur et à mesure de ses lettres, il travaille à l’élaboration d’une conception chrétienne du pouvoir qui confère au directeur de conscience la tâche, tant dans la défense de l’intérêt personnel de son souverain que dans celle de la prospérité de l’État, de guider l’empereur de ses conseils dans les décisions que celui-ci doit prendre, de lui rappeler ses devoirs de bon chrétien et, le cas échéant, de le rappeler à l’ordre\autocite[219]{muller_conflits}.

\bigskip

L'objectif principal de cette réflexion politique pour l'évêque de Milan est d'augmenter l'autorité et l'influence de l'Église, et par la même occasion la sienne pour se retrouver avec les mains de plus en plus libre dans l'Empire. Sa lutte doit alors se concentrer sur un point précis : éviter la censure que peuvent imposer les empereurs aux différents évêques. C'est pour cette raison qu'il est question d'obligation dans la liberté de parole : l'intervention de l'évêque, comme conseiller impérial, se développe comme un point à part entière de son rôle dans l'Empire. L'ensemble de cette théorie prend évidemment naissance dans l'action d'Ambroise, notamment lors de son intervention dans l'affaire de Callinicum, risquée puisque allant à l'encontre du choix de l'empereur. Deux passages de sa lettre à Théodose nous permettent d'établir avec précision sa pensée sur le sujet du dialogue entre évêques et empereurs. Tout d'abord avec cette idée qui semble relier la parole des évêques à la légitimité de l'empereur d'exercer son pouvoir :

\bigskip

\begin{quote}
    «~Mais il ne convient ni à un empereur de refuser la liberté de parler ni à un évêque de ne pas dire ce qu’il pense. [...] Chez un évêque aussi, rien n’est plus dangereux devant Dieu, ni plus honteux devant les hommes, que de ne pas déclarer librement ce qu’il pense.\footnote{« \latin{Sed neque imperiale est libertatem dicendi negare neque sacerdotale quod sentiat non dicere. [...] Nihil etiam in sacerdote tam periculosum apud Deum, tam turpe apud homines quam quod sentiat non libere denuntiare.}~». \cite[\nopp 74,2]{ambroise_lettres}.}~»
\end{quote}

\bigskip

Et quelques lignes plus loin l'explication de ce à quoi doit aboutir la liberté de parole :

\bigskip

\begin{quote}
    «~Je préfère donc, empereur, partager avec toi le bien plutôt que le mal, c’est pourquoi le silence de l’évêque doit déplaire à ta Clémence, mais sa franchise lui plaire. [...] Je ne me mêle pas en importun de ce qui ne me regarde pas, je ne m’ingère pas dans les affaires d’autrui, mais j’obéis à mon devoir, je me soumets aux commandements de notre Dieu.\footnote{« \latin{Malo igitur, imperator, bonorum mihi esse te cum quam malorum consortium et ideo clementiae tuae displicere debet sacerdotis silentium libertas placere. [...] Non ergo importunus indebitis me intersero, alienis ingero, sed debitis obtempero, mandatis Dei nostri oboedio.}~». \cite[\nopp 74,3]{ambroise_lettres}.}~»
\end{quote}

\bigskip

Plusieurs points majeurs de la pensée ambrosienne sont à relever dans ces citations. Les écrits de l'évêque de Milan oscillent en permanence entre réflexion politique large et rhétorique utile dans un cadre bien précis, les deux se mélangeant souvent dans ses correspondances. Une chose est certaine, il ancre en permanence ses besoins utilitaires dans des théories politiques précises qu'il cherche à appliquer à grande échelle. Ainsi dans cette lettre ce n'est pas seulement avec sa personne, pourtant proche des empereurs et évêque particulièrement important, qu'il essaie d'introduire l'idée de la contestation épiscopale, mais bien avec l'ensemble des ministres de la foi. Il légitime par ce biais l'ensemble de l'\latin{auctoritas} de l'Église et en appel à ce droit de parole. Dieu est invoqué comme source d'autorité incontestable : si il est permis par Dieu de parler librement, alors l'empereur n'a pas le pouvoir de s'y opposer. Mais surtout, Ambroise légitime sa prise de parole par des références constantes aux Écritures. Nous l'avons déjà évoqué, l'évêque de Milan utilise les textes de la Bible comme un traité politique qui peut et doit servir d'exemple à la société chrétienne romaine. Ainsi, dans les deux passages cités, il s'appuie sur des références scripturaires précises, telles que le Psaume 118 : « Devant les rois je parlerai de ton témoignage et je n'aurai nulle honte.\footnote{Psaume 118, 46.} » ou la deuxième lettre à Timothée : « Prêche la parole, insiste en toute occasion, favorable ou non, reprends, censure, exhorte, avec toute douceur et en instruisant.\footnote{2 Timothée 4, 2.}~». Le silence se présente donc comme une faute morale et un manquement aux exigences du rôle d'évêque.

\bigskip

Mais cette « \latin{libertas dicendi} ne doit pas pour autant être vu comme de la contestation politique ou de l'ingérence. Tout comme le développement de l'\latin{auctoritas} épiscopale, la parole et le conseil de l'évêque ne dépasse pas chez Ambroise la sphère d'action de l'Église, et donc les affaires religieuses. La mission principale d'un évêque, et c'est ce que nous allons étudier tout au long du reste de ce chapitre, reste d'empêcher autrui de péché, et à plus forte raison si cet évêque est proche de l'empereur. La deuxième citation de l'\latin{Epistolae} 74 permet à Ambroise de rassurer Théodose sur ses intentions. Il utilise cette fois la première personne pour montrer qu'il n'intervient que dans un cadre légitime, en phase avec ses droits. En réalité, le fait même qu'il soit obligé de théoriser son intervention et de la replacer dans une pensée politique plus large faisant appel à la liberté de parole des évêques, montre bien qu'il n'est pas rassuré et donc qu'il se place aux limites de son autorité. Bien qu'il s'agisse d'une réflexion politique pleinement aboutie, elle est ici et comme souvent utilisée comme une arme d'action immédiate.

\bigskip

Justemment, pour sortir du cadre « utile » de la pensée ambrosienne, il est intéressant de piocher une citation issue de son commentaire sur certains Psaumes. Ambroise apparaît comme conscient de la nécessité de modérer cette liberté des évêques, afin qu'elle se fasse dans le cadre le plus utile et juste possible :

\bigskip

\begin{quote}
    «~Tu vois donc qu'il ne faut pas que les prophètes de Dieu ou les prêtres fassent injure aux rois à la légère, s'il n'y a pas de fautes assez graves pour qu'ils doivent être repris. Mais là où les fautes sont plus graves, il ne semble pas que le prêtre doive les épargner, afin qu'ils soient corrigés par de justes réprimandes.\footnote{« \latin{Vides ergo, quia regibus non temere vel a prophetis Dei vel a sacerdotibus facienda iniuria sit, si nulla sint graviora peccata in quibus debeant argui. Ubi autem peccata graviora sunt, ibi non videtur a sacerdote parcendum, ut iustis increpationibus corrigantur.} » \cite[\latin{Enarratio in psalmum} 37,43. Page 172]{ambroise_csel_64}.}~»
\end{quote}

\bigskip

La figure de David, puisqu'Ambroise l'utilise comme auteur du Psaume 37, permet de parler aussi bien aux empereurs qu'aux évêques. Il tire ainsi des concepts s'adressant aux deux institutions et utilise son commentaire comme outil de réflexion politique qui doit s'appliquer dans l'Empire. D'une façon générale, ses homélies sur les Psaumes, « tout en étant attentive à marquer leur sens messianique, sont soucieuses d'interprétation morale liée concrètement à la situation eccleśiale et politique du moment.\autocite[223]{quasten_initiation}~». Par cette citation, il limite concrètement les raisons pouvant pousser aux interventions épiscopales, ce qui a pour effet de valoriser considérablement ces fameuses interventions qu'il qualifie lui même de « nécessaire réprimande ». Il est difficile de dater avec précision les homélies sur douze des Psaumes de David, mais Palanque semble indiquer que ce commentaire est sans doute prononcé autour de 389. Ce qui signifie qu'Ambroise écrit ce passage après l'affaire de Callinicum. Bien qu'il ne s'adresse pas aux empereurs, il accentue donc la théorie politique de liberté de parole en placant l'intervention d'Ambroise auprès de Théodose comme une correction liée à une « faute grave ».

\bigskip

Dans ses textes destinés à son clergé, Ambroise appuie tout de même bien sur les règles qui découlent du rôle de conseiller. Il joue donc grandement sur le fait d'encadrer les façons d'intervenir auprès des empereurs, afin de rendre ces moments plus fort et impactant, comme ici dans le deuxième volume de son \latin{De Officiis} :

\bigskip

\begin{quote}
    «~Aussi celui qui veut donner conseil à autrui doit être tel qu’il fournisse en lui-même, aux autres, un modèle pour « l’exemple des bonnes oeuvres, dans sa doctrine, dans sa chasteté, dans son sérieux » que sa conversation soit salutaire et irréprochable, son conseil utile, sa vie belle et son avis convenable.\footnote{«~\latin{???}.~». \cite[\nopp II, 86]{ambroise_devoirs_2}.}~»
\end{quote}

\bigskip

Par ce texte ou celui sur les Psaumes, on peut donc constater qu'il n'y a pas chez Ambroise de liberté sans restriction. Le rôle politique qu'il façonne chez les évêques doit suivre une réussite personnelle dans la foi, la morale et la justice afin d'apporter aux détenteurs du pouvoir impérial les conseils les plus à même d'aider à la prospérité de l'Empire. Bien que l'évêque de Milan semble tout faire pour développer son propre pouvoir et autorité au sein du gouvernement impérial\footnote{C'est principalement le coeur de la suite de ce chapitre}. Il ne cherche pas à rendre l'Église chrétienne toute puissante, mais simplement à créer une autorité religieuse capable d'agir concrètement pour le bien de l'Empire.Ainsi, la revendication d'une \latin{libertas dicendi} marque un point central dans la pensée politique ambrosienne.

\bigskip

Ces droits à la parole et aux conseils peuvent également servir à encadre les actions impériales pour les guider vers une justice commune. Dans son court passage sur l'évêque de Milan, Giuseppe Zecchini analyse la réflexion d'Ambroise sur le pouvoir impérial et se penche notamment sur le risque de voir le souverain devenir un tyran. Face à ce risque, la \latin{libertas dicendi} intervient comme la « première forme et expression de toute liberté politique\autocite[165]{zecchini_pensiero_politico}~». Avec Ambroise, la parole de l'évêque se place comme un outil de contrôle du pouvoir impérial. Zecchini note que, tout comme l'\latin{auctoritas}, cette fonction semble passer des mains des sénateurs aux « \latin{sacerdoti di Cristo} », les prêtres du Christ. Le Sénat, affaibli par la toute puissance de l'empereur, ne s'affiche plus comme le garant de la liberté face au prince. L'évêque, par cette fondamentale liberté de parole utilisée par Ambroise, obtient donc ce devoir religieux d'imposer une limitation morale aux volontés impériales.

\bigskip

Ainsi, la position de l'évêque auprès de l'empereur se définit entre autres par son devoir d'interpellation qui ne peut être supprimé. Le dialogue se doit donc d'être entretenu pour respecter le droit à la liberté de parole, qui fonde en partie le statut politique de l'Église. Le pouvoir romain chrétien est même légitimé par ce droit que possèdent les évêques à exprimer de « justes remontrance », qui permet au souverain de rester dans le chemin de la foi et d'éviter les péchés qui guettent le pouvoir.

\bigskip

\subsubsection{La proximité d'Ambroise avec les princes chrétiens}

Si la \latin{libertas dicendi} offre aux évêques le droit théorique de s'exprimer, tous n'ont pas pour autant la même autorité ou influence face au pouvoir impérial. Bien qu'ils puissent être reçus et entendus, seule la qualité de la relation tissée avec le souverain garantit une véritable écoute et prise en compte de cette parole. Ambroise est évidemment conscient des limites de l'\latin{auctoritas} institutionnelle de l'Église face à la réalité du pouvoir et s'attache fortement à développer un lien personnel important, dépassant le statut politique, avec les empereurs pour conforter son autorité au sein de la cour. L'idée est simple et résume parfaitement la vision qu'a Ambroise de s'impliquer dans la vie politique : si chacune de ses actions est encadrée par un cadre théorique poussé, il n'en oublie jamais les réalités immuables de la société, transformant alors son poste d'évêque étranger à la cour en un personnage intimement lié aux souverains qui ne peut s'en passer. Cette proximité humaine avec les détenteurs du pouvoir ne signifie pas pour autant un asservissement total envers eux. Au contraire, Ambroise a confiance en son rôle et en ses capacités et use de ses relations pour exprimer pleinement et librement ses pensées.

\bigskip

Cette dynamique s'observe dès le début de son épiscopat, à travers ses premiers rapports avec Gratien. La relation entre les deux débute par une demande de l'empereur qui souhaite avoir un exposé sur la foi trinitaire afin de mieux comprendre les tensions existantes entre les courants chrétiens. Et déjà, les réponses de l'évêque se font attendre, la correspondance prend du temps et la rédaction du \latin{De Fide} n'est pas précipitée par Ambroise qui ne semble pas réagir à la pression d'une demande impériale. Comme le note Yves-Marie Duval, «~Ambroise n’est pas le courtisan zélé qui se hâte de satisfaire la demande d’un puissant ou qui l’encombre de ses productions.\autocite[223]{duval_lettres_gratien}~». Ce refus de la course à la production le place dans une position de force : c'est bien l'empereur qui doit relancer l'évêque dans sa requête d'apprentissage de la foi. L'état de cette relation est un signe de confiance absolue de Gratien envers Ambroise dans sa capacité à répondre aux problèmes religieux puisque l'empereur ne cherche pas de réponse auprès d'autres évêques, s'appuyant sur l'autorité certaine d'Ambroise en Italie. Avec Gratien, Ambroise nourrit une relation de respect et de dépendance intellectuelle, valorisant son statut à la cour impériale.

\bigskip

Avec Valentinien II, empereur beaucoup plus jeune, Ambroise ajoute à sa posture d'autorité religieuse celle du père et maître à penser pour réussir à forger une influence durable. En revanche, cette relation met du temps à se mettre en place à cause de l'influence de la mère du prince, Justine, fervente défenseuse de la foi arienne. Ambroise se retrouve à intervenir à plusieurs reprises pour défendre la foi chrétienne et même nicéenne, lors des affaires de l'autel de la Victoire en 384 ou des basiliques de Milan en 386. Et pourtant, après la mort du jeune empereur par suicide ou assassinat en 392, Ambroise insiste dans son \latin{De Obitu Valentiniani} sur la tristesse qui l'envahit, signe d'une proximité certaine bien que sûrement en partie exagérée. Ce changement de comportement est notamment à relier à la mort de Justine en 388 et aux «~pressions discrètes de Théodose~» pour arriver à ce respect filial et cette affection paternelle\autocite[Introduction]{ambroise_mort_valentinien}. Cette proximité avec Valentinien est notamment exprimée dans la Lettre 25 à destination de Théodose, dans laquelle il revient sur la mort de l'empereur, pour rappeler son rôle auprès du pouvoir impérial en Occident :

\bigskip

\begin{quote} «~Il professait qu'il devait son éducation à moi, il me désirait comme un père attentif, et lorsque certains prétendaient avoir reçu des nouvelles de mon arrivée, il les anticipait avec impatience. D'ailleurs, pendant ces jours mêmes de deuil public, bien qu'il eût sous la main des évêques saints et éminents dans les limites de la Gaule, il jugea néanmoins bon de m'écrire pour que je lui confère le Sacrement du Baptême. Par cette demande, sinon raisonnablement, du moins affectueusement, il témoignait de son amour envers moi.\footnote{« \latin{Ille se a me nutritum praeferebat, ille ut sedulum patrem desiderabat, ille simulato a quibusdam adventus mei nuntio inpatienter praestolabatur. Quin etiam illis ipsis publici doloris diebus, cum sanctos et summos sacerdotes domini intra Gallias haberet, ut a me tamen sacramentis baptismatis initiaretur, scribendum arbitratus est ; quod etsi non rationabiliter, amabiliter tamen erga me suum studium testificatus est.}~». \cite[\nopp 25, 2.]{ambroise_csel_82_1}. {Traduction personnelle.}}~» \end{quote}

\bigskip

Le terme de « père attentif » est le plus intéressant, Ambroise n'est pas un simple évêque remplaçable au sein de la cour, mais bien celui qui offre aussi bien l'éducation que le Salut aux empereurs. Comme toujours, il est difficile de savoir la part d'exagération dans les propos d'Ambroise, surtout que nous ne possédons pas la demande de baptême par Valentinien, mais une intimité, bien que régulièrement tendue, existe, ou du moins est recherchée par l'évêque. En effet, plus que de savoir la réalité de la relation qu'il entretient avec l'empereur, il est intéressant de comprendre qu'Ambroise insiste sur le fait qu'il a toujours été proche, afin de développer une autorité auprès de tous qui dépasse le cadre institutionnel. Et c'est cette position qui lui permet d'agir aussi fermement dans l'affaire de l'autel de la Victoire. Il cherche à se différencier de son opposant Symmaque en adoptant un ton personnel dans ses lettres, comme pour démontrer une forme d'intimité dans la relation, que les arguments politico-religieux du sénateur ne peuvent dépasser. Dans son article sur la gestion des conflits par Ambroise, Gernot Michael Muller se permet même d'aller plus loin dans l'importance de la démonstration de l'intimité chez Ambroise : « C’est donc moins l’exercice réel du pouvoir sur son destinataire qui importait à Ambroise, que le fait de marquer implicitement la position de force qui était la sienne et qui dérivait d’une place privilégiée par rapport à ce même destinataire, à savoir celle du directeur de conscience.\autocite{muller_conflits}~». L'idée d'un «~directeur de conscience~» regroupe parfaitement les deux aspects de l'éducation et de la foi, et c'est ce qu'il vise avec chacun des empereurs qu'il côtoie. La relation personnelle apparaît ainsi comme un levier d'action politique, qu'Ambroise utilise en parallèle de ses rôles religieux et politiques intrinsèques à sa position d'évêque de Milan.

\bigskip

En revanche, avec Théodose, construire une relation personnelle forte se révèle être un défi plus important. Général victorieux, empereur en Orient, régulièrement en concurrence avec les pouvoirs impériaux de Gratien ou Valentinien II, Théodose ne propose pas le même profil capable d'écouter et de suivre facilement les conseils de l'évêque de Milan, et pourtant Ambroise se permet tout autant d'intervenir et de parler au nom du Salut de l'empereur. L'idée cette fois n'est plus de se mettre dans une position de père ou d'ami, mais d'égal et de conseiller, capable d'apporter un jugement moral lucide et juste. La seule position d'évêque ne lui suffit plus : si en 388, après sa lettre pour réprimander la décision de Théodose au sujet de la synagogue de Callinicum, il réussit à obtenir de Théodose le pardon à un évêque, il connaît un échec virulent en début d'année 390 alors que l'empereur refuse de le consulter suite au massacre de Thessalonique\footnote{Voir plus de détail sur Ambroise et l'affaire de Thessalonique dans le 2.2.3.}. Ambroise ne peut donc plus simplement se reposer sur l'autorité innée de son rôle d'évêque. C'est ce qu'analyse avec justesse Peter Brown à partir de la page 152 de son livre \latin{Pouvoir et persuasion dans l'Antiquité tardive} en montrant qu'Ambroise doit relancer sa relation avec Théodose avec « le courage d'un philosophe. » Brown parle ici des deux approches ambrosiennes envers le pouvoir, fondé sur l'image de la Grèce antique du philosophe affrontant l'autorité politique : « Ambroise se présentait comme l’exemple chrétien de l’ancienne \latin{karteria}, l’obstination inspirée avec laquelle les philosophes affrontaient le puissant. [...] Avec Théodose, le temps était venu de la \latin{parrhésia}, du franc-parler.\autocite[155]{brown_power_1992}~». Ambroise se montre comme le conseiller éclairant l'empereur dans le chemin de la foi.

\bigskip

Évidemment, même avec Théodose, l'évêque de Milan cherche à jouer de sa proximité avec le pouvoir impérial pour mieux se faire entendre, comme le montrent les premiers mots de sa lettre pour demander à l'empereur de se repentir : « Le souvenir de notre ancienne amitié m'est doux.\footnote{« \latin{Et veteris amicitiae dulcis mihi recordatio est}~». \cite[\nopp ec. 11, 1]{ambroise_csel_82_3}.}~». Il est même question ici d'un temps long, comme pour justifier le fait que ses propos ne sont pas là pour entraver la vie de l'empereur. Les différentes lettres aux empereurs nous permettent donc de saisir avec précision les relations qu'Ambroise a su tisser avec le pouvoir impérial tout au long de son épiscopat, qui se révèlent tout aussi importantes, si ce n'est plus, que l'aspect institutionnel de l'épiscopat, la foi et l'Église dans son ensemble.

\bigskip

Finalement, le meilleur exemple dans la vie d'Ambroise de l'importance de l'entretien des liens intimes avec les souverains de l'Empire se situe dans son échec auprès d'Honorius, et plus précisément de Stilicon. À la mort de Théodose en 395, la structure politique de l'Empire est modifiée. Alors qu'il s'était fait maître de tout le territoire romain suite à la mort de Valentinien II et surtout suite à la défaite d'Eugène en 394, Théodose laisse son pouvoir dans les mains de ses deux fils : Arcadius âgé de 18 ans qui s'empare de l'Orient, et Honorius, seulement âgé de 10 ans, se retrouve avec l'Occident. Honorius est d'ailleurs présent à Milan lors de l'enterrement de Théodose, point important puisqu'une partie de l'Oraison Funèbre d'Ambroise est dédiée au très jeune héritier. Pourtant, Ambroise ne parvient pas à reproduire le schéma de proximité paternelle qu'il avait employé avec Valentinien II. L'évêque doit à ce moment faire face à un nouvel adversaire politique : Stilicon, général romain proche de Théodose qui affirme avoir reçu la régence de l'Empire. L'Orient échappe tout de même rapidement à son contrôle à cause de l'opposition d'Arcadius plus grand et donc apte à se présenter comme empereur, et du préfet du prétoire Rufin. En revanche, Honorius passe bien sous la tutelle de Stilicon, ce qui lui permet d'imposer son pouvoir dans les provinces d'Occident, reléguant le jeune prince à un rôle pratiquement protocolaire.

\bigskip

Alors que sur la fin du règne de Théodose l'influence ambrosienne semble au plus haut au sein du pouvoir impérial, son \latin{auctoritas} est contestée par Stilicon. En effet, celui-ci se légitime par son mariage avec la nièce de Théodose, son rôle de commandant militaire ainsi que le soi-disant legs de l'autorité impériale par l'empereur défunt. Stilicon ne semble pas avoir besoin d'une légitimité religieuse que peut apporter l'évêque de Milan, et choisit de l'écarter progressivement de son rôle politique en se montrant comme le véritable tuteur d'Honorius\autocite[à partir de la page 298]{palanque_ambroise}. Ambroise se voit donc forcé de revenir à un rôle purement clérical pour les derniers mois de sa vie. Le signe le plus flagrant de cette perte d'influence et de rôle politique est l'absence de lettre adressée à Stilicon ou Honorius pendant la durée de leur règne. Comme l'explique Neil McLynn, ce silence d'Ambroise, qui ne rejette ni n'adhère à la politique menée par le régent, ne doit pas être simplement perçu comme un manque de source, mais bien comme une preuve d'une fissure entre l'\latin{auctoritas} d'Ambroise et la \latin{potestas} de l'empereur\autocite[366]{mclynn_ambrose}.

\bigskip

En ayant conscience de cet aspect de la fin de vie d'Ambroise, il est intéressant de relire le \latin{De Obitu Theodosii}, non pas comme une simple louange à Théodose mais bien, comme le dit McLynn, comme une tentative politique de récupération par Ambroise de la tutelle d'Honorius et donc de se placer comme son mentor à la place de Stilicon\autocite[358-360]{mclynn_ambrose}. L'appui dans son discours d'une continuité entre Théodose et ses fils, aussi bien dans la politique impériale que dans la foi\footnote{\cite[\nopp 6-7]{ambroise_mort_theodose}.} a pour objectif, en plus d'établir une stabilité politique, de faire apparaître la tutelle d'Ambroise comme évidente. Puisque l'évêque était un proche conseiller de Théodose, il est normal qu'il le reste pour ses fils. Les volontés d'instructions et de préoccupations morales d'Ambroise envers les deux jeunes souverains sont donc certainement un simple outil politique pour garantir son rôle à la cour, ce qui s'avère être un échec. Stilicon s'empare pleinement du pouvoir en Occident et fait disparaître l'\latin{auctoritas} ambrosienne.

\bigskip

L'ensemble des relations d'Ambroise avec les empereurs de la fin du IVème siècle agit comme un révélateur des limites de l'\latin{auctoritas} épiscopale. Aussi importante soit-elle dans le discours d'Ambroise, le rôle politico-religieux de l'Église au sein de l'Empire reste principalement dépendant du bon vouloir du souverain, et donc de la réussite ou non dans la création d'un lien intime entre un évêque et son empereur. Cette compréhension par Ambroise des réalités politiques lui permet de mettre en place ses théories et réflexions et de perdurer, sous différentes postures, au coeur du jeu politique de l'Empire.

\bigskip

\subsubsection{La maîtrise des armes populaires et liturgiques}

Pour autant, lorsque les relations personnelles ne suffisent pas à s'imposer politiquement, et que les ordres impériaux viennent confronter les idées sociétales ou religieuses d'Ambroise, l'évêque de Milan ne se retrouve pas démuni et sait user d'armes variées pour arriver à ses fins. Il délaisse son \latin{auctoritas} politique pour revenir à un rôle d'évêque charismatique de la capitale occidentale. Au cours de la crise arienne de 386, le pouvoir impérial cherche de plus en plus à utiliser la force pour soutenir les objectifs de Justine et d'Auxence. Ambroise fait alors appel à un outil d'influence et d'autorité qui lui est propre : la maîtrise de la ferveur populaire et de l'espace liturgique. L'idée étant d'opposer à l'empereur les symboles du peuple et de la foi pour rendre impossible toute intervention.

\bigskip

Dans sa description de l'affrontement de 386, Jean-Rémy Palanque illustre cet appui sur la force populaire pour lutter contre les mesures juridiques que parvient à mettre en place Justine en faveur de la foi arienne, notamment une loi de janvier 386 accordant la liberté de culte aux « tenants de la foi de Rimini\autocite[150]{palanque_ambroise}.~». Il est difficile de savoir si Valentinien est allé jusqu'à demander à Ambroise de s'exiler, mais une chose est certaine, l'évêque de Milan préfère l'affrontement direct avec le pouvoir impérial, plutôt que la défaite de ses convictions religieuses, comme on peut le constater dans sa lettre 75 à Valentinien :

\bigskip

\begin{quote}
    «~À présent les évêques me disent qu'il n'y a pas grande différence entre quitter volontairement l'autel du Christ et le livrer, car le quitter, c'est le livrer.\footnote{«~\latin{Nunc mihi sacerdotibus dicitur non multum interesse utrum uolens relinquas an tradas altare Christi, cum enim reliqueris trades.}~». \cite[\nopp 75, 18]{ambroise_lettres}.}~»
\end{quote}

\bigskip

Selon Gérard Nauroy, Valentinien n'aurait pas pu le condamner à l'exil alors qu'il lui demande de venir débattre au Consistoire, ça serait contradictoire. Ainsi, la demande d'exil dont il est question dans les écrits de Paulin sur la vie d'Ambroise provient sûrement de la non prise en compte du moment précis de la requête. Dans sa lettre, Ambroise feint de regretter que la demande d'exil arrive trop tard, et qu'il aurait été enclin à ne pas résister plus tôt, mais qu'il est désormais trop tard : «~Tu aurais dû m'envoyer où tu voulais, car je m'offrais moi-même à tous.\footnote{«~\latin{Debuisti me quo uolueras destinare, quem ipse omnibus.}~». \cite[\nopp 75, 18]{ambroise_lettres}.}~»

\bigskip

Mis à mal par Justine et Valentinien, Ambroise choisit de s'enfermer dans la basilique, soutenu par la foule milanaise très attachée à sa figure. C'est dans ce contexte qu'il met en place une nouvelle arme pour parvenir à imposer ses idées : la cohésion par la liturgie. Augustin, proche des événements notamment par le biais de sa mère, raconte dans ses \latin{Confessions} comment Ambroise instaure le chant des hymnes et des psaumes « comme cela se fait en Orient », pour éviter que la force populaire ne se fasse emporter par la peur ou l'ennui. Il ne s'agit donc pas seulement d'un caprice dans la pratique religieuse, mais bien d'une tactique politique en faveur de l'évêque de Milan. Augustin écrit sur cette ferveur comme signe de la toute-puissance ambrosienne à Milan : «~La foule des pieux fidèles passait les nuits dans l'Église, prête à mourir avec son évêque, votre serviteur.\footnote{«~\latin{...}~». \cite[IX, 7]{augustin_confessions}.}~» Le pouvoir impérial se retrouve paralysé par la simple présence de la population, amenant à une nouvelle démonstration de force d'Ambroise et de son \latin{auctoritas}.

\bigskip

Cependant, pour dépasser le rejet qu'a Valentinien II de la \latin{libertas dicendi} et de la relation personnelle, et s'imposer définitivement comme la source d'autorité à Milan, Ambroise doit chercher plus loin que la simple résistance « passive » avec le soutien du peuple. Ainsi, arrive au «~bon moment~» la découverte des corps des martyrs Gervais et Protais en juin 386. C'est un passage que regarde avec beaucoup de critique Neil McLynn, qui souligne qu'Ambroise, politiquement isolé, avait besoin d'une forme de validation divine pour montrer que sa position était celle à suivre dans la foi chrétienne, et donc que son opposition dangereuse au pouvoir impérial était juste et légitime. Cette découverte d'importance pour la religion chrétienne agit comme une source d'autorité majeure, tout en rassemblant d'autant plus la population milanaise derrière ses idées et la foi catholique, contre l'objectif impérial : «~Le thème de l'unité, propre à Ambroise, était de toute façon bien choisi pour séduire ceux qui étaient soucieux de rétablir la concorde à Milan. L'événement constituait tout autant une démonstration opportune de l'engagement de l'évêque envers cette cause qu'un déploiement triomphal de la force de son parti. C'est là que réside l'explication de sa réussite à briser la frénésie de la persécution.\autocite[215]{mclynn_ambrose}~»

\bigskip

Évidemment, Ambroise défend la véracité de cette découverte face aux accusations de tromperie émanant du camp arien, mais dans une de ses lettres à sa sœur Marcelline, il ne se cache pas pour autant d'utiliser les figures des martyrs comme une arme politique contre la force armée :

\bigskip

\begin{quote}
    «~Grâce te soient rendues, Seigneur Jésus, de nous avoir suscité une telle puissance spirituelle des saints martyrs en ce moment où ton Église ressent le besoin de plus grandes protections. [...] Les uns se glorifient de leurs chars et les autres de leurs chevaux, mais nous, nous nous glorifierons du nom de notre Seigneur Dieu.\footnote{«~\latin{Gratias tibi, Domine Iesu, quod hoc tempore tales nobis sanctorum martyrum spiritus excitasti, quo ecclesia tua praesidia maiora desiderat. [...] Hi in curribus et hi in equis, nos autem in nomine domini dei nostri magnificabimur.}~». \cite[\nopp 77, 10]{ambroise_lettres}.}~»
\end{quote}

\bigskip

Par l'opposition entre les «~chars et chevaux~» et la «~puissance spirituelle~», Ambroise retire à Valentinien la légitimité de l'usage de la force publique : Dieu s'est prononcé en faveur de la foi de Nicée. On peut également noter que dans cette lettre, de nouveau, l'évêque de Milan se rattache à un passage des Écritures pour renforcer sa position : la victoire de Ghiézi contre les Syriens, grâce au soutien de Dieu et des «~soldats du Christ\footnote{\cite[\nopp 77, 11]{ambroise_lettres}.}~».

\bigskip

Cet épisode nous permet de prendre en compte un point essentiel de l'\latin{auctoritas} ambrosienne : la capacité à mobiliser différentes armes politiques, liturgiques, rhétoriques, populaires, pour arriver à ses fins et faire en sorte que ses réflexions politiques se rapprochent de la réalité de l'Empire. Comme le note Peter Brown, Ambroise possède un atout que les philosophes antiques ne possèdent pas, et qui lui permet d'outrepasser la plupart des situations : il est le «~maître de la basilique~» et donc la plus grande autorité d'un espace aussi important pour le peuple que pour l'empereur en tant qu'outil de mise en scène du pouvoir impérial\autocite[156]{brown_power_1992}. Le contrôle de la foi couplé à l'appui populaire qu'il possède à Milan, fait d'Ambroise une autorité difficilement contestable, même lorsque la \latin{libertas dicendi} ne suffit pas.

\subsection{L'aboutissement de l'\latin{auctoritas} personnelle}

\subsubsection{L'évêque au service de la stabilité impériale}

Ce qui est intéressant au sujet de cette \latin{auctoritas}, c'est qu'elle permet à Ambroise d'atteindre des sommets de l'influence dans la partie occidentale de l'Empire, que ce soit par des actions pour la défense de la foi, par la démocratisation d'un statut de conseiller, ou bien par des événements véritablement politiques demandés par l'empereur lui-même. Comme le souligne Peter Brown tout au long de son ouvrage \latin{Pouvoir et Persuasion}, la fin du IV\textsuperscript{e} siècle marque le début d'un tournant dans l'exercice de l'autorité, et Ambroise en est un des instigateurs majeurs. L'autorité morale et locale qu'apportent les évêques tend à remplacer les philosophes et conseillers politiques laïcs, signe d'un changement important dans la perception de l'influence\autocite{brown_power_1992}. Et c'est cet aspect de l'\latin{auctoritas} qui amène Ambroise à endosser à deux reprises, en 383 et en 384 ou 386, le costume d'ambassadeur diplomatique agissant au nom de l'empereur Valentinien II. Bien que la relation entre le pouvoir impérial et l'évêque de Milan soit plutôt dans un moment de tension entre 383 et 387, c'est bien vers la figure populaire et intellectuelle d'Ambroise que Valentinien se tourne pour se rendre à Trèves et s'assurer du maintien de la paix avec l'usurpateur Maxime, signe d'une reconnaissance de ses capacités politiques et oratoires, ainsi que de sa maîtrise des codes sociaux de l'aristocratie romaine dont il est issu.

\bigskip

La première mission diplomatique qui lui est confiée date de 383 et, bien que l'on ne possède pas de lettre ou de texte d'Ambroise sur cet événement, semble être une réussite. Alors que l'empereur Gratien vient d'être assassiné par les troupes de Maxime, celui-ci se revendique empereur en s'installant à Trèves et s'empare des provinces les plus à l'ouest de l'Empire. Alors que l'usurpateur parvient rapidement à se faire reconnaître comme Auguste par Théodose\footnote{Hervé Savon explique cette reconnaissance par le fait que Théodose ne regrettait pas particulièrement la mort de Gratien. Celui-ci, malgré son jeune âge, se montrait régulièrement comme le supérieur et n'hésitait pas à intervenir dans les affaires de l'épiscopat oriental en menant une politique religieuse parfois opposée. Reconnaître Maxime comme empereur légitime lui permettait donc de se débarrasser d'une menace importante. \cite[181]{savon_ambroise}.}, le jeune empereur à Milan envoie Ambroise pour s'assurer du maintien de la paix et éviter que Maxime ne jette son dévolu sur l'Italie. C'est une des premières fois qu'un évêque joue officiellement un rôle diplomatique au service de l'Empire en mettant ses capacités pour protéger la paix entre Valentinien II et Maxime.

\bigskip

Mais notre intérêt pour comprendre l'aboutissement de l'\latin{auctoritas} ambrosienne doit se faire en se concentrant principalement sur la deuxième ambassade d'Ambroise, puisqu'il s'agit de la plus documentée, notamment grâce à l'\latin{Epistolae} 30 où Ambroise apporte à Valentinien un compte-rendu de sa mission. L'objectif, en plus de continuer à s'assurer du maintien de la paix et d'un \latin{statu quo} sur les possessions territoriales, est de récupérer la dépouille de Gratien afin de lui rendre les honneurs liés à son statut. Il est important de noter qu'un débat historiographique existe au sujet de la datation précise de cette ambassade. La lettre d'Ambroise n'est pas datée, et il est donc nécessaire de s'appuyer sur d'autres sources, extérieures ou non à l'évêque de Milan. Jean-Rémy Palanque, qui s'appuie sur la chronologie des écrits de Paulin de Milan, la situe en fin d'année 386 ou début 387, après la résolution du conflit contre les ariens. Hervé Savon, en revanche, propose une datation plus précoce, autour de 384, donc peu de temps après la première ambassade. Il défend sa théorie dans sa biographie d'Ambroise en faisant appel à un extrait du \latin{De Obitu Valentiniani} :

\bigskip

\begin{quote} «~Il a été doux pour moi de s'acquitter de cette fonction, la première fois pour vous sauver, la seconde pour la paix et la piété avec laquelle vous demandiez les restes de votre frère : vous n'étiez pas encore en sécurité, et déjà vous vous souciiez de donner à votre frère les honneurs de la sépulture\footnote{« \latin{Dulce mihi officium fuit, quod pro salute tua primum suscepi, deinde pro pace, et pietate, qua fraternas reliquias postulabas: necdum pro te securus, et iam pro fraterni funeris honore sollicitus.}~». \cite[\nopp 28]{ambroise_mort_valentinien}.}.~» \end{quote}

\bigskip

Pour Savon, l'utilisation du terme «~déjà~» dans ce passage où il évoque les deux missions qui lui ont été confiées, constituerait une « ironie déplacée » si Valentinien avait attendu trois ans pour réclamer le corps de Gratien\autocite[200]{savon_ambroise}. Il est difficile d'avoir un avis tranché sur ce sujet, bien que l'échec d'Ambroise sur cette ambassade soit certainement l'un des éléments ayant précipité la campagne de Maxime contre Valentinien en Italie, ce qui pousserait à une datation tardive, en début 387, malgré les tensions toujours existantes entre l'évêque et le pouvoir impérial.

\bigskip

C'est donc par le compte-rendu de son ambassade dans la lettre 30, qu'il a certainement rédigée par peur d'être vu comme un traître aux yeux de Valentinien après son échec à récupérer la dépouille, qu'Ambroise théorise le plus en profondeur sa conception du service de la politique impériale. L'évêque de Milan se montre ici comme un véritable membre de la cour impériale, prêt à mettre son statut épiscopal de côté pour remplir sa mission. C'est notamment le cas de son passage où il accepte de rencontrer Maxime dans le Consistoire, lieu où il refuse pourtant d'aller en 386 alors qu'il est convoqué par Valentinien tant il représente le pouvoir impérial, signe d'un transfert complet de son autorité vers un rôle d'ambassadeur :

\bigskip

\begin{quote} « Je dis que cette démarche était incompatible avec la fonction que j'occupais, mais que je n'hésiterais pas à remplir le devoir que j'avais entrepris, et que surtout, dans le service de Votre Majesté, et puisque c'était réellement pour soutenir votre affection fraternelle, je me réjouissais de m'humilier.\footnote{«~\latin{Dicens incongruum id esse muneri, quod gerebam; sed tamen non refugere me officium, quod suscepissem: et praesertim in servitio Vestrae Pietatis, et cum vere fraterno foveretis affectu, gratulari me, quod pro vobis humiliarer.}~». \cite[\nopp 30, 2]{ambroise_lettres}.}~» \end{quote}

\bigskip

Ambroise se sert de la lettre comme d'un outil pour expliciter et légitimer ses actions en tant qu'ambassadeur, et ainsi s'assurer que son autorité d'évêque et de conseiller impérial ne soit pas atteinte par le résultat de sa mission diplomatique. Cette précision de son dévouement pour l'empereur légitime lui permet de s'assurer d'avoir le soutien de Valentinien à son retour, tout en justifiant ses prises de positions virulentes envers Maxime, qu'il retransmet également dans la lettre. En effet, son rôle diplomatique ne semble pas l'empêcher d'utiliser l'ensemble de son \latin{auctoritas} pour s'adresser à Maxime et obtenir ce qu'il souhaite, usant à la fois de sa \latin{libertas dicendi} et de la \latin{karteria}, la position du philosophe s'opposant au pouvoir. Cette posture qu'Ambroise construit tout au long de son épiscopat l'amène à un discours parfois brutal envers son interlocuteur :

\bigskip

\begin{quote} « J'ai fait périr mon ennemi, dit-il [Maxime]. Ce n'était pas lui qui était votre ennemi ; c'est vous qui étiez le sien. [...] Si je ne me trompe, c'est l'usurpateur qui commence la guerre ; l'empereur ne fait que défendre son droit.\footnote{«~\latin{Hostem, inquit, occidi. Non ille tuus hostis, sed tu illius. [...] Nisi fallor, usurpatorem bellum inferre, imperatorem ius suum tueri.}~». \cite[\nopp 30, 10]{ambroise_lettres}.}~» \end{quote}

\bigskip

Évidemment, comme il s'agit d'un récit fait par la main même d'Ambroise, il ne faut pas mettre de côté la possible exagération de sa propre rhétorique. Mais le simple fait qu'Ambroise veuille se montrer comme un personnage sévère envers ses ennemis est en soi une démonstration d'un aboutissement politique de son \latin{auctoritas}. L'utilisation du terme « usurpateur » va dans ce sens d'un évêque ayant des positions tranchées, capable de rassembler ses qualités épiscopales et politiques pour réaliser ce qui lui semble juste pour l'Empire, comme ici entrer dans un esprit de contestation vis-à-vis de Maxime. Paradoxalement, le lancement d'une campagne militaire par Maxime et la prise de l'Italie en 387 est le signe de l'influence d'Ambroise dans la diplomatie impériale\footnote{Valentinien choisit de se réfugier auprès de Théodose, qui hésite à l'idée de se lancer dans une guerre civile. Finalement, à l'été 388 les deux empereurs lancent une campagne contre Maxime qui est battu puis exécuté, ce qui permet à Théodose de mettre la main sur l'Italie, repoussant le jeune Valentinien dans les provinces les plus à l'ouest.}. Alors que la première ambassade trouve sa réussite dans l'instauration d'un \latin{statu quo} et d'une reconnaissance officieuse de l'autorité de Maxime, l'échec de la deuxième mission et la tension instaurée aboutissent à un conflit armé entre les empereurs. L'\latin{auctoritas} d'Ambroise ne se résume donc pas seulement à sa popularité au cœur de Milan, ni à sa capacité à influencer la foi des souverains, mais aussi à impacter l'ensemble de la diplomatie impériale, faisant de ses réussites et échecs des tournants politiques impossibles à éclipser.

\bigskip

\subsubsection{La posture du prophète}

À l'opposé du rôle d'ambassadeur, qui s'inscrit dans une logique de service diplomatique pour l'Empire, et pourtant émanant tout aussi directement de l'\latin{auctoritas} personnelle, Ambroise développe et entretient longuement une autre facette de son rôle d'évêque : le positionnement en tant que nouveau prophète aux côtés des détenteurs du pouvoir. L'idée derrière cette posture est de trouver le meilleur moyen de modeler la conscience de l'empereur et de changer à sa guise la direction des décisions impériales lorsque celles-ci heurtent la morale chrétienne. Palanque s'y attarde avec précision, en montrant que le développement d'un avis politique d'Ambroise n'est que l'aboutissement de son autorité sous toutes ses formes : l'obligation religieuse liée à son rôle d'évêque, le développement de son statut de conseiller impérial et bien évidemment la mise en valeur et l'éloge du franc-parler. Autant d'outils utilisés par Nathan pour s'écarter du risque d'un prince mal éclairé pouvant nuire à la politique impériale\autocite[208-210]{palanque_ambroise}.

\bigskip

Cette position de prophète ne provient pas de nulle part, il l'institutionnalise et la légitime en s'emparant d'une typologie biblique bien connue et respectée : celle du roi David et du prophète Nathan. Lorsqu'Ambroise intervient pour reprendre moralement Théodose en 388 puis en 390, il cite immédiatement dans ses remontrances et demandes la figure de Nathan, qui n'est pas étrangère à l'éducation religieuse de l'empereur. Par ce simple fait d'invoquer l'autorité prophétique de Nathan qui impose sa vision sur David, au sein d'un événement contemporain similaire, Ambroise s'approprie la place de Nathan comme un intermédiaire indispensable entre Dieu et le souverain, y compris pour juger des actions non religieuses mais où la conduite morale est attaquée. L'\latin{auctoritas} personnelle de l'évêque prend donc une place importante dans la conduite du Salut du prince, et si ce chemin interfère avec les actions politiques, alors Ambroise n'hésite pas à imposer sa volonté.

\bigskip

Comme évoqué, c'est donc déjà le cas dès l'affaire de Callinicum, où il rappelle à Théodose que ses succès militaires, notamment sa victoire contre l'usurpateur Maxime, n'ont été permis que par l'aide de Dieu. L'autorité d'Ambroise vise donc également à s'assurer que la politique impériale reste dans une reconnaissance envers la foi chrétienne, qui s'explique par les paroles de Nathan :

\bigskip

\begin{quote}
    « Et que te dira le Christ après cela ? Ne te souviens tu pas de ce qu'il a fait dire au saint David par le prophète Nathan ?\footnote{«~\latin{Et quid te cum posthac Christus loquetur ? Non recordaris quid David sancto per Nathan prophetam dauerit ?}~». \cite[\nopp 40, 22]{ambroise_lettres}.}~»
\end{quote}

\bigskip

Mais plutôt que simplement notifier le rôle divin dans la réussite de Théodose, ce passage sert à rappeler que les souverains des Écritures ont toujours eu des figures religieuses à leur côté, comme Nathan devant David\footnote{2 Samuel 7.}, ou Élie face à Achab\footnote{1 Rois 21.}. Ainsi, la parole d'Ambroise n'est pas soumise aux conflits contemporains puisqu'elle n'est que la retransmission de la parole divine. Et cette parole a également comme objectif d'éviter au souverain de s'enfermer dans le péché qui apparaît comme inévitable par sa fonction dans l'Empire. En effet, la morale chrétienne prônée par les écrits d'Ambroise s'oppose à l'arbitraire monarchique. Il est quasiment impossible de suivre parfaitement le chemin de la foi chrétienne en résolvant les problèmes politiques et sociétaux de l'Empire. Et c'est notamment le cas chez Ambroise, du moins c'est ce que semble dire l'évêque dans sa lettre \latin{extra collectionem} 11 :

\bigskip

\begin{quote}
    «~Daignez me permettre, empereur gracieux. Vous avez un zèle pour la foi, je l’avoue, vous avez la crainte de Dieu, je le confesse : mais vous avez une véhémence de tempérament, qui, si elle est apaisée, peut aisément se changer en compassion, mais si elle est enflammée, devient si violente qu’il vous est à peine possible de la maîtriser.\footnote{«~\latin{Permitte, quaeso, imperator auguste. Habes zelum fidei, fateor, habes timorem dei, confiteor: sed habes naturae impetum, quem si quis leniat, cito uertes ad misericordiam; si quis stimulet, in maius suscitatur ut eum reuocare uix possis.}~». \cite[\nopp ec 11, 4]{ambroise_csel_82_3}. Traduction personnelle.}~»
\end{quote}

\bigskip

L'évêque de Milan se permet ici de tenir des propos audacieux à l'encontre de Théodose, signe de la confiance de sa posture prophétique, soutenue entre autres par l'image de Nathan. Ce passage sur la possible violence de l'empereur est à relier à certains extraits de l'\latin{Apologie de David} qu'Ambroise rédige peu de temps avant et qui s'attarde sur les risques plus grands qu'ont les détenteurs du pouvoir de fauter : «~Il a péché : c'est la marque de sa condition ; il s'est prosterné : c'est la marque de son amendement. Sa faute, c'est le lot commun ; mais sa confession, c'est son mérite distinctif. Ainsi être tombé dans le péché, c'est le propre de la nature, mais avoir lavé sa faute, c'est le propre de la vertu.\footnote{«~\latin{Peccavit, quod solet regiae esse potestatis; lapsus est, sed poenitentiam gessit, flevit, ingemuit. [...] Ergo quod lapsus est, naturae fuit; quod emendavit, virtutis.}~». \cite[\nopp IV, 15]{ambroise_apologie_david}.}~» Ambroise possède bien une certaine cohérence politique entre ses écrits et réflexions, ne délaissant pas ses idées dogmatiques lorsqu'il s'agit de confronter le pouvoir impérial. Pour Ambroise, il ne s'agit pas d'inculper Théodose de ses crimes, mais bien de lui rappeler la difficulté de son rôle, y compris par des comparaisons avec les récits du Livre de Samuel et des Rois, afin de le contenir dans la morale chrétienne et de le pousser à se repentir pour faciliter ses objectifs politiques. La vision d'un Ambroise prophète a donc pour objectif de faire rentrer l'idéal chrétien dans le quotidien politique du pouvoir.

\bigskip

Ainsi, l'identification de Nathan comme source d'autorité et d'influence sur le pouvoir permet à Ambroise de théoriser une nouvelle forme de souveraineté chrétienne : pour être un empereur victorieux, il faut être un empereur pénitent. Ambroise n'use pas de son autorité pour discréditer l'homme, mais pour sauver le prince. C'est en partie ce que nous avons déjà évoqué au sujet de la \latin{libertas dicendi} : Ambroise se veut proche mais ne cherche pas à humilier l'empereur. Son rôle de « nouveau Nathan » se fait dans un cadre intime et non pas comme un orateur public cherchant le scandale. Il privilégie, du moins dans un premier temps, la voie privée, celle de la direction de conscience :

\bigskip

\begin{quote}
    «~J’ai préféré recommander secrètement cette véhémence à votre réflexion, plutôt que de courir le risque de l’enflammer publiquement par mes actes.\footnote{«~\latin{Hunc ego impetum malui occulte tuae committere considerationi quam meis factis publice fortasse mouere.}~». \cite[\nopp ec 11, 5]{ambroise_csel_82_3}. Traduction personnelle.}~»
\end{quote}

\bigskip

Ambroise apparaît de nouveau conscient de son influence et de l'importance d'amener une remontrance privée, même s'il est possible de voir cette citation comme un avertissement envers l'empereur du fait que l'évêque est toujours en mesure d'utiliser son autorité contre la notoriété de l'empereur. C'est dans cette lettre, écrite à la suite du massacre de Thessalonique sur lequel je reviens en détail dans la fin de ce chapitre, qu'Ambroise relie définitivement son rôle auprès de Théodose à celui de Nathan. Il ne laisse plus de doute quant à sa volonté d'imposer cette grille de lecture à Théodose, utilisant les mots de l'Écriture pour convaincre la pénitence impériale :

\bigskip

\begin{quote}
    «~Que Votre Majesté ne se montre donc pas impatiente d’entendre ce qu’on lui dit, comme le prophète le dit à David, car si vous l’écoutez obéissamment et dites : « J’ai péché contre le Seigneur », si vous employez ces mots du roi Prophète, « Venez, adorons et prosternons-nous, agenouillons-nous devant le Seigneur notre Créateur », à vous aussi il sera dit : «~Parce que tu te repens, le Seigneur a mis de côté ton péché, tu ne mourras pas.~»\footnote{«~\latin{Non ergo impatiens sit imperator, si dicatur ei quod dixit propheta Dauid regi: 'in me peccatum est'; si audiat digne et dicat: 'peccaui domino'; si dicat regium illud propheticum: 'uenite adoremus et procidamus ante dominum et ploremus ante dominum qui fecit nos', dicatur et tibi: 'quoniam paenitet te, dimittit tibi dominus peccatum tuum et non morieris'.}~». \cite[\nopp ec 11, 7]{ambroise_csel_82_3}. Traduction personnelle.}~»
\end{quote}

\bigskip

La précision « à vous aussi » est la plus importante : elle relie aussi bien Théodose à David qu'Ambroise à Nathan et permet à l'évêque de citer directement les paroles de l'Ancien Testament comme un signe de similitude des événements et donc de résolution du problème. En acceptant cette réprimande, Théodose accepte définitivement l'\latin{auctoritas} d'Ambroise. Que ce soit pour Callinicum ou pour Thessalonique, il accepte son autorité dans l'idée d'éviter de se mettre à dos une population milanaise acquise à la cause de son évêque. C'est le signe ultime de l'autorité politique d'Ambroise qui est bel et bien renforcée par sa position de prophète aux côtés d'un souverain chrétien.

\bigskip

\subsubsection{Le devoir d'exemplarité morale : l'affaire de Thessalonique}

La théorie du prophète et du roi pénitent que nous venons d'exposer trouve son application concrète et éclatante lors de l'affaire de Thessalonique en 390. C'est notamment cet événement qui permet à Ambroise de transformer son \latin{auctoritas} spirituelle en une véritable contrainte politique, obligeant l'empereur à suivre son autorité, non seulement pour sa foi personnelle, mais également par nécessité d'État.

\bigskip

Le massacre de Thessalonique est un ordre impérial déclenché par Théodose au printemps 390. Cet ordre fait suite au soulèvement de la population contre Buthéric, chef des armées locales ayant fait emprisonner un cocher populaire de la ville, qui est tué. La réaction de l'empereur est immédiate et violente : il prend la décision de rassembler les citoyens dans le cirque de la ville et déclenche un massacre faisant près de sept mille victimes. Cet ordre prend place alors que les relations entre l'empereur et l'évêque de Milan sont difficiles, c'est une période que Jean-Rémy Palanque nomme la «~froideur de 389\autocite[223]{palanque_ambroise}.~». Après le conflit lié à l'affaire de Callinicum, les défiances sont nombreuses et les membres du Consistoire ont l'interdiction de parler à Ambroise pour éviter qu'il s'immisce dans les affaires politiques. Ambroise prend connaissance du massacre tardivement et décide d'intervenir, mais en évitant la confrontation directe. Il fait alors le choix de lui adresser une lettre privée qui cherche à pousser Théodose à la pénitence en liant le Salut personnel de l'empereur à la réussite de son pouvoir et donc de l'Empire. Cette pénitence ne vient évidemment pas de nulle part, elle fait écho aux péchés des rois David et Salomon et de leur longue recherche du pardon à travers la pénitence. Ambroise écrit ainsi à Théodose :

\bigskip

\begin{quote}
    «~Ce que j’écris, ce n’est pas pour vous confondre, mais pour que ces exemples royaux vous incitent à écarter ce péché de votre royaume ; ce que vous ferez en vous humiliant devant Dieu. Vous êtes un homme ; la tentation est tombée sur vous ; vainquez-la. Le péché ne se lave que par les larmes et la pénitence.\footnote{« \latin{Non ut confundam te, haec scribo; sed ut regum exempla provocent te ad tollendum hoc peccatum de regno tuo; tolles autem humiliando te Deo. Homo es, tentatio te apprehendit; vince eam. Peccatum non tollitur, nisi lacrymis et poenitentia.}~». \cite[\nopp ec 11, 11]{ambroise_csel_82_3}. Traduction personnelle.}~»
\end{quote}

\bigskip

Pour convaincre Théodose que cette pénitence n'est pas une faiblesse politique mais au contraire un signe de réussite future pour l'Empire, Ambroise joint à sa lettre son \latin{Apologie de David}. L'objectif est double : valoriser son propos par l'exemple d'une figure biblique majeure et rassurer Théodose sur les risques de faire pénitence alors qu'il détient le pouvoir. Ce texte est très certainement dans un premier temps un sermon lu devant la population de Milan en automne 388, avant qu'il ne soit modifié pour se rattacher à la défaite de Maxime et au massacre de Thessalonique. L'\latin{Apologie de David} permet ainsi à Ambroise d'illustrer ses propos sur le «~nouveau David~» et de fournir une sorte de manuel de l'exemplarité royale dans la foi. Plus que seulement le Salut du souverain, Ambroise essaie de pousser Théodose à la pénitence en le mettant face au risque d'une répercussion de ses péchés à l'échelle de l'Empire. Cette fameuse exemplarité morale chez l'empereur est donc dans le champ d'action de l'\latin{auctoritas} de l'évêque puisqu'elle impacte l'ensemble de la politique impériale :

\bigskip

\begin{quote}
    «~Il ne pouvait nier son péché, mais en tant que coupable, il l’avouait avec douleur, sachant qu’il était tenu par les liens d'autant plus étroits que plus grandes étaient ses obligations : on exige en effet davantage de celui à qui l’on a confié davantage.\footnote{« \latin{Peccatum suum negare non poterat: sed quasi reus cum dolore fatebatur; eo magis se obligatum sciens, quo majora debet. Cui enim plus committitur, plus ab eo exigitur.}~». \cite[\nopp 51]{ambroise_apologie_david}.}.~»
\end{quote}

\bigskip

Le message d'Ambroise est donc que son intervention est légitimée par le fait que la pénitence n'est pas une option privée, mais un acte nécessaire au Salut de l'Empire. C'est ainsi que l'évêque de Milan réussit à justifier son implication dans une affaire qui ne semble pourtant pas se référer au champ d'action de l'\latin{auctoritas} épiscopale que nous avons décrit tout au long de ce chapitre.

\bigskip

La victoire de l'influence ambrosienne se fait ressentir à travers deux aspects. Premièrement, Théodose finit par accepter de se plier aux conseils d'Ambroise et fait pénitence en se rendant à la basilique de Milan sans les insignes impériaux, à égalité devant Dieu. L'autorité religieuse de l'évêque atteint bien définitivement la sphère politique en réussissant à interférer avec les actions impériales, signe d'une victoire de l'influence d'Ambroise. Il est tout de même important de noter que cette pénitence n'est pas un signe de soumission de Théodose à l'Église chrétienne mais plutôt une action lui permettant de redorer son image aux yeux de tous en se montrant comme un empereur pieux. Deuxièmement, l'empereur décide dans le même temps d'introduire une constitution sur les condamnations à mort, obligeant un délai de trente jours entre la condamnation et son exécution. Bien que cette loi émane de Théodose et de ses conseillers, il ne fait aucun doute qu'elle fait écho à la pression morale exercée par l'ensemble de l'\latin{auctoritas} d'Ambroise :

\bigskip

\begin{quote}
    « Si, par un mouvement de notre puissance royale, nous ordonnons de sévir contre certains plus sévèrement que de coutume, nous ne voulons pas qu'ils subissent la peine ou reçoivent la sentence immédiatement, mais que leur sort et leur fortune soient suspendus pendant trente jours.\footnote{« \latin{Si vindicari in aliquos severius contra nostram consuetudinem pro motu regiae potestatis iusserimus, nolumus statim eos aut subire poenam aut excipere sententiam, sed per dies triginta super statu eorum sors et fortuna suspensa sit.}~». \cite[IX, 40, 13]{code_theodosien}. Bien que les premières sources datent cette constitution de 382, il s'agit très certainement d'une erreur de copiste car la constitution est clairement issue des conséquences du massacre de Thessalonique. La critique historique, à la suite de Mommsen et Palanque, s'accorde à la restituer à 390.}~»
\end{quote}

\bigskip

La mise en place de la constitution ainsi que la pénitence de Théodose sont les preuves que l'action d'Ambroise dépasse le cadre épiscopal pour finalement affirmer la direction de conscience et son impact sur la sphère impériale. L'\latin{auctoritas} d'Ambroise trouve son aboutissement dans l'encadrement moral où ses théories politiques et religieuses sur l'idéal de la royauté chrétienne rejoignent des réalités concrètes mêlant évêques et empereurs.

\subsection*{Conclusion}

L'étude autour de la construction d'une \latin{auctoritas} personnelle chez Ambroise nous permet de comprendre que l'évêque de Milan conçoit son rôle dans l'Empire comme celui d'un conseiller diplomatique tout autant que comme un régulateur moral indispensable au souverain et donc à l'Empire. Après avoir montré comment l'Église, sous l'impulsion de la réflexion ambrosienne, s'est positionnée en institution indépendante, la seconde partie du chapitre s'est concentrée sur la mise en lumière des méthodes permettant à Ambroise d'agir au plus près du pouvoir et d'accroître son influence sur les décisions impériales.

\bigskip

Ambroise se révèle capable d'user de l'ensemble des outils à sa disposition pour construire à sa manière son influence. L'objectif de l'évêque semble toujours osciller entre son ambition personnelle de se placer comme une figure importante du pouvoir impérial et sa volonté d'agir pour le Salut de l'empereur et donc de l'ensemble de l'Empire. L'aspect le plus marquant de cette autorité reste sans aucun doute la critique morale et politique qu'il n'hésite pas à faire à l'égard des souverains lorsqu'ils s'écartent de sa vision des événements. C'est le cas de la querelle religieuse en 386 contre le soutien qu'apporte Valentinien II aux ariens, mais également lors des moments où Ambroise juge que l'empereur s'écarte de la foi chrétienne et met à mal la paix et la justice dans l'Empire. Critique notamment possible grâce à sa création d'un lien intime avec chacun des empereurs lui donnant accès à une véritable liberté de parole, plus grande encore que la \latin{libertas dicendi} qu'il réclame pour l'ensemble du monde épiscopal. Son autorité est donc régulièrement déployée sur le terrain diplomatique, à travers ses missions d'ambassadeur à Trèves ou son rappel à l'ordre contre Théodose après la punition proclamée contre l'évêque de Callinicum.

\bigskip

L'affaire de Thessalonique marque l'aboutissement de cette réussite ambrosienne alors qu'il parvient à faire de son \latin{auctoritas} un outil de l'idéal du pouvoir chrétien qu'il forge dans ses écrits. L'utilisation de la typologie biblique de David et Nathan pour s'adresser à Théodose lui permet de se placer non pas comme un moralisateur et opposant, mais bien comme un proche conseiller agissant pour son bien et celui de l'Empire, n'hésitant pas à relier le Salut de l'âme du prince à la prospérité du territoire romain.
