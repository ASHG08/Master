\section{Construction et utilisation d'une \latin{auctoritas} personnelle}

\subsection*{Introduction}

(il convient désormais d'analyser comment cette autorité s'incarne concrètement dans la figure de l'évêque face au souverain.)
Quelle posture AMbroise utilise pour mettre en place ces idées et dépasser les limites de son rôle

\subsection{L'instauration du rapport de force : entre obligation morale et liens personnels}

\subsubsection{La « \latin{libertas dicendi} » : entretenir le dialogue avec les empereurs}

L'étude ambrosienne sur l'\latin{auctoritas} nous a permis de comprendre les fondements institutionnels de l'autorité épiscopale ainsi que la mise en place d'une nouvelle relation entre le pouvoir impérial et l'Église. Mais l'ensemble de ce cadre politique ne peut s'exercer qu'à travers un dialogue appronfondi entre les évêques et l'empereur. Le coeur de cette dynamique se trouve dans ce qu'Ambroise nomme la « \latin{libertas dicendi} », ce droit à la parole ancré dans la tradition romaine et réutilisé par Ambroise comme un principe chrétien. Pour l'évêque de Milan, cette liberté de parole n'est pas un simple privilège que l'Église revendique mais bien une obligation qui permet de mettre clairement l'évêque dans une position de conseiller légitime face au pouvoir. Cet aspect se place dans la continuité du développement d'une autonomie cléricale dans laquelle Ambroise théorise les droits et devoirs des évêques pour définir leur rôle et leur fournir un poid important dans la balance politique.

\bigskip

L'objectif principal de cette réflexion politique pour l'évêque de Milan est d'augmenter l'autorité et l'influence de l'Église, et par la même occasion la sienne pour se retrouver avec les mains de plus en plus libre dans l'Empire. Sa lutte doit alors se concentrer sur un point précis : éviter la censure que peuvent imposer les empereurs aux différents évêques. C'est pour cette raison qu'il est question d'obligation dans la liberté de parole : l'intervention de l'évêque, comme conseiller impérial sur les questions religieuses, se développe comme un point à part entière de son rôle dans l'Empire. L'ensemble de cette théorie prend évidemment naissance dans l'action d'Ambroise, notamment lors de son intervention dans l'affaire de Callinicon, risquée puisque allant à l'encontre du choix de l'empereur. Deux passages de sa lettre à Théodose nous permettent d'etablir avec précision sa pensée sur le sujet du dialogue entre évêques et empereurs. Tout d'abord avec cette idée qui semble relier la parole des évêques à la légitimité de l'empereur d'exercer son pouvoir :

\bigskip

\begin{quote}
    « Mais il ne convient ni à un empereur de refuser la liberté de parler ni à un évêque de ne pas dire ce qu’il pense. [...] Chez un évêque aussi, rien n’est plus dangereux devant Dieu, ni plus honteux devant les hommes, que de ne pas déclarer librement ce qu’il pense.\footnote{« \latin{Sed neque imperiale est libertatem dicendi negare neque sacerdotale quod sentiat non dicere. [...] Nihil etiam in sacerdote tam periculosum apud Deum, tam turpe apud homines quam quod sentiat non libere denuntiare.} » \cite[\nopp 74,2]{ambroise_lettres}.}~»
\end{quote}

\bigskip

Et quelques lignes plus loin l'explication de ce à quoi doit aboutir la liberté de parole :

\bigskip

\begin{quote}
    « Je préfère donc, empereur, partager avec toi le bien plutôt que le mal, c’est pourquoi le silence de l’évêque doit déplaire à ta Clémence, mais sa franchise lui plaire. [...] Je ne me mêle pas en importun de ce qui ne me regarde pas, je ne m’ingère pas dans les affaires d’autrui, mais j’obéis à mon devoir, je me soumets aux commandements de notre Dieu.\footnote{« \latin{Malo igitur, imperator, bonorum mihi esse te cum quam malorum consortium et ideo clementiae tuae displicere debet sacerdotis silentium libertas placere. [...] Non ergo importunus indebitis me intersero, alienis ingero, sed debitis obtempero, mandatis Dei nostri oboedio.} » \cite[\nopp 74,3]{ambroise_lettres}.}~»
\end{quote}

\bigskip

Plusieurs points majeurs de la pensée ambrosienne sont à relever dans ces citations. Les écrits de l'évêque de Milan oscillent en permanence entre réflexion politique large et rhétorique utile dans un cadre bien précis, les deux se mélangeant souvent dans ses correspondances. Une chose est certaine, il ancre en permanence ses besoins utilitaires dans des théories politiques précises qu'il cherche à appliquer à grande échelle. Ainsi dans cette lettre ce n'est pas seulement avec sa personne, pourtant proche des empereurs et évêque particulièrement important, qu'il essaie d'introduire l'idée de la contestation épiscopale, mais bien avec l'ensemble des ministres de la foi. Il légitime par ce biais l'ensemble de l'\latin{auctoritas} de l'Église et en appel à ce droit de parole. Dieu est invoqué comme source d'autorité incontestable : si il est permis par Dieu de parler librement, alors l'empereur n'a pas le pouvoir de s'y opposer. Mais surtout, Ambroise légitime sa prise de parole par des références constantes aux Écritures. Nous l'avons déjà évoqué, l'évêque de Milan utilise les textes de la Bible comme un traité politique qui reflète la société chrétienen romaine. Ainsi, dans les deux passages cités, il s'appuie sur des références scipturaires précises, telles que le Psaume 118 : « Devant les rois je parlerai de ton témoignage et je n'aurai nulle honte.\footnote{Psaume 118, 46.} » ou la deuxième lettre à Timothée : « Prêche la parole, insiste en toute occasion, favorable ou non, reprends, censure, exhorte, avec toute douceur et en instruisant.\footnote{2 Timothée 4, 2.}~» Le silence se présente donc comme une faute morale et un manquement aux exigences du rôle d'évêque.

\bigskip

Mais cette « \latin{libertas dicendi} ne doit pas pour autant être vu comme de la contestation politique ou de l'ingérence. Tout comme le développement de l'\latin{auctoritas} épiscopale, la parole et le conseil de l'évêque ne dépasse pas chez Ambroise la sphère d'action de l'Église, et donc les affaires religieuses. La mission principale d'un évêque, et c'est ce que nous allons étudier tout au long du reste de ce chapitre, reste d'empêcher autrui de péché, et à plus forte raison si cet évêque est proche de l'empereur. La deuxième citation de l'\latin{Epistolae} 74 permet à Ambroise de rassurer Théodose sur ses intentions. Il utilise cette fois la première personne pour montrer qu'il n'intervient que dans un cadre légitime, en phase avec ses droits. En réalité, le fait même qu'il soit obligé de théoriser son intervention et de la replacer dans une pensée politique plus large faisant appel à la laibrté de parole des évêques, montre bien qu'il n'est pas rassuré et donc qu'il se place aux limites de son autorité. Bien qu'il s'agisse d'une réflexion politique pleinement aboutie, elle est ici et comme souvent utilisée comme une arme d'action immédiate.

\bigskip

Justemment, pour sortir du cadre « utile » de la pensée ambrosienne, il est intéressant de piocher une citation issue de son commentaire sur certains Psaumes. Ambroise apparaît comme conscient de la nécessité de modérer cette liberté des évêques, afin qu'elle se fasse dans le cadre le plus utile et juste possible :

\bigskip

\begin{quote}
    « Tu vois donc qu'il ne faut pas que les prophètes de Dieu ou les prêtres fassent injure aux rois à la légère, s'il n'y a pas de fautes assez graves pour qu'ils doivent être repris. Mais là où les fautes sont plus graves, il ne semble pas que le prêtre doive les épargner, afin qu'ils soient corrigés par de justes réprimandes.\footnote{« \latin{Vides ergo, quia regibus non temere vel a prophetis Dei vel a sacerdotibus facienda iniuria sit, si nulla sint graviora peccata in quibus debeant argui. Ubi autem peccata graviora sunt, ibi non videtur a sacerdote parcendum, ut iustis increpationibus corrigantur.} » \cite[\latin{Enarratio in psalmum} 37,43. Page 172]{ambroise_csel_64}.}~»
\end{quote}

\bigskip

La figure de David, puisqu'Ambroise l'utilise comme auteur du Psaume 37, permet de parler aussi bien aux empereurs qu'aux évêques. Il tire ainsi des concepts s'adressant aux deux institutions et utilise son commentaire comme outil de réflexion politique qui doit s'appliquer dans l'Empire. D'une façon générale, ses homélies sur les Psaumes, « tout en étant attentive à marquer leur sens messianique, sont soucieuses d'interprétation morale liée concrètement à la situation eccleśiale et politique du moment.\autocite[223]{quasten_initiation}~» Par cette citation, il limite concrètement les raisons pouvant pousser aux interventions épiscopales, ce qui a pour effet de valoriser considérablement ces fameuses interventions qu'il qualifie lui même de « nécessaire réprimande ». Il est difficile de dater avec précision les homélies sur douze des Psaumes de David, mais Palanque semble indiquer que ce commentaire est sans doute prononcé autour de 389. Ce qui signifie qu'Ambroise écrit ce passage après l'affaire de Callinicon. Bien qu'il ne s'adresse pas aux empereurs, il accentue donc la théorie politique de liberté de parole en placant l'intervention d'Ambroise auprès de Théodose comme une correction liée à une « faute grave ».

\bigskip

La revendication d'une \latin{libertas dicendi} marque un point central dans la pensée politique ambrosienne. Dans son court passage sur l'évêque de Milan, Giuseppe Zecchini analyse sa pensée sur le pouvoir impérial et se penche notamment sur le risque de voir le souverain devenir un tyran. Face à ce risque, la \latin{libertas dicendi} intervient comme la « première forme et expression de toute liberté politique\autocite[165]{zecchini_pensiero_politico}~». Avec Ambroise, la parole de l'évêque se place comme un outil de contrôle du pouvoir impérial. Zecchini note que, tout comme l'\latin{auctoritas}, cette fonction semble passer des mains des sénateurs aux « \latin{sacerdoti di Cristo} », les prêtres du Christ. Le Sénat, affaibli par la toute puissance de l'empereur, ne s'affiche plus comme le garant de la liberté face au prince. L'évêque, par cette fondamentale liberté de parole utilisée par Ambroise, obtient donc ce devoir religieux d'imposer une limitation morale aux volontés impériales.

\bigskip

Ainsi, la position de l'évêque auprès de l'empereur se définit entre autre par son devoir d'interpellation qui ne peut être supprimé. Le dialogue se doit donc d'être entretenu pour respecter le droit à la liberté de parole, qui fonde en partie le statut politique de l'Église. Le pouvoir romain chrétien est même légitimé par ce droit que possède les évêques à exprimer de « justes remontrance », qui permet au souverain de rester dans le chemin de la foi et d'éviter les péchés qui guettent le pouvoir.

\bigskip

\subsubsection{La proximité d'Ambroise avec les princes chrétiens}

Si la \latin{libertas dicendi} offre aux évêques le droit théorique de s'exprimer, tous n'ont pas pour autant la même autorité ou influence face au pouvoir impérial. Bien qu'ils puissent être reçus et entendus, seule la qualité de la relation tissée avec le souverain garantit une véritable écoute et prise en compte de cette parole. Ambroise est évidemment conscient des limites de l'\latin{auctoritas} institutionnelle de l'Église face à la réalité du pouvoir et s'attache fortement à développer un lien personnel important, dépassant le statut politique, avec les empereurs pour conforter son autorité au sein de la cour. L'idée est simple et résume parfaitement la vision qu'a Ambroise de s'impliquer dans la vie politique : si chacune de ses actions est encadrée par un cadre théorique poussé, il n'en oublie jamais les réalités immuables de la société, transformant alors son poste d'évêque étranger à la cour en un personnage intimement lié aux souverains qui ne peut s'en passer. Cette proximité humaine avec les détenteurs du pouvoir ne signifie pas pour autant un asservissement total envers eux. Au contraire, Ambroise a confiance en son rôle et en ses capacités et use de ses relations pour exprimer pleinement et librement ses pensées.

\bigskip

Cette dynamique s'observe dès le début de son épiscopat, à travers ses premiers rapports avec Gratien. La relation entre les deux débute par une demande de l'empereur qui souhaite avoir un exposé sur la foi trinitaire afin de mieux comprendre les tensions existantes entre les courants chrétiens. Et déjà, les réponses de l'évêque se font attendre, la correspondance prend du temps et la rédaction du \latin{De Fide} n'est pas précipitée par Ambroise qui ne semble pas réagir à la pression d'une demande impériale. Comme le note Yves-Marie Duval, «~Ambroise n’est pas le courtisan zélé qui se hâte de satisfaire la demande d’un puissant ou qui l’encombre de ses productions.\autocite[223]{duval_lettres_gratien}~» Ce refus de la course à la production le place dans une position de force : c'est bien l'empereur qui doit relancer l'évêque dans sa requête d'apprentissage de la foi. L'état de cette relation est un signe de confiance absolue de Gratien envers Ambroise dans sa capacité à répondre aux problèmes religieux puisque l'empereur ne cherche pas de réponse auprès d'autres évêques, s'appuyant sur l'autorité certaine d'Ambroise en Italie. Avec Gratien, Ambroise nourrit une relation de respect et de dépendance intellectuelle, valorisant son statut à la cour impériale.

\bigskip

Avec Valentinien II, empereur beaucoup plus jeune, Ambroise ajoute à sa posture d'autorité religieuse celle du père et maître à penser pour réussir à forger une influence durable. En revanche, cette relation met du temps à se mettre en place à cause de l'influence de la mère du prince, Justine, fervente défenseuse de la foi arienne. Ambroise se retrouve à intervenir à plusieurs reprises pour défendre la foi chrétienne et même nicéenne, lors des affaires de l'autel de la Victoire en 384 ou des basiliques de Milan en 386. Et pourtant, après la mort du jeune empereur par suicide ou assassinat en 392, Ambroise insiste dans son \latin{De Obitu Valentiniani} sur la tristesse qui l'envahit, signe d'une proximité certaine bien que sûrement en partie exagérée. Cette proximité avec Valentinien est également exprimée dans la Lettre 25 à destination de Théodose, dans laquelle il revient sur la mort de l'empereur, pour rappeler son rôle auprès du pouvoir impérial en Occident :

\bigskip

\begin{quote} « Il professait qu'il devait son éducation à moi, il me désirait comme un père attentif, et lorsque certains prétendaient avoir reçu des nouvelles de mon arrivée, il les anticipait avec impatience. D'ailleurs, pendant ces jours mêmes de deuil public, bien qu'il eût sous la main des évêques saints et éminents dans les limites de la Gaule, il jugea néanmoins bon de m'écrire pour que je lui confère le Sacrement du Baptême. Par cette demande, sinon raisonnablement, du moins affectueusement, il témoignait de son amour envers moi.\footnote{« \latin{Ille se a me nutritum praeferebat, ille ut sedulum patrem desiderabat, ille simulato a quibusdam adventus mei nuntio inpatienter praestolabatur. Quin etiam illis ipsis publici doloris diebus, cum sanctos et summos sacerdotes domini intra Gallias haberet, ut a me tamen sacramentis baptismatis initiaretur, scribendum arbitratus est ; quod etsi non rationabiliter, amabiliter tamen erga me suum studium testificatus est.} » \cite[\nopp 25, 2.]{ambroise_csel_82_1}. {Traduction personnelle.}}~» \end{quote}

\bigskip

Le terme de « père attentif » est le plus intéressant, Ambroise n'est pas un simple évêque remplaçable au sein de la cour, mais bien celui qui offre aussi bien l'éducation que le Salut aux empereurs. Comme toujours, il est difficile de savoir la part d'exagération dans les propos d'Ambroise, surtout que nous ne possédons pas la demande de baptême par Valentinien, mais une intimité, bien que régulièrement tendue, existe, ou du moins est recherchée par l'évêque. En effet, plus que de savoir la réalité de la relation qu'il entretient avec l'empereur, il est intéressant de comprendre qu'Ambroise insiste sur le fait qu'il a toujours été proche, afin de développer une autorité auprès de tous qui dépasse le cadre institutionnel. Et c'est cette position qui lui permet d'agir aussi fermement dans l'affaire de l'autel de la Victoire. Il cherche à se différencier de son opposant Symmaque en adoptant un ton personnel dans ses lettres, comme pour démontrer une forme d'intimité dans la relation, que les arguments politico-religieux du sénateur ne peuvent dépasser. Dans son article sur la gestion des conflits par Ambroise, Gernot Michael Muller se permet même d'aller plus loin dans l'importance de la démonstration de l'intimité chez Ambroise : « C’est donc moins l’exercice réel du pouvoir sur son destinataire qui importait à Ambroise, que le fait de marquer implicitement la position de force qui était la sienne et qui dérivait d’une place privilégiée par rapport à ce même destinataire, à savoir celle du directeur de conscience.\autocite{muller_conflits}~» L'idée d'un « directeur de conscience » regroupe parfaitement les deux aspects de l'éducation et de la foi, et c'est ce qu'il vise avec chacun des empereurs qu'il côtoie. La relation personnelle apparaît ainsi comme un levier d'action politique, qu'Ambroise utilise en parallèle de ses rôles religieux et politiques intrinsèques à sa position d'évêque de Milan.

\bigskip

En revanche, avec Théodose, construire une relation personnelle forte se révèle être un défi plus important. Général victorieux, empereur en Orient, régulièrement en concurrence avec les pouvoirs impériaux de Gratien ou Valentinien II, Théodose ne propose pas le même profil capable d'écouter et de suivre facilement les conseils de l'évêque de Milan, et pourtant Ambroise se permet tout autant d'intervenir et de parler au nom du Salut de l'empereur. L'idée cette fois n'est plus de se mettre dans une position de père ou d'ami, mais d'égal et de conseiller, capable d'apporter un jugement moral lucide et juste. La seule position d'évêque ne lui suffit plus : si en 388, après sa lettre pour réprimander la décision de Théodose au sujet de la synagogue de Callinicon, il réussit à obtenir de Théodose le pardon à un évêque, il connaît un échec virulent en début d'année 390 alors que l'empereur refuse de le consulter suite au massacre de Thessalonique\footnote{Voir plus de détail sur Ambroise et l'affaire de Thessalonique dans le 2.2.3.}. Ambroise ne peut donc plus simplement se reposer sur l'autorité innée de son rôle d'évêque. C'est ce qu'analyse avec justesse Peter Brown à partir de la page 152 de son livre \latin{Pouvoir et persuasion dans l'Antiquité tardive} en montrant qu'Ambroise doit relancer sa relation avec Théodose avec « le courage d'un philosophe. » Brown parle ici des deux approches ambrosiennes envers le pouvoir, fondé sur l'image de la Grèce antique du philosophe affrontant l'autorité politique : « Ambroise se présentait comme l’exemple chrétien de l’ancienne \latin{karteria}, l’obstination inspirée avec laquelle les philosophes affrontaient le puissant. [...] Avec Théodose, le temps était venu de la \latin{parrhésia}, du franc-parler.\autocite[155]{brown_power_1992}~» Ambroise se montre comme le conseiller éclairant l'empereur dans le chemin de la foi.

\bigskip

Évidemment, même avec Théodose, l'évêque de Milan cherche à jouer de sa proximité avec le pouvoir impérial pour mieux se faire entendre, comme le montrent les premiers mots de sa lettre pour demander à l'empereur de se repentir : « Le souvenir de notre ancienne amitié m'est doux.\footnote{« \latin{Et veteris amicitiae dulcis mihi recordatio est} » \cite[\nopp ec. 11, 1]{ambroise_csel_82_3}.}~» Il est même question ici d'un temps long, comme pour justifier le fait que ses propos ne sont pas là pour entraver la vie de l'empereur. Les différentes lettres aux empereurs nous permettent donc de saisir avec précision les relations qu'Ambroise a su tisser avec le pouvoir impérial tout au long de son épiscopat, qui se révèlent tout aussi importantes, si ce n'est plus, que l'aspect institutionnel de l'épiscopat, la foi et l'Église dans son ensemble.

\bigskip

Finalement, le meilleur exemple dans la vie d'Ambroise de l'importance de l'entretien des liens intimes avec les souverains de l'Empire se situe dans son échec auprès d'Honorius, et plus précisément de Stilicon. À la mort de Théodose en 395, la structure politique de l'Empire est modifiée. Alors qu'il s'était fait maître de tout le territoire romain suite à la mort de Valentinien II et surtout suite à la défaite d'Eugène en 394, Théodose laisse son pouvoir dans les mains de ses deux fils : Arcadius âgé de 18 ans qui s'empare de l'Orient, et Honorius, seulement âgé de 10 ans, se retrouve avec l'Occident. Honorius est d'ailleurs présent à Milan lors de l'enterrement de Théodose, point important puisqu'une partie de l'Oraison Funèbre d'Ambroise est dédiée au très jeune héritier. Pourtant, Ambroise ne parvient pas à reproduire le schéma de proximité paternelle qu'il avait employé avec Valentinien II. L'évêque doit à ce moment faire face à un nouvel adversaire politique : Stilicon, général romain proche de Théodose qui affirme avoir reçu la régence de l'Empire. L'Orient échappe tout de même rapidement à son contrôle à cause de l'opposition d'Arcadius plus grand et donc apte à se présenter comme empereur, et du préfet du prétoire Rufin. En revanche, Honorius passe bien sous la tutelle de Stilicon, ce qui lui permet d'imposer son pouvoir dans les provinces d'Occident, reléguant le jeune prince à un rôle pratiquement protocolaire.

\bigskip

Alors que sur la fin du règne de Théodose l'influence ambrosienne semble au plus haut au sein du pouvoir impérial, son \latin{auctoritas} est contestée par Stilicon. En effet, celui-ci se légitime par son mariage avec la nièce de Théodose, son rôle de commandant militaire ainsi que le soi-disant legs de l'autorité impériale par l'empereur défunt. Stilicon ne semble pas avoir besoin d'une légitimité religieuse que peut apporter l'évêque de Milan, et choisit de l'écarter progressivement de son rôle politique en se montrant comme le véritable tuteur d'Honorius\autocite[à partir de la page 298]{palanque_ambroise}. Ambroise se voit donc forcé de revenir à un rôle purement clérical pour les derniers mois de sa vie. Le signe le plus flagrant de cette perte d'influence et de rôle politique est l'absence de lettre adressée à Stilicon ou Honorius pendant la durée de leur règne. Comme l'explique Neil McLynn, ce silence d'Ambroise, qui ne rejette ni n'adhère à la politique menée par le régent, ne doit pas être simplement perçu comme un manque de source, mais bien comme une preuve d'une fissure entre l'\latin{auctoritas} d'Ambroise et la \latin{potestas} de l'empereur\autocite[366]{mclynn_ambrose}.

\bigskip

En ayant conscience de cet aspect de la fin de vie d'Ambroise, il est intéressant de relire le \latin{De Obitu Theodosii}, non pas comme une simple louange à Théodose mais bien, comme le dit McLynn, comme une tentative politique de récupération par Ambroise de la tutelle d'Honorius et donc de se placer comme son mentor à la place de Stilicon\autocite[358-360]{mclynn_ambrose}. L'appui dans son discours d'une continuité entre Théodose et ses fils, aussi bien dans la politique impériale que dans la foi\footnote{\cite[\nopp 6-7]{ambroise_mort_theodose}.} a pour objectif, en plus d'établir une stabilité politique, de faire apparaître la tutelle d'Ambroise comme évidente. Puisque l'évêque était un proche conseiller de Théodose, il est normal qu'il le reste pour ses fils. Les volontés d'instructions et de préoccupations morales d'Ambroise envers les deux jeunes souverains sont donc certainement un simple outil politique pour garantir son rôle à la cour, ce qui s'avère être un échec. Stilicon s'empare pleinement du pouvoir en Occident et fait disparaître l'\latin{auctoritas} ambrosienne.

\bigskip

L'ensemble des relations d'Ambroise avec les empereurs de la fin du IVème siècle agit comme un révélateur des limites de l'\latin{auctoritas} épiscopale. Aussi importante soit-elle dans le discours d'Ambroise, le rôle politico-religieux de l'Église au sein de l'Empire reste principalement dépendant du bon vouloir du souverain, et donc de la réussite ou non dans la création d'un lien intime entre un évêque et son empereur. Cette compréhension par Ambroise des réalités politiques lui permet de mettre en place ses théories et réflexions et de perdurer, sous différentes postures, au coeur du jeu politique de l'Empire.

\bigskip

\subsubsection{La maîtrise des armes liturgiques}

Idée que quoi qu'il arrive, peu importe le conflit avec le pouvoir impérial, Ambroise trouvera une solution pour faire appliquer ses envies et ses idées. Il maitrise la rhétorique et les liens intimes, mais également en maitrise du peuple et de l'importance de la liturgie, que ce soit par les hymnes faisant du peuple devant la basilique un groupe unis, ou par la découevrte des martyrs, sans doute complètement inventée, permetant à Ambroise d'obtenir une validation divine contre les ariens et Justine/Valentinien

Découverte des corps des martyrs et déplacement dans l'Ambrosienne, une nouvelle église, projet porté par Ambroise. Le problème est évidemment dans le coeur de la crise de 385/386 : Ambroise doit céder devant l'empereur et la communauté arienne, quitte même à s'exiler, ce qu'évidemment il refuse, grandement soutenu par la population milanaise. Il met en place également des chants d'hymnes et psaumes pour apporter un mouvement de groupe contre la puissance impérial : Ambroise sait parfaitement manipuler toutes les armes à sa disposition pour instaurer son autorité. Entre-autre, Augustin parle dans ses confessions de ce moment et de la mise en place de ces chants pour conserver le soutien du peuple : « C'est alors que pour empêcher le peuple de se démoraliser à force d'ennui et d'inquiétude, on décida de lui faire chanter des hymnes et des psaumes comme cela se fait en Orient.» Augustin, Confessions livre 9, chapitre 7.

Autre citation pour montrer la ferveur du peuple en soutien d'Ambroise : « la foule des pieux fidèles passait les nuits dans l'Église, prête à mourir avec son évêque, votre serviteur » Augustin, Confessions livre 9 chapitre 7. = rappel de la popularité, du charisme et de l'autorité d'AMbroise à Milan

Contexte : Il s’enferme dans l’une des basiliques, protégé par ses soutiens. Après une défense allant jusqu’au sang, il remporte la partie face à un pouvoir impérial en déroute. La découverte de deux corps de martyrs milanais permet à Ambroise de conforter sa victoire et son prestige.

Vu par Palanque : En 384 arrivé d’Auxence dans l’entourage de Justine, donnant à la communauté arienne de Milan un évêque sur lequel s’appuyer. Mais pas de bâtiment alors en 385 Justine demande à Ambroise de lui céder une basilique. Il s' oppose frontalement devant le Consistoire. Mais une foule de fidèle vient manifester lourdement devant le siège impérial pour éviter une victoire des ariens. La Cour décide de retirer la demande et demande à Ambroise de calmer les émeutes. La passion populaire pour Ambroise le renforce alors fortement. Justine s’appuie sur l’autorité de Valentinien pour porter un projet de loi qui aboutit en janvier 386 : liberté de culte aux tenants de la foi de Rimini. Interdiction, sous peine de mort, de gêner cette liberté de culte. Amboise accuse Auxence d’être l’investigateur d’une telle loi et se déclare prêt à subir le martyre plutôt que lâcher sa position. Il s’enferme dans la basilique que veulent les ariens et une foule de fidèle s’y masse pour le soutenir : de nouveau le soutien populaire de Milan fait la différence. Échec du pouvoir impérial de lever la main sur Ambroise. Il fait connaître son refus à Valentinien par la Lettre 21 et montre qu’il n’abandonnera pas son Eglise. La Cour décide de prendre de force la basilique fin Mars sans succès, nouvelle victoire pour Ambroise.

Citations de la lettre 77 à apporter :

Grâce te soient-rendues, Seigneur Jésus, de nous avoir suscité une telle puissance spirituelle des saints martyrs en ce moment où ton Église ressent le besoin de plus grandes protections. [...] Les uns se glorifient de leurs chars et les autres de leurs chevaux, mais nous, nous nous glorifierons du nom de notre Seigneur Dieu. » 77, 10. Démonstration de la défense de la foi catholique contre la force impérial et la foi arienne grâce au soutien de Dieu représenté par les deux martyrs, un évènement que Ambroise, une fois encore, rattache à un passage des Écritures : la victoire de Ghiézi contre les Syriens, grâce au soutien de Dieu et des « soldats du Christ»


«J'aurais souhaité que Votre Majesté n'ait pas déclaré que je pouvais aller en exil où je le voulais.» pour montrer que l'empereur impose qqch de fort à Ambroise : le départ de Milan, lettre 75, 18.

Avis de Mclynn sur la question : Ambroise acculé par Valentinien doit sortir une arme de foi et de force collective pour retrouver sa prédominance. « Le thème de l'unité, propre à Ambroise, était de toute façon bien choisi pour séduire ceux qui étaient soucieux de rétablir la concorde à Milan. L'événement constituait tout autant une démonstration opportune de l'engagement de l'évêque envers cette cause qu'un déploiement triomphal de la force de son parti. C'est là que réside l'explication de sa réussite à briser la frénésie de la persécution » page 215 Ambroise fait ici, et c'est ce que rappel Mclynn, appel à 2 aspects défensifs à la fois : une logique liturgique indépassable par les corps des martyrs ce qui lui permet de s'opposer aux ariens + une défense par le nombre, la force du groupe, la ferveur populaire en faisant venir des fidèles à la basilique par le biais des martys, il s'assure une impossibilité d'intervention du coté du camp impérial. C'est ce qui amène cette « fin de persécution », victoire sans partage d'Ambroise.

Tout au long de ses lettres, et donc ici à sa soeur, il expose tout les outils disponibles pour conserver une autorité politique et religieuses, que ce soit par un droit légal à la parole, par la création d'une intimité indisociable donnant un poid personnel à ses interventions, ou enfin dans une moindre mesure mais utile otut de même, par l'utilisation plus strict de son rôle d'évêque, capable d; invoquer des arguments liturgiques et de fédérer le peuple de Milan derrière sa cause.

Et d'une certaie façon, c'est ce que rappel Brown ici, dans un passage sur la mise ne place d'une pénitence sur l'empereur Théodose : Ambroise peut utiliser tous les atouts de rhétoriques et d'actions : le philosophe, le politicien, ou l'évêque proche du peuple. Ambroise possède qqch que n’ont pas les philosophes : c’est le maître de la basilique de Milan, il est au cœur des cérémonies impériales et donc au coeur de l’image de Théodose auprès du peuple. Il est nécessaire pour un empereur d’avoir un espace rituel dans les villes de résidences. Pour Théodose, se réconcilier avec Ambroise voulait aussi dire regagner un rôle de premier plan dans la grande-messe solennelle à la cathédrale de Milan [Peter Brown] (trouver citation précise pour conclure sur Brown)
