Voir G. Nauroy : la méthode de composition et la structure du de Isaac et beata vida
Revenir sur Brown pour la partie sur Thessalonique car des mots intéressants.
dans le 2.1.2 revenir sur Mclynn et le début de l'oraison funèbre pour une analyse plus précise.
Important : il me faut revenir sur l’idée d’une aucotritas de l’Église issu du divin, c’et vrai mais c’est pas novateur par rapport à l’auctoritas de l’empereur qui est divinisé. Quel est le changement, comment ambroise s’y prend il etc + qu’en est il d’ailleurs de l’idée d’un empereur divin, arrete t il d’utiliser ce terme, en parle t il clairement ou le passe t il sous silence ? Voir surtout dans les horizons funèbres.
checker Peter Brown et essayer de + le citer
sur la libertas dicendi, dire que ça amène les évêques à être les meilleurs conseillers + ajouter des trucs sur le fait d'être conseiller


Le terme d’auctoritas dans les textes d’Ambroise :
Utilisation de l’idée d’auctoritas dans le domaine religieux, il l’utilise en De officiis III 129-132 ?
L’auctoritas apparait comme une qualité permenente, une désignation de la dignité. Il en parle en Ep 37 8, en parlant d’isaac qui place jacob au dessu d Esau
Il utilise ce terme de facon assez large : qqn ayant la foi intérieur, qqn ayant le droi d’agir etc, le sujet de l’auctoritas peut même être un comporterment de qqn.
p. 122, lors de l’éloge funèbre à Valentinien II, il loue la iustitia et l’auctoritas de l’empereur défunt, mais plutot comme une qualité personnelle que un point de vu politique.
