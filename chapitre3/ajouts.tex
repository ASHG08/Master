Voir G. Nauroy : la méthode de composition et la structure du de Isaac et beata vida
Revenir sur Brown pour la partie sur Thessalonique car des mots intéressants.
dans le 2.1.2 revenir sur Mclynn et le début de l'oraison funèbre pour une analyse plus précise.
Important : il me faut revenir sur l’idée d’une aucotritas de l’Eglise issu du divin, c’et vrai mais c’est pas novateur par rapport à l’auctoritas de l’empereur qui est divinisé. Quel est le changement, comment ambroise s’y prend il etc + qu’en est il d’ailleurs de l’idée d’un empereur divin, arrete t il d’utiliser ce terme, en parle t il clairement ou le passe t il sous silence ? Voir surtout dans les horizons funèbres.
checker Peter Brown et essayer de + le citer
