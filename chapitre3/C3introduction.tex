\section*{Introduction}

À la fin du IV\textsuperscript{e} siècle, les relations entre l'Église et l'Empire se voient fortement évoluer : l'intervention fréquente des empereurs dans les affaires de foi depuis Constantin est de plus en plus critiquée au sein de l'Église qui peine à se détacher de l'influence du pouvoir impérial. Lors de son arrivée dans l'épiscopat de Milan, Ambroise ne tarde pas à se montrer comme un penseur autonome, prêt à défier les empereurs pour servir sa vision de la foi chrétienne et lutter contre ce qu'il considère être des hérésies, dont certaines sont soutenues par le pouvoir impérial. Si les chapitres précédents ont permis d'établir les fondements théoriques de la pensée politique d'Ambroise ainsi que sa conception d'un pouvoir chrétien idéal sous le regard de la foi, il convient désormais d'analyser comment cette réflexion s'incarne concrètement dans la réalité du gouvernement impérial. Le troisième chapitre de ce mémoire a pour objectif d'explorer la vision d'Ambroise sur les relations que le monde épiscopal doit entretenir avec le pouvoir impérial, aussi bien sur le plan institutionnel que personnel. Ce qui nous amène évidemment à regarder les diverses actions de l'évêque de Milan pour interagir de façon concrète avec les souverains, en tant que figure qui redéfinit le rôle d'évêque et de conseiller impérial.

\bigskip

Le thème central de cette étude est celui d'une nouvelle \latin{auctoritas} : celle de l'Église. C'est une idée chère à Ambroise qui, bien qu'il ne l'évoque pas de façon explicite, guide l'ensemble de ses réflexions et actions dans le champ politique. Il ne se contente plus de réclamer une liberté de culte ou une protection juridique pour les représentants de la foi chrétienne, mais il cherche à instaurer une forme d'autorité religieuse capable de s'impliquer de façon indépendante dans les affaires impériales. Cette \latin{auctoritas} doit pouvoir se faire sentir par divers domaines : le pouvoir intrinsèque de l'institution, l'autorité offerte par le fait d'être un représentant de la foi, la légitimité diplomatique à travers les compétences d'orateur, ou encore par la confiance octroyée par les empereurs grâce à la construction d'une relation personnelle. Autant de domaines théorisés et maîtrisés par Ambroise qui lui permettent d'agir de façon importante dans la politique romaine. Giuseppe Zecchini souligne particulièrement bien la réflexion d'Ambroise au sujet de l'\latin{auctoritas} dans son ouvrage sur la pensée politique romaine : «~l'évêque de Milan fut le premier à comprendre, théoriser et appliquer le nouveau rôle politique de l'Église, appelée à combler par son autorité le vide progressivement laissé par le Sénat.\autocite[???]{zecchini_pensiero_politico}~» Cette observation est capitale : l'historien suggère ici que l'\latin{auctoritas} dont veut s'emparer Ambroise est une captation d'une prérogative sénatoriale puis impériale et non pas une invention pour donner un plus grand pouvoir à l'Église.\footnote{Voir l'explication de l'\latin{auctoritas} dans les réflexions d'Ambroise dans le 1.1.1.} L'enjeu est donc de comprendre comment l'évêque de Milan met en place un travail de collaboration avec les empereurs tout en imposant une autonomie importante de l'Église dans les affaires religieuses.

\bigskip

Cette réflexion nous oblige à questionner la nature exacte de cette nouvelle autorité. Bien qu'il soit évident qu'Ambroise revendique une \latin{auctoritas} indiscutable sur les dogmes et les affaires ecclésiastiques, comment arrive-t-il à justifier l'arrivée d'une nouvelle autorité, concurrente sur certains points au pouvoir de l'empereur ? Un siècle plus tard, la séparation des sphères d'influence de l'autorité épiscopale et du pouvoir impérial semble être plus largement définie et l'\latin{auctoritas} du pouvoir spirituel atteint un tout autre niveau lorsque l'évêque de Rome Gélase écrit à l'empereur d'Orient Anastase pour lui rappeler que « la direction du monde est assurée de concert par l'\latin{auctoritas sacrata pontificum} et par la \latin{regalis potestas}\autocite{sassier_auctoritas}.~» L'Église parvient donc à consolider son autorité, y compris dans le domaine politique tout au long du V\textsuperscript{e} siècle. Mais que parvient à faire concrètement Ambroise sur ce sujet pendant son épiscopat ? Il est donc nécessaire d'observer la façon dont il construit cette nouvelle autorité, entre autonomie religieuse, nécessité du soutien impérial, conflits avec les représentants du pouvoir et enfin ambition personnelle principalement marquée par sa proximité avec les différents souverains.

\bigskip

L'ensemble de ce chapitre s'inscrit tout particulièrement dans la suite des analyses de Peter Brown développées dans son \textit{Pouvoir et Persuasion}\autocite{brown_power_1992}. L'historien y décrit l'arrivée des évêques comme une nouvelle figure d'autorité avec laquelle le pouvoir impérial est contraint de composer. Les philosophes et simples conseillers politiques sont dépassés par ces évêques dont la légitimité explose en tant que représentants d'une communauté chrétienne importante. Le dialogue entretenu diffère largement face à des interlocuteurs qui ne cherchent plus seulement à plaire mais à guider l'empereur dans un chemin de foi et de justice propre à la chrétienté. Cette figure de pouvoir et d'influence est particulièrement bien représentée par le personnage d'Ambroise qui use de sa popularité et de ses connaissances pour s'imposer comme un outil indispensable du pouvoir impérial.

\bigskip

Les différentes analyses et réflexions de ce chapitre s'appuient majoritairement sur le corpus des \latin{Epistolae} d'Ambroise plus que sur ses écrits religieux et théoriques. En effet, les traités dogmatiques ou exégétiques restent souvent au stade théorique, ce qui ne nous permet pas, dans le cadre d'un chapitre sur les relations au pouvoir et les actions politiques, de véritablement comprendre tout l'enjeu de cette \latin{auctoritas} ambrosienne et de son influence sur les directives impériales. Ses lettres, en revanche, nous permettent de faire le lien entre ses réflexions et ses actions. Elles agissent comme le reflet réaliste de son idéal, obligé de s'adapter à son interlocuteur et aux circonstances politiques pour imposer son autorité et celle de l'Église sans risquer la rupture définitive des relations. C'est donc principalement à partir de ce corpus que se perçoit la manière dont Ambroise amène l'Église à s'installer comme une institution majeure de l'Empire romain, tout en se frayant une place de conseiller impérial indispensable.

\bigskip

Ainsi, ce chapitre s'attache à démontrer comment Ambroise transforme ses réflexions théoriques en une influence politique concrète, à travers une analyse la plus complète possible de l'\latin{auctoritas} institutionnelle puis personnelle que développe Ambroise aux côtés des empereurs qu'il côtoie.
