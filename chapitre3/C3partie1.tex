\section{Une nouvelle dynamique entre l'Église et le pouvoir impérial}

\subsection*{Introduction}

Au coeur de la réflexion politique d’Ambroise se présente une complexité intéressante : comment affirmer l'existence d’une autorité de l’Église chrétienne, qui soit à la fois complètement autonome vis-à-vis du pouvoir impérial et capable de jouer un rôle politique important dans la gestion de la société, sans pour autant s'emparer d’un pouvoir temporel qui entrerait en conflit avec les prérogatives de l’empereur et de son administration. Si la séparation entre les domaines du politique et du religieux existe depuis le début du christianisme, Ambroise apporte des précisions sur les sphères d’influence, faisant de l’indépendance épiscopale une nécessité pour lancer des actions publiques. L’évêque de Milan réfléchit alors à l’ensemble des relations existantes entre les institutions de l’Église et du pouvoir impérial pour développer et légitimer l’autorité de l’Église dans la société. Ces relations sont notamment mises en évidence par le corpus de correspondance entre Ambroise et les empereurs. C’est ce qui amène Giuseppe Visonà à qualifier le livre dix des Lettres d'Ambroise de « manuel de nouvelle doctrine des rapports entre l’Église et l’État par son principal théoricien.\footnote{« \latin{E in definitiva un prontuario con la nuova dottrina Chiesa-Stato a firma del suo maggiore elabitore}.». \cite[33]{pizzolato_nec_timeo}.}~»

\bigskip

Dans cette perspective de questionnement sur le rôle de l’Église, Ambroise fait revenir le concept d’\latin{auctoritas}, pour le placer au service des évêques. Il est évident que cette idée se fait dans un objectif pleinement religieux de revendication de l’autorité divine, il ne faut pas y chercher une volonté de concurrence avec l’empereur et d’appropriation des missions administratives appartenant au pouvoir impérial. Cette première partie d’un chapitre centré sur le pouvoir politique de l’Église et d’Ambroise vise donc à comprendre les nouvelles relations dont parle Visonà, et la façon dont l’évêque de Milan les influence selon une vision qui lui est propre. En laissant pour l’instant de côté les relations personnelles et le rôle direct d’Ambroise auprès des empereurs, nous nous concentrerons sur la définition de cette \latin{auctoritas} de l’Église et sur la façon dont elle s’impose comme une institution indépendante et influente, redessinant l’équilibre des pouvoirs au sein de l’Empire romain.

\subsection{Introduire une autorité chrétienne}

\subsubsection{\latin{Auctoritas}: Héritage Républicain et appropriation ambrosienne}

Bien que l’on ne puisse pas qualifier de révolution politique l’appropriation du concept d’\latin{auctoritas} par Ambroise et l’Église chrétienne, il est tout de même nécessaire de comprendre l’impact qu’a l’utilisation de ce terme par une nouvelle autorité institutionnelle, qui cherche à se détacher des souvenirs glorieux du Sénat de la période républicaine, et de l’actuelle puissance de l’empereur, détenteur officiel de cette fameuse \latin{auctoritas} depuis Auguste. L’évêque de Milan est le premier personnage clérical à utiliser, de façon conséquente, le terme \latin{auctoritas}. Cette notion est par exemple pratiquement absente des traductions de l’Ecriture par Jérôme, alors qu’Ambroise l’utilise pour définir la provenance divine d’une sanction d’ordre spirituel \autocite[214]{sassier_auctoritas}. Pour pleinement saisir la portée de l’appropriation de ce concept par Ambroise, il est nécessaire de revenir sur certaines structures de la pensée romaine traditionnelle, qui englobe une séparation juridique entre les deux notions d’\latin{auctoritas}, souvent simplement traduit par autorité, et de\latin{potestas}, pouvant signifier puissance ou pouvoir. L’évêque de Milan est un ancien gouverneur et membre de l’aristocratie romaine, sa pensée politique est donc fondamentalement ancrée dans les traditions romaines. Il connaît parfaitement l’ancienne République et donc l’origine de la notion d’\latin{auctoritas}, ce qui lui permet de l’utiliser avec plus de justesse à son profit.

\bigskip

Il est difficile de comprendre avec précision ce qu’englobe l’\latin{auctoritas} en tant que notion juridique, c’est une idée implantée dans la société romaine qui n’est pas remise en question, mais également jamais franchement définie. Il est par contre clair qu’une séparation importante existe entre l’autorité et le pouvoir, des notions pouvant être dans les mains d’une ou de plusieurs personnes. Les historiens Michel Christol, Frédéric Hurlet et Pierre Cosme en parlent comme ceci : « L’autorité telle que les Romains la concevaient était un surcroît de pouvoir reconnu à un groupe où un individu, conférant une influence et un ascendant à celui qui en était le dépositaire. Elle s’exprimait à travers les initiatives que le groupe ou l’individu prenait et qui étaient suivies d’effet : elle était cette puissance qui permettait de l’emporter dans la prise de décision, sans avoir recours à la force. ». \autocite{christol_histoire_2021} Tout au long de la période républicaine, dans les institutions romaines, l’\latin{auctoritas} est plutôt dans les mains du Sénat, et la \latin{potestas}, dans celles des magistrats. L’autorité du Sénat n’est donc pas un pouvoir exécutif, mais l’institution peut présenter des décisions « obligatoires », que doivent appliquer les magistrats. C’est bien par leur influence, par l’\latin{auctoritas}, que s’exerce le véritable pouvoir du Sénat. C’est sur cette base juridique que va s’appuyer Ambroise pour instaurer une influence grandissante et pratiquement obligatoire de l’Église sur l’empereur.

\bigskip

Un premier changement juridique intervient avec Auguste, qui fait de cette distinction entre autorité et puissance le cœur de son renouveau politique. Le premier empereur s’impose très rapidement comme le détenteur unique de l’\latin{auctoritas}, à l’insu du Sénat qui s’efface dans un rôle presque uniquement symbolique. Dans le chapitre 34 de ses \latin{Res Gestae}\footnote{« \latin{Post id tempus auctoritate omnibus praestiti, potestatis autem nihilo amplius habui quam ceteri qui mihi quoque in magistratu conlegae fuerunt.}.» {Traduction de Frédéric Hurlet : « après cette époque, je l’ai emporté sur tous par mon \latin{auctoritas}, mais je n’ai pas eu plus de \latin{potestas} que tous les autres qui ont été mes collègues dans chaque magistrature. »}.}, Auguste décrit clairement sa supériorité par son autorité, tout en précisant qu’il possède un pouvoir identique à ceux des autres magistrats. Auguste est alors princeps, celui qui propose et influence, par son \latin{auctoritas}, et magistrat, celui qui exécute, par sa \latin{potestas}\autocite{magdelain_auctoritas}. Le pouvoir moral de l'\latin{auctoritas} fondé sur l’influence personnelle de l’empereur se développe tout au long de l’Empire pour devenir un statut juridique affirmé chez le pouvoir impérial. En étant pleinement conscient de cette culture politique, Ambroise de Milan se retrouve devant le double défi de faire renaître une autorité institutionnelle à travers l’Église, tout en supplantant l’\latin{auctoritas} impérial, à la fois par son statut d’évêque et de membre de l’Église que je développe dans cette partie, ainsi que par sa proximité avec les empereurs et ses relations personnelles, que je développerai dans une seconde partie du chapitre.

\bigskip

Ambroise ne rejette pas dans ses écrits le schéma juridique de l’\latin{auctoritas} et de la \latin{potestas}, au contraire, il s’en empare pleinement et cherche à donner à l’Église et une autorité et une influence capable de travailler à égalité avec le pouvoir impérial dans la gestion de la société romaine. Et pour légitimer le nouveau rôle qu’il attribue à l’Église chrétienne, Ambroise apporte un aspect divin à l’\latin{auctoritas}. Dans sa thèse sur l’utilisation du terme \latin{auctoritas} par Cyprien, Tertullien et Ambroise\autocite{ring_auctoritas}, Thomas Gerhard Ring démontre que l’évêque de Milan parle peu, explicitement, d’\latin{auctoritas}, et qu’il limite son utilisation aux formes d’autorités religieuses.

\textcolor{red}{Le terme d’auctoritas dans les textes d’Ambroise :
Utilisation de l’idée d’auctoritas dans le domaine religieux, il l’utilise en De officiis III 129-132 ?
L’auctoritas apparait comme une qualité permanente, une désignation de la dignité. Il en parle en Ep 37 8, en parlant d’isaac qui place jacob au dessu d Esau
Il utilise ce terme de facon assez large : qqn ayant la foi intérieur, qqn ayant le droi d’agir etc, le sujet de l’auctoritas peut même être un comporterment de qqn.
p. 122, lors de l’éloge funèbre à Valentinien II, il loue la iustitia et l’auctoritas de l’empereur défunt, mais plutot comme une qualité personnelle que un point de vu politique.} Je développerai cet aspect tout au long du chapitre, mais il n’y a pas chez Ambroise de volonté de soustraire à l’empereur une partie de son pouvoir temporel. Ainsi, il trouve les fondements de l’autorité épiscopale dans les volontés divines ou dans les apôtres, des formes d’autorité incontestable pour un empereur chrétien. C’est cette exemple qu’utilise également Yves Sassier que j’ai cité précédemment :

\bigskip

\begin{quote}
    « Tu vois qu’il condamne cet homme par l’autorité apostolique [...] et pourtant il ne lui a pas ôté l’espérance, lui qu'il a invité à la pénitence\footnote{« \latin{Vides quod hunc apostolica auctoritate condemnet [...] et tamen non interclusit ei spem, quem invitavit ad poenitentiam.}~». Ambroise, De poenitentia, II, 4, 40. Traduction personnelle.}~»
\end{quote}

\bigskip

L’utilisation de la notion d’\latin{auctoritas} avec celle d’apostolica permet de définir une sanction d’ordre spirituel. Et c’est à partir de ce fondement chrétien que l’évêque développe sa réflexion : l’autorité divine est immuable et évidente, par conséquent l’Église en tant que corps du Christ et l’évêque en tant que successeur des apôtres se retrouvent en possession de cette nouvelle forme d’autorité. L’\latin{auctoritas} divine apparaît dans la société romaine christianisée comme une puissance absolue et contraignante pour la conscience, la rendant bien supérieure à l'influence politique traditionnelle, ce qui amène une forme d’autorité innée chez les évêques, c’est une idée que je développe dans le point suivant. Thomas Gerhard Ring pousse les comparaisons, et la compréhension l'utilisation du terme par Ambroise, un peu plus loin. L’évêque de Milan s’appuie très nettement sur la pensée de Tertullien qui définit l’\latin{auctoritas} divine comme la volonté révélée de Dieu qui ne peut donc qu’être acceptée et suivie par les Hommes ayant la foi chrétienne. Il y a donc une rupture avec l’origine première de l’\latin{auctoritas} présent chez le Sénat ou l’empereur. L’historien allemand la décrit comme fonctionnant de manière horizontale, d’humains à humains, grâce à un consensus social qui reconnaît les qualités d’un individu ou d’une institution. Avec Ambroise, la nouvelle vision de l’\latin{auctoritas} est verticale, elle provient du divin et descend de Dieu vers l’Église pour créer une \latin{auctoritas} épiscopale qui influence les décisions impériales\autocite[À partir de la page 163]{ring_auctoritas}.

\bigskip

La réflexion ambrosienne sur la nouvelle autorité de l’Église, largement décrite par T.G. Ring, ne s’arrête pourtant pas à un phénomène de théologie abstrait. Ambroise évoque à plusieurs reprises la réalisation concrète, et bien connue dans l’empire, d’une autorité institutionnelle et juridique dans l’Église chrétienne : l’audientia episcopalis. Cette instance est une forme de tribunal épiscopal, utilisé depuis plusieurs siècles par les communautés chrétiennes, puis rendu légal et officiel par Constantin en 318, qui permet aux évêques de rendre la justice sur des préoccupations civiles, à partir du moment où l'une des deux parties souhaite y avoir recours. Dans le cadre que j’ai posé tout au long de cette sous-partie, il est possible de concevoir cette fonction de l’évêque comme la réalisation de cette \latin{auctoritas}. La justice de l’évêque est en effet demandée car son jugement est vu comme rapide, non corrompu comme peuvent l’être certaines institutions romaines, mais surtout fondé sur une influence divine, bien moins contestable que le droit civil.

\bigskip

\begin{quote}
    « Quelle autorité a-t-il pour dénoncer la fraude celui qui a pu saisir l’appât si laid des plaisirs ? Celui en effet qui accuse autrui de péché doit être lui-même exempt de péché. Je n’en appellerai donc pas à des balivernes de ce genre pour appuyer sur ce point l’autorité de la censure de l’Église [...]\footnote{«~\latin{Quam hic redarguendi habet auctoritatem doli, qui tam turpe captarit aucupium deliciarum. Qui enim alterum peccati arguit, ipse a peccato debet alienus esse. Non ergo huiusmodi nugas ego in hanc ecclesiasticae censionis auctoritatem uocabo [...]}~». \cite[\nopp \latin{De Officiis} III, 72]{ambroise_devoirs_2}.}~»
\end{quote}

\bigskip

Bien qu’adressé à son clergé, cet extrait de son traité \latin{De Officiis} nous permet de réaliser le recours récurrent, et même nécessaire pour l’empire, à cette forme d’autorité et de législation. On peut tout de même noter ici qu’Ambroise met également en évidence les limites de la juridiction cléricale : le tribunal repose sur les notions de péchés, de foi et de morale, il y est donc impossible d’y juger des affaires purement civiles si les deux parties sont considérés comme étant dans l’erreur et le péché, comme c’est le cas dans cet exemple autour de l’achat d’une propriété. Dès lors, on retrouve cette importance chez Ambroise de développer une nouvelle autorité sans entrer en concurrence avec les formes de pouvoirs temporels existantes. L’\latin{auctoritas} que l’évêque implante dans l’Église chrétienne ne se fonde pas sur la seule motivation de contester le pouvoir absolu du système impérial et des institutions qui en sont issues. Au contraire, il cherche à travailler en parallèle de l’empereur et profite d’une foi chrétienne et d’une autorité divine très largement diffusée et respectée pour développer considérablement le rôle de l’Église et des évêques dans divers domaines de la société.

\bigskip

\subsubsection{Imposer les évêques au coeur du gouvernement impérial}

Le postulat d’Ambroise est donc que l’\latin{auctoritas} sur laquelle doit s’appuyer l’Église chrétienne émane du divin, ce qui légitime l’institution à exercer un pouvoir sur la société sans en référer à l’empereur. En suivant le raisonnement, cette \latin{auctoritas} permet également aux évêques de s’imposer comme des membres à part entière du système administratif impérial, et donc comme conseillers ou juges dans les affaires religieuses. Grâce à l’aspect divin derrière ce changement, découle une autorité innée des évêques, qui ne gagnent pas leur influence par une concession ou un soutien impérial, mais qui provient simplement de sa charge. L’évêque, par le simple fait d’être évêque, est la seule source légitime d’autorité pour toute question relative à la foi ou la morale, créant une forme d'expertise religieuse que le pouvoir se doit de reconnaître et consulter. Ambroise pousse cette idée sur le principe simple de la compétence professionnelle : il est nécessaire d’être établi et reconnu comme autorité religieuse pour pouvoir juger la foi, et ce peu importe son importance civile. Dans le cadre de ce qui est aujourd’hui appelé le conflit des basiliques milanaises\footnote{Crise religieuse entre 385 et 386 opposant la communauté arienne de Milan, derrière le personnage d’Auxence et soutenu par la mère de l’empereur, Justine, à Ambroise. Il est question du don d’une basilique aux ariens, ce à quoi l’évêque de Milan s’oppose fermement, allant jusqu’à s’écarter de l’avis de Valentinien II, au nom de la foi chrétienne.}, l’empereur Valentinien II convoque Ambroise au consistoire afin qu'ait lieu un débat contre Auxence, évêque arien fidèle au dogme de Rimini. Une affaire religieuse, en l'occurrence un conflit entre deux pensées chrétiennes opposées, doit donc être réglée par l’empereur et chez l’empereur. Evidemment, Ambroise repousse fermement cette idée et évoque, pour se défendre et éviter de se faire qualifier de « rebelle », l’expertise religieuse des évêques et la nécessité de les consulter :

\bigskip

\begin{quote}
    « À cette convocation je fais, je pense, la réponse qui convient. Et personne ne saurait me juger rebelle quand je soutiens ce que ton père, d’auguste mémoire, a non seulement prescrit oralement mais aussi sanctionné par ses lois : « Dans une affaire concernant la foi ou quelque dignité ecclésiastique, doit juger celui qui n’est pas d’un rang inférieur et n’a pas un statut juridique différent. » Ce sont les mots du rescrit\footnote{Ces textes de lois ne nous sont pas parvenus.}, autrement dit, il a voulu que ce soient des évêques qui jugent des évêques.\footnote{« \latin{Cui rei respondeo, ut arbitror, competenter. Nec quisquam contumacem iudicre me debet, cum hoc asseram quod augustae memoriae pater tuus non solum sermone respondit sed etiam legibus suis sanxit : « In causa fidei uel ecclesiastici alicuius ordinis eum iudicare debere qui nec munere impar sit nec iure dissimilis. » Haec enim uerba rescripti sunt hoc es sacerdotibus uoluit iudicare.}~». \cite[\nopp 75,2]{ambroise_lettres}.}~»
\end{quote}

\bigskip

Bien qu’il soit important de noter que ce conflit ne nous est connu que par la vision d’Ambroise et de ses proches, tel que Paulin, et qu’il est donc impossible de connaître les véritables objectifs et volontés de Valentinien ou Auxence et leur réaction après les écrits d’Ambroise, cette citation permet de constater la mise en place du respect de l’autorité ecclésiastique, tel que le veut notre évêque. Bien sûr, Ambroise sait se plier aux exigences du pouvoir impérial, il précise d’ailleurs dans un autre passage qu’il se rendra au consistoire si telle est la volonté de l’empereur, en revanche il cherche à aborder la question du rôle des évêques avec plus de précision et de justesse pour toucher directement Valentinien II et apporter une grande compréhension de la situation. Ici, en citant Valentinien Ier, il place l’actuel empereur face à une autorité plus personnelle et respectée dans l’empire, et affirme ainsi la position de l’Église par une légitimité juridique de longue date. L’application concrète dans cette lettre du recours, désormais obligatoire, aux évêques reflète parfaitement l’objectif théorique d’Ambroise : faire en sorte que l’Église ne soit plus une simple administration religieuse mais une institution incontournable dans la gestion de l’empire romain. À partir de cette perspective, Ambroise refuse que les débats de foi se tiennent dans le consistoire, lieu de la \latin{potestas} impériale par excellence, afin d'éviter de voir le déplacement de l’autorité épiscopale vers  l’empereur :

\bigskip

\begin{quote}
    « Je me rendrai volontiers au palais de l’empereur si c’était en accord avec mon devoir d’évêque, pour mener ce débat dans le palais plutôt que dans l’église. Mais, dans le consistoire, l’usage veut que le Christ soit non pas un accusé, mais un juge. Une question de foi c’est dans l’Église qu’elle doit être plaidé, qui peut dire le contraire ?\footnote{« \latin{Ad palatium imperatoris irem libenter, si hoc congrueret sacerdotis officio, ut in palatio magis certarem quam in ecclesia. Sed in consistorio non reus solet Christus esse sed iudex. Causam fidei in ecclesia agendam quis abnuat ?}~». \cite[\nopp 75A,3]{ambroise_lettres}.}~»
\end{quote}

\bigskip

L’autorité de l’Église apparaît ainsi comme une réalité avec laquelle l’empereur se doit de travailler. Il ne s’agit pas d’une concurrence mais bien d’un travail en commun où chaque institution contrôle et agit dans son domaine d’expertise. Il n’est donc pas possible pour des affaires de foi de se régler auprès d’une justice impériale, car celà mettrait les évêques directement sous les ordres de l’empereur. Ambroise semble pousser les empereurs, et en particulier Valentinien II avec qui il partage une plus forte relation, à intégrer l’institution ecclésiale dans son processus de prise de décision, comme c’est le cas pour les autres instances du gouvernement :

\bigskip

\begin{quote}
    « Du reste, si on ne me fait pas assez confiance, fais venir les évêques que tu voudras, qu’on débatte, empereur, des mesures à prendre dans le respect de la foi. Si sur des affaires financières, tu prends l’avis de tes comtes, combien est-il plus équitable que, sur une affaire qui concerne la religion, tu prennes l’avis des prêtres du Seigneur.\footnote{« \latin{Certe si mihi parum fidei defertur, iube adesse quos putaueris episcopos, tractetur, imperator, quid salua fide agi debeat. Si de causis pecuniariis comites tuos consulis, quanto magis in causa religionis sacerdotes Domini aequum est consulas.}~». \cite[\nopp 74,27]{ambroise_lettres}.}~»
\end{quote}

\bigskip

La comparaison qui est faite avec les cours financières n’est pas anodine, Ambroise évoque l’un des aspects les plus importants du pouvoir impérial, à la fois pour placer l’Église comme une institution de premier plan dans la gestion de l’empire, mais également pour rappeler à l’empereur qu’il possède toujours la mainmise dans la plupart des secteurs du pouvoir, les questions économiques et militaires entre autres. L’idée, et je continuerai de la développer tout au long de ce chapitre, est vraiment d’imposer l’autorité ecclésiastique sans « effrayer » le pouvoir impérial qui pourrait avoir le sentiment d’une perte sèche de capacité politique. De la même façon que les magistrats en charge des impôts ou du budget sont indispensables dans la gestion de la société, il n’est plus possible de diriger un empire chrétien sans s’appuyer sur l’autorité des évêques. Ambroise établit véritablement l’Église comme une institution indispensable du gouvernement impérial. La relation entre les évêques et l’empereur se forme donc avec Ambroise dans un esprit de collaboration nécessaire, avec tout de même l’idée, et l’évêque de Milan n’y manque pas, de revendiquer une pleine et entière autonomie de l’Église.

\bigskip

\subsubsection{La défense d'une indépendance cléricale}

En se revendiquant seule experte des questions de la foi, l’Église se doit d'administrer et de contrôler ses interventions elle-même, sans ingérence impériale, c’est ce qui constitue le cœur de la défense de l’indépendance épiscopale. L’important corpus de lettres que nous possédons, d’Ambroise s’adressant aux empereurs, permet de pousser la compréhension de l’objectif d’Ambroise vis-à-vis du développement d’une \latin{auctoritas} des « prêtres du Seigneur » qui se rapproche de l’influence que possédait le Sénat dans la période Républicaine. Ainsi, après avoir démontré le droit des évêques d’intervenir auprès de l’empereur lors des questions religieuses, Ambroise cherche à clarifier l’idée d’autonomie qui doit émaner de l’Église chrétienne. Comme indiqué précédemment, l’\latin{auctoritas}  de l’Église provient directement du divin, donc une autorité intrinsèque et qui ne peut pas se soumettre au pouvoir impérial. Cette autonomie est en effet nécessaire pour que puisse s’exercer sans contestation la nouvelle autorité spirituelle. Charles Norris Cochrane s’arrête un moment sur les détails de l’indépendance que vise Ambroise. On peut notamment comprendre qu’elle englobe à la fois « l'autodétermination de l’organisme et la liberté des ministres, en tant que représentant, de s’exprimer et d’agir comme ils l’entendent.\autocite[347]{cochrane_christianity}~». L’idée d’Ambroise n’est en aucun cas d’interdire l’empereur d’intervenir dans toutes questions concernant la foi d’un citoyen ou la gestion interne de l’Église, mais bien de limiter ces interventions à des cas demandés par les membres de l’Église\footnote{Je reviens longuement sur les droits et devoirs des interventions impériales dans le domaine religieux en 1.2.1 et 1.2.2.}. Mais pour en arriver à délimiter les interventions impériales, il est nécessaire d’imposer l’indépendance de l’Église dans les relations politiques. Et évidemment ce que l’on peut immédiatement citer est de nouveau un extrait de la lettre 75 adressée à Valentinien II, qui met de nouveau en avant l’autorité ecclésiastique dans les affaires religieuses et rappelle le rôle du souverain en matière de foi :

\bigskip

\begin{quote}
    « Quand as-tu entendu dire, empereur très clément, que, dans une affaire touchant la foi, des laïcs aient jugé le cas d’un évêque ? [...] En tout cas si nous passons en revue les livres des divines Ecritures et les temps anciens, quelle est la personne qui peut contester que, dans une affaire touchant la foi, je dis bien une affaire touchant la foi, ce sont les évêques qui jugent habituellement la conduite des empereurs chrétiens, et non pas les empereurs celle des évêques ?\footnote{« \latin{}~». \cite[\nopp 75,4]{ambroise_lettres}.}~»
\end{quote}

\bigskip

Ce passage se révèle particulièrement intéressant pour comprendre la pensée d’Ambroise au sujet de l’autonomie épiscopale. Une chose frappe ou dérange dès la première lecture : l’appuie qui est fait sur la notion « d’affaire touchant la foi », « \latin{in causa fidei} ». Tout comme je l’ai évoqué lors de la précédente partie, Ambroise veut accentuer l’idée que l’\latin{auctoritas} de l’Église ne s’impose que dans un seul domaine : la foi. L’objectif derrière est double, premièrement rassurer le pouvoir impérial sur le fait que le développement d’une nouvelle forme d’autorité ne vient pas mettre en péril la \latin{potestas} du prince, et deuxièmement le fait de recentrer les volontés de l’Église autour d’un seul thème garantit un contrôle absolu de cet aspect, et donc une plus grande autonomie. Un autre point intéressant de la citation est le recours de deux autorités incontestable pour un aristocrate chrétien : les Écritures et les temps anciens. Si cette idée peut paraître un peu vague dans ce à quoi elle fait référence, il faut savoir que l’expression « livres ou textes des Saintes Écritures » est très fréquente dans ses travaux. Gérard Nauroy dénombre 16 occurrences et désigne ce terme comme « la succession des divers livres qui constituent le corpus des Écritures.\footnote{Gérard Nauroy, note 5, page 503. \cite[\nopp 77,3]{ambroise_lettres}.}~». Ambroise en appel donc à une mémoire chrétienne plus que romaine, faisant confiance à la culture religieuse de l’empereur, pour soumettre la légitimité de sa vision derrière une autorité textuelle capable de s’imposer sur les décisions impériales.

\bigskip

Mais l’indépendance de l’épiscopat apparaît par cette lettre et par ce conflit comme une lutte « de son temps » plutôt qu’un combat théorique et politique plus important. Dans le même contexte, une autre lettre nous permet de constater que ce combat pour l’autonomie mené par Ambroise n’est pas un simple opportunisme d’occasion mais bien le cœur de ses objectifs longs termes pour l’Église. Il s’agit de la lettre 76 adressée à sa sœur, Marcelina, qui semble, selon ses dires, se préoccuper de l’avenir de l’Église et de l'épiscopat de Milan. Cette source est donc bien moins « officielle » qu’un texte qui doit être lu par un empereur, elle est donc à même de refléter avec plus de justesse sa pensée. Et même s’il est possible que l’évêque de Milan mente sur certains sujets, il est difficile de contrôler la véracité de l’ensemble des faits,  il n’exprime finalement ici que son idéal théorique à propos des relations entre l’Église et le pouvoir impérial. Il n’a plus besoin de camoufler le fond de sa pensée derrière des formules de politesse. Le ton est alors bien plus direct, ne laissant pas de doute à sa volonté d'indépendance de l’\latin{auctoritas} de l’Église :

\bigskip

\begin{quote}
    « Bref, on me fait dire : « Livre la basilique ». Je réponds : « Il ne m’est pas permis de la livrer et il n’est pas dans ton intérêt, empereur, de la recevoir. Alors que tu ne peux en droit mettre la main sur la maison d’un particulier, penses-tu que tu peux te saisir de la maison de Dieu ? » On prétend que tout est permis à l’empereur car tout lui appartient. Je ŕeponds : « Ne charge pas ta conscience , empereur, en pensant que tu disposes de quelque droit impérial sur les choses divines. [...]\footnote{« \latin{Mandatur denique : « Trade basilicam. » Respondeo : « Nec mihi fas est tradere nec tibi accipere, imperator, expedit. Domum priuati nullo potes iure temerare, domum Dei existimas auferendam ? » Allegatur imperatori licere omnia ipsius esse uniuersa. Respondeo : « Noli te grauare imperator, ut putes te in ea quae diuina sunt imperiale aliquod ius habere.}~». \cite[\nopp 76,19]{ambroise_lettres}.}~»
\end{quote}

\bigskip

Ambroise met ici en place l’argument d’une autorité ecclésial fondée sur son espace public. L’évêque de Milan est conscient de la nécessité d’englober toutes les formes d’autorités possibles pour légitimer la nouvelle place de l’Église chrétienne. Replacer la figure de Dieu derrière la symbolique des bâtiments religieux comme les églises ou les basiliques fait partie de l’ensemble du processus permettant une reconnaissance politique par le pouvoir impérial. La séparation bien connue, aussi bien dans la pensée chrétienne que romaine, entre les biens publics et les biens sacrés permet de continuer dans le sens de la distinction nette des domaines d’actions des deux autorités, définissant toujours plus la relation entre l’Église et le l’empereur comme deux autorités complémentaires pour l’empire. La dernière phrase du passage que j’ai choisi de citer permet de clarifier une fois pour toute le fond de la pensée ambrosienne à travers l’autonomie cléricale et la liberté politiques des figures comme Ambroise.

\bigskip

Il est donc important de retenir qu’il n’y a dans cette pensée aucune soumission d’une autorité envers une autre. Ambroise ne cherche pas à mettre l’empereur sous la puissance de l'Église, mais ne veut pas voir non plus l’empereur interférer de son plein gré dans les décisions épiscopales. L’indépendance recherchée devient un prérequis nécessaire à l’entretien de l’\latin{auctoritas} des évêques et à leur place dans le gouvernement impérial, comme je l’ai abordé précédemment. Il est désormais important de comprendre la façon dont l’évêque de Milan veut que l’empereur se comporte devant cette autorité, en regardant notamment les droits d’interventions que lui laisse Ambroise, ou les priorités politiques qui lui sont attribuées.

\bigskip

\subsection{L'empereur face à cette \latin{auctoritas}}

\subsubsection{La priorité du christianisme sur la société romaine}

Ambroise s’efforce, notamment à travers ses communications avec les empereurs, de construire une indépendance de l’autorité et des champs d’action de l’Église, dans l’objectif de se constituer une défense contre les empiètements du pouvoir impérial. Mais derrière cette action politique ne se cache pas l’envie de rendre les domaines de la religion inaccessibles, bien au contraire l’évêque de Milan exige que la \latin{potestas} de l’empereur se mette activement, et de façon prioritaire, au service de la foi. Ambroise théorise assez clairement une hiérarchie des domaines d’action où la cause de la religion doit systématiquement l’emporter sur la rigueur du droit civil et même sur le maintien de l'ordre public.

\bigskip

Ce principe de priorité est tout d’abord évoqué dans ses lettres à Valentinien II en 384, lors d’un moment de tension au sujet de la restitution de l’autel de la Victoire dans la curie sénatoriale. Le jeune empereur fait ici face à un dilemme à la fois moral et politique qui rentre exactement dans le questionnement politique d’Ambroise. La \latin{relatio} de Symmaque, porte-parole de la frange païenne du Sénat qui vise le retour de la Victoire, démontre à Valentinien l'importance des traditions romaines et de la stabilité de cette institution ayant des membres toujours majoritairement païens. L’intervention d’Ambroise, d'abord en prévention avec l’\latin{Epistolae} 72, puis de nouveau et avec plus de précision après la lecture de la \latin{relatio}, à travers l’\latin{Epistolae} 73, a pour objectif de convaincre l’empereur que le choix en faveur de la foi chrétienne, et donc du rejet de la délégation du Sénat, est toujours le choix juste et même le seul choix valable :

\bigskip

\begin{quote}
    « Quand il s’agit de décider d’une affaire militaire, il convient de prendre l’avis d’un homme exercé aux combats et d’adopter son conseil ; quand il s’agit de religion, songe à Dieu. On ne fait aucun tort à qui on préfère le Dieu tout-puissant. Cet homme a son opinion, vous ne le contraignez pas à pratiquer malgré lui un culte dont il ne veut pas. Que le même droit vous soit accordé, empereur, [...]\footnote{« \latin{Si de re militari est consulendum, debet exercitati in proeliis uiri expectari sententia, consilium comprobari ; quando de religione tractatus est, Deum cogita. Nullius iniuria est cui Deus omnipotens antefertur. Habet ille sententiam suam. Inuitum non cogitis colere quod nolit. Hoc idem uobis liceat, imperator, [...]}~». \cite[\nopp 72, 7]{ambroise_lettres}.}~»
\end{quote}

\bigskip

Le parallèle entre l’expertise militaire et les questions religieuses a pour but de placer Dieu au-dessus de toute réflexion politique. Ambroise amène cette idée presque sous la forme d’un rappel à l’ordre : songe à Dieu. L’ensemble de la politique dans l’Empire doit se faire derrière le regard de Dieu et en ayant cette foi comme seul véritable chemin de justice. Ambroise amène Valentinien II à ce constat par un détour intéressant : il transforme la foi personnelle du souverain en un acte politique, faisant de la préférence du Dieu unique une obligation à l’échelle de l’Empire. De la même manière que Symmaque, puisque c’est de lui dont il est question dans le « cet homme », a le droit d’apporter son avis sur la présence de l’autel de la Victoire dans le Sénat, l'empereur a le droit, et donc ici même le devoir, de préférer Dieu dans sa prise de décision. La précision « on ne fait aucun tort » ne s’applique pas seulement pour le sénateur ni pour une personne seule, mais bien pour l'ensemble des décisions impériales. La fidélité à Dieu empêche toute accusation d’injustice envers le souverain, du moins selon la vision ambrosienne. L’évêque, par ce passage, introduit l’idée d’une priorité de la religion sur les autres questions : il légitime ici, par avance, toutes les décisions pouvant aller en faveur de l’Église. Une fois encore, Ambroise amène le jeune empereur dans une direction qui lui est favorable, en faisant appel à l’autorité de Dieu et de la foi chrétienne.

\bigskip

Cette approche politique, nouvellement mise en valeur par l’évêque de Milan, trouve son application la plus concrète et la plus stricte au cœur de l'\latin{Epistolae} 74 sur l’affaire de Callinicum, qui cette fois concerne l’empereur Théodose. Dans cet événement dont nous avons déjà parlé précédemment\footnote{Voir ???}, la logique administrative et impériale de l’Empire romain semble respectée à travers la décision prise par Théodose. Celui-ci décide de punir l’évêque de Callinicum responsable du mouvement ayant incendié la synagogue. Ainsi, même s’il s’agit d’un acte allant contre l’Église chrétienne, il permet un maintien de l’ordre cohérent, et donc la continuité dans l’ordre public. Mais Ambroise ne voit pas cette décision comme une aide à la société romaine, mais bien comme un affront à la foi et à Dieu. Bien qu’il agisse après la décision impériale, il se permet, dans sa lettre envoyée au souverain, des paroles virulentes ne laissant aucun doute sur la position de l’évêque. Dans sa thèse sur Ambroise, Jean-Rémy Palanque émet même l’hypothèse qu’il puisse s’agir d’une façon pour Ambroise « d’humilier l’empereur d’Orient dès son arrivée en Italie\autocite[374]{palanque_ambroise}.~». Bien qu’il soit impossible de connaître toute l’ampleur de la réflexion d’Ambroise derrière sa lettre 74, il est évident qu’elle vise en partie à imposer, dans la politique impériale, un principe de priorité religieuse qui donne à la foi chrétienne une position de domination constante face aux autres préoccupations de l’Empire. L'aboutissement de cette pensée se fait ressentir par cette phrase :

\bigskip

\begin{quote}
    « Mais c’est la considération de l’ordre public qui te préoccupe, empereur. Qu’est-ce qui a le plus d’importance, le prétexte de l’ordre public ou la cause de la religion ? La répression doit céder devant la dévotion.\footnote{« \latin{Sed disciplinae te ratio, imperator, mouet. Quid igitur est amplius, disciplinae species an causa religionis ? Cedat oportet censura deuotioni.}~». \cite[\nopp 74, 11]{ambroise_lettres}.}~»
\end{quote}

\bigskip

Le vocabulaire employé a son importance : en qualifiant le maintien de l'ordre public de « prétexte », Ambroise soumet l'un des concepts clés du pouvoir impérial à Dieu. L'ordre dans la société n'est, selon l'évêque, plus légitime d'être maintenu s'il va à l'encontre du christianisme, peu importe sa forme. Dans cet extrait, on peut constater une hiérarchie stricte, que cherche à installer Ambroise, qui place « la cause de la religion », donc le fait d'agir en faveur de la foi chrétienne, au-dessus de toute autre considération politique. Le terme « prétexte » dévalorise l'argument politique de l'empereur, soumis à sa religion et principalement ici à l'autorité d'Ambroise. L'opposition entre répression et dévotion insiste sur cette forme de soumission de la moralité civile à la foi chrétienne. Théodose se voit accusé d'avoir utilisé son droit fondamental de punition contre la dévotion. Ainsi, avec Ambroise, la suprématie chrétienne est prête à entacher la paix sociale, en favorisant la lutte religieuse à l'entente dans l'Empire.

\bigskip

Il ne semble donc, dans la pensée ambrosienne, n'y avoir pas plus important que la sauvegarde du christianisme et de tout ce qui en découle. Plusieurs historiens se sont penchés sur cette exigence qui amène une nouvelle conception de la fonction impériale. C'est en particulier le cas de Michael Müller dans son article sur le traitement des affaires politiques et religieuses par Ambroise, dans lequel il s'attarde sur le rôle de l'empereur : « Cet argument est justifié par l’ébauche d'une conception du pouvoir dans laquelle incombe à l’empereur, parce qu’il est à la tête de l’Empire romain, une obligation particulière, celle, en tant que premier serviteur du Dieu des chrétiens, de protéger et de propager la foi de ce même Dieu dans le monde\autocite[???]{muller_conflits}.~». Ce qui rejoint fortement l'idée de Thomas Ring qui note qu'Ambroise ne tolère aucune distinction entre la confession privée et l'action politique\autocite[124]{ring_auctoritas}. C'est-à-dire que l'empereur, par le simple fait d'être chrétien, se doit d'intervenir politiquement en faveur du christianisme et de l'Église. Il ne peut pas se cacher derrière une relation privée à la foi, ni agir de façon impartiale sur les questions religieuses en se cachant derrière son rôle politique. Pour Ambroise, l'empereur est un représentant de Dieu, qui doit agir comme tel. Enfin, Jean-Rémy Palanque résume parfaitement cette fusion des devoirs : l'empereur, soumis aux lois morales, conserve sa \latin{potestas}, mais doit l'utiliser pour servir la cause de Dieu en protégeant notamment l'Église. L'historien précise lui aussi qu'Ambroise ne distingue pas l'empereur et la fonction impériale, faisant de sa foi personnelle, une arme de politique générale\autocite[355]{palanque_ambroise}.

\bigskip

L'analyse de la théorie politique ambrosienne démontre donc que l'autonomie de l'Église et la montée en puissance d'une autorité parallèle ne signifie pas pour le pouvoir impérial un relâchement au sujet des questions religieuses. Au contraire, non seulement celles-ci continuent de se montrer majeures, mais elles sont plus que jamais résolues par la domination de la foi chrétienne, prioritaire sur toutes raisons morales. Charles Norris Cochrane le visualise particulièrement bien dans son ouvrage sur la pensée chrétienne, il explique notamment que la défense de la liberté par Ambroise est cadrée par une «~note d'autoritarisme typiquement romaine\autocite[355]{cochrane_christianity}.~». On peut alors penser que la liberté qu'il veut est celle de l'évêque plutôt que celle du peuple. Un paradoxe semble tout de même apparaître : comment valoriser l'indépendance de l'Église tout en ayant besoin d'une sollicitation de l'empereur sur les questions religieuses ? Ambroise le résout par la nécessité d'un encadrement rigoureux des interventions impériales, ce que nous allons désormais examiner.

\bigskip

\subsubsection{L'encadrement des interventions impériales}

L'un des points majeurs de la réflexion politique d'Ambroise est la nécessité de collaboration constante entre le pouvoir de l'Église et celui de l'empereur. L'évêque de Milan, comme présenté plus tôt, instaure l'\latin{auctoritas} épiscopale au cœur du fonctionnement de l'Empire, tout en ayant conscience de l'importance de la \latin{potestas} impériale pour faire fonctionner la «~machine~» administrative chrétienne. Dans ce cadre, le plus important est de mettre en place des limites strictes du pouvoir de l'empereur dans les questions religieuses, sans pour autant interdire les interventions. Ambroise continue de marquer par son réalisme politique, et possède une véritable conscience des limites de l'autorité spirituelle des évêques, qui peut se retrouver démunie face aux agitations publiques, par exemple des groupes considérés par la foi nicéenne comme hérétiques. La paix prônée par l'Église chrétienne, ainsi que l'ambition d'hégémonie, doit donc se reposer en partie sur le pouvoir du souverain.

\bigskip

Cette reconnaissance de l'utilité du pouvoir civil est notamment exprimée dans sa correspondance avec Gratien, dans laquelle il ne se contente pas de féliciter l'empereur pour ses actions religieuses, mais où il valorise l'usage de la force contre les ennemis de la foi :

\bigskip

\begin{quote}
    « En effet, vous m'avez rendu le repos de l'Église, vous avez fermé la bouche, puisque vous ne pouviez pas fermer le cœur, des perfides, et cela, vous l'avez fait par une autorité qui ne doit pas moins à votre foi qu'à votre pouvoir.\footnote{« \latin{Reddidisti enim mihi quietem ecclesiae, perfidorum ora atque utinam et corda clausisti ; et hoc non minore fidei quam potestatis auctoritate fecisti.}~». \cite[\nopp ec. 12, 2]{ambroise_csel_82_3}. Traduction personnelle.}~»
\end{quote}

\bigskip

Cette citation est révélatrice du pragmatisme d'Ambroise : il admet les limites de son propre pouvoir face à ses opposants et semble faire appel à une forme de répression plus directe. L'évêque de Milan évoque son échec, ainsi que celui de Gratien, de « fermer les cœurs », c'est-à-dire de changer la foi des hérétiques. En revanche, il le remercie pour avoir « fermé la bouche », donc très certainement d'avoir mis en place une interdiction de parole publique et peut-être de rassemblement, aspect réalisable uniquement par la \latin{potestas}. Bien qu'il soit difficile de savoir par ce simple extrait les détails de ce qu'évoque Ambroise, il est tout de même possible de constater la satisfaction par l'évêque de l'utilisation du pouvoir temporel du souverain au service de l'Église catholique. L'action impériale n'est donc pas toujours vue comme une ingérence mais plutôt comme un complément indispensable à la mission épiscopale.

\bigskip

Pourtant, je l'ai pleinement démontré dans les premières parties de ce chapitre, Ambroise fait régulièrement face aux empereurs, et principalement quand il s'agit d'ingérence dans un domaine qui se doit d'être réglé par l'Église. C'est le cas de l'affaire de l'autel de la Victoire, ou également du conflit concernant le don d'une basilique de Milan aux ariens. Cette opposition provient notamment de sa connaissance de l'histoire récente de la politique religieuse de l'Empire, et donc de sa conscience de l'importance de séparer les sphères de pouvoirs et de pleinement contrôler la capacité d'intervention des empereurs dans les questions de foi. En effet, comme le rappelle notamment l'ouvrage \latin{The Cambridge history of Greek and Roman political thought}, à partir de Constantin, l'Église se retrouve dans des disputes théologiques qui poussent les empereurs, malgré la volonté d’union, à prendre position, notamment avec Constance. Les différents camps de la foi chrétienne étaient dans un questionnement constant entre faire appel à l’empereur pour les aider, ou au contraire, revendiquer une indépendance cléricale quand l’empereur était contre eux\autocite[658]{rowe_history}. La mise en place par Constance II du concile de Rimini en 359 est une forme d'apogée du pouvoir impérial sur le religieux. Ce concile définit un dogme différent de celui de Nicée qui est ensuite utilisé par les ariens comme preuve de la légitimité de leur foi. C'est donc entre autres par cet événement qu'Ambroise et les évêques nicéens se retrouvent à s'opposer fermement aux ingérences impériales provenant de leur volonté propre. Une règle se dessine alors dans la seconde moitié du IV\textsuperscript{e} siècle : l'action de l'empereur ne peut pas venir de sa seule initiative, mais seulement d'une demande de l'Église catholique.

\bigskip

Cette dynamique d'appel et d'exécution se retrouve notamment dans les lettres d'Ambroise faisant état du concile d'Aquilée\autocite[ec 4, 5 et 6]{ambroise_csel_82_3} aux trois empereurs Gratien, Valentinien et Théodose. Ambroise fait ici intervenir la lutte contre les partisans de Photin, évêque de Sirmium sans doute entre 344 et 351 puis ayant bénéficié du règne de Julien pour revenir rapidement vers 361, qui forment un groupe toujours considéré comme hérétique par les évêques catholiques. Ambroise en appelle à l'application des lois et au pouvoir civil des empereurs :

\bigskip

\begin{quote}
    « Quant aux Photiniens, dont vous avez déjà décidé par une loi précédente qu'ils ne devaient tenir aucune assemblée [...], nous demandons que, puisque nous avons appris qu'ils tentent encore de s'assembler dans la ville de Sirmium, votre Clémence ordonne d'interdire leurs réunions de nouveau. Cela permettra de marquer du respect d'abord à l'Église catholique, et ensuite à vos propres lois.\footnote{« \latin{Fotinianos quoque quos et superiore lege censuistis nullos facere debere conventus [...], petimus ut quoniam in Sirmiensi oppido adhuc conventus temptare cognovimus, clementia vestra interdicta etiam nunc coitione reverentiam primum ecclesiae catholicae, deinde etiam legibus vestris deferre iubeatis.}~». \cite[\nopp ec. 4, 12]{ambroise_csel_82_3}. Traduction personnelle.}~»
\end{quote}

\bigskip

Par le « nous demandons », on comprend qu'il s'agit d'une initiative ecclésiastique, sans doute de l'ensemble du concile qui cherche à activer un levier juridique pour résoudre un problème de foi. Cette requête s'avère très intéressante pour comprendre toute la profondeur de la pensée d'Ambroise sur ce sujet. Il n'est pas demandé aux empereurs de trancher sur le fond de l'hérésie photinienne, mais bien d'appliquer une sanction civile qui semble nécessaire au maintien de l'ordre. L'intervention impériale est alors légitimée par la demande épiscopale, mais également par le fait qu'elle ne vise pas la résolution d'un problème dogmatique par une autorité non compétente, ce qui avait été critiqué dans le concile de Rimini notamment. Il est intéressant de relever le parallèle entre l'initiative ambrosienne et la question sur ce qu'est le maintien de l'ordre. Nous l'avons vu, Ambroise fait passer la réussite de la foi chrétienne au-dessus de la paix civile, comme pour l'affaire de Callinicum. En revanche, il semble ici que les divergences dogmatiques à Sirmium aillent, selon l'évêque, contre l'ordre dans l'Empire. D'une façon générale, Ambroise impose donc sa façon de penser dans la politique impériale, avec plus ou moins de succès bien sûr.

\bigskip

Ambroise veut donc mettre entre les mains des empereurs un rôle de facilitateur religieux, également dans les questions que l'on pourrait qualifier d'administratives ou de logistiques. Palanque parle de l'Empire comme le « bras séculier de l'Église » et démontre qu'Ambroise veut une ratification aveugle et docile des décisions du clergé et non pas un jugement moral ou juridique de la part de l'empereur\autocite[372]{palanque_ambroise}. Le résultat du concile de Rimini et ses conséquences se trouvent de nouveau être le parfait exemple de la nécessité de séparer les sphères d'intervention. Mais nous le comprenons bien en lisant Ambroise, le pouvoir impérial doit répondre aux exigences de l'Église :

\bigskip

\begin{quote}
    « C'est pourquoi nous vous demandons, très cléments et chrétiens princes, de décider qu'un concile de tous les évêques catholiques se tienne également à Alexandrie, afin qu'ils traitent entre eux plus amplement et définissent à qui la communion doit être accordée et pour qui elle doit être réservée.\footnote{« \latin{Ideoque petimus vos, clementissimi et Christiani principes, ut et Alexandriae sacerdotum catholicorum omnium concilium fieri censeatis, qui inter se plenius tractent atque definiant quibus impertienda communio quibusve servanda sit.}~». \cite[\nopp ec. 6, 5]{ambroise_csel_82_3}. Traduction personnelle.}~»
\end{quote}

\bigskip

Les évêques catholiques cherchent à utiliser la puissance administrative de l'Empire pour faciliter la gestion de leurs problématiques. La répartition des rôles apparaît très nettement : les empereurs fournissent une sorte de cadre matériel civil en convoquant le concile, tandis que l'objet du débat reste hors de leur portée. Ambroise ne fait qu'utiliser la puissance «~logistique~» de l'Empire.

\bigskip

Enfin, comme à son habitude, Ambroise n'hésite pas à adapter son idéal à la réalité politique et sociale qu'il confronte. Ainsi, bien qu'il défende en théorie l'élection des évêques par le clergé et le peuple, pour éviter des problèmes comme l'affaire d'Athanase, il n'hésite pas à faire appel au pouvoir impérial pour résoudre des complications d'arbitrages dans l'élection entre deux candidats. C'est le cas par exemple dans l'\latin{Epistolae extra collectionem} 9 qu'il adresse à Théodose pour lui demander son soutien à la nomination de l'évêque d'Antioche. Ainsi, on constate que l'indépendance que recherche Ambroise n'est pas un dogme aveugle de toute considération politique et pratique. Tant que l'intervention impériale se fait dans le cadre d'une demande épiscopale, elle se légitime en elle-même.

\bigskip

La réflexion politique d'Ambroise se forme donc autour d'un équilibre en constant mouvement, entre véritable autonomie et besoin de collaboration. Il utilise volontiers le pouvoir impérial pour débloquer des situations morales ou dogmatiques difficiles mais ne tolère aucunement toute prise d'initiative personnelle. La \latin{potestas} de l'empereur chrétien semble devoir se mettre à la disposition de l'Église et n'agir, dans la religion, que sous l'angle d'une vision cléricale.

\bigskip

\subsubsection{Une limitation de la \latin{potestas} impériale ?}

L'évêque de Milan instaure tout au long de son épiscopat une position politique de l'Église de plus en plus imposante. Que ce soit par l'avènement d'une nouvelle autorité morale et juridique dans l'Empire, ou encore plus nettement par l'influence de cette Église et d'Ambroise lui-même vis-à-vis de l'empereur et des décisions impériales. L'ensemble de l'appareil gouvernemental de l'Empire se retrouve donc modifié. Évidemment ce changement n'est pas à relier uniquement à l'évêque de Milan, mais celui-ci joue un rôle crucial dans la théorisation et l'application d'une forme de théologie politique des acteurs de la foi chrétienne. Pour autant, Ambroise ne remet jamais en cause la légitimité du pouvoir politique. Nous avons pu le voir tout au long de ce mémoire, bien que réfléchissant longuement sur le gouvernement et le souverain, il ne se place nullement comme véritable opposant politique ni théoricien « révolutionnaire », mais bien comme une source d'influence dont il est conscient et qu'il utilise pour tirer le pouvoir civil dans la direction qu'il souhaite. Concernant l'instauration d'une nouvelle source d'\latin{auctoritas}, nous avons pu le constater principalement par des extraits tirés de ses lettres aux empereurs, Ambroise s'applique à établir une distinction nette des domaines d'action. Ainsi, semble-t-il que la limitation qu'il impose aux empereurs ne soit pas une attaque contre son pouvoir ni son statut, mais une définition précise des frontières que chaque institution ne doit pas franchir pour l'équilibre de l'Empire.

\bigskip

Ambroise n'hésite d'ailleurs pas à s'exprimer sur ce sujet, et ce pour rappeler sa place et le fait qu'il n'outrepasse pas les droits obtenus par l'Église. L'évêque de Milan est donc conscient de la difficulté d'imposer à tous sa vision politique de la gestion des affaires religieuses, et est prêt à faire face aux accusations. Le meilleur exemple de cette défense d'Ambroise de ses droits se trouve dans sa correspondance avec sa sœur Marcelline. De nouveau, il fait part des événements dans la lutte contre l'arianisme et expose sa position, que l'on peut croire sincère puisque la lettre n'est pas à destination d'un empereur. Alors qu'il est accusé de se comporter en « tyran » en défendant sa basilique contre les ordres impériaux, il se défend en rappelant la nature historique et spirituelle du sacerdoce et en appelle aux autorités de l'Ancien et du Nouveau Testament :

\bigskip

\begin{quote}
    « Dans l’ancien droit les prêtres conféraient les pouvoirs, ils ne se les arrogeaient pas, et on disait habituellement que les empereurs désiraient davantage le sacerdoce que les prêtres le pouvoir impérial, le Christ a fui pour ne pas devenir roi\footnote{« Et Jésus, sachant qu'ils allaient venir l'enlever pour le faire roi, se retira de nouveau sur la montagne, lui seul. » Jean 6, 15.}. [...] J’ai ajouté que les prêtres n’ont jamais été des tyrans, mais ont eu souvent à souffrir des tyrans.\footnote{« \latin{Veteri iure a sacerdotibus donata imperia, non usurpata, et uulgo dici quod sacerdotes, Christus fugit, ne rex fieret. [...] Addidi quia numquam sacerdotes tyranni fuerunt sed tyrannos saepe sunt passi.}~». \cite[\nopp 76, 23]{ambroise_lettres}.}~»
\end{quote}

\bigskip

En évoquant « l'ancien droit », Ambroise fait référence aux royautés de l'Ancien Testament, où les figures de Saül, David et d'autres se sont faits rois par la légitimité apportée par les prêtres et prophètes. Par cette référence, l'évêque de Milan rappelle qu'un lien a toujours existé entre les représentants de la foi et le pouvoir civil, légitimant ainsi l'\latin{auctoritas} épiscopale comme une forme d'autorité ayant toujours existé. Puis, en plus d'exposer ses droits en tant qu'évêque, il se défend de toute ingérence cléricale envers le pouvoir impérial. Il s'appuie cette fois sur l'autorité du Christ : si même lui a fui le pouvoir temporel, alors pourquoi les évêques agiraient différemment ? Ce passage est une démonstration de l'ensemble de la réflexion ambrosienne sur ce sujet. Il maîtrise son sujet et sait se défendre, ce qui lui permet une plus grande liberté d'action. Une fois encore, il utilise sa théorie politique et religieuse pour l'exploiter dans des cas d'actions concrets. Il se permet même, du moins c'est ce qu'il pense puisqu'il l'exprime à Marcelline, une critique du pouvoir impérial et des interventions dans le monde religieux par leur propre initiative, rappelant ici certainement l'exemple d'Achab, souvent présenté comme un tyran impie pour avoir honoré Baal et avoir persécuté les représentants du Dieu unique de la chrétienté et du judaïsme. L'idée des prêtres souffrant des tyrans vise également sûrement à rappeler les multiples persécutions des empereurs païens envers les communautés chrétiennes. Ainsi, Ambroise place pratiquement l'Église comme une victime potentielle du pouvoir impérial plutôt que comme un opposant politique.

\bigskip

Dans le même contexte de l'affaire de 386 sur les basiliques de Milan, Ambroise défend avec virulence les positions catholiques contre les demandes impériales de restitution d'une église aux ariens. Cette position l'amène entre autres à être critiqué par les représentants de l'empereur comme un opposant à l'empereur, et donc le signe d'une Église dans la confrontation de la \latin{potestas} impériale. Ce à quoi il répond fermement dans son \latin{Sermo contra Auxentium} :

\bigskip

\begin{quote}
    « Que je me sois exprimé avec la déférence due à l'empereur, personne ne peut le nier. Comment montrer plus de déférence qu'en disant que l'empereur est le fils de l'Église ? En disant cela, ce n'est pas une faute de le dire, nous lui sommes loyaux. L'empereur, en effet, est au-dedans de l'Église, il n'est pas au-dessus de l'Église ; un bon empereur, en effet, demande le secours de l'Église, il ne le rejette pas. Tout cela, si nous le disons avec humilité, nous le déclarons avec fermeté.\footnote{« \latin{Quod cum honorificentia imperatoris dictum nemo potest negare. Quid enim honorificentius quam ut imperator ecclesiae filius esse dicatur ? Quod cum dicitur sine peccato dicitur, cum gratia dicitur. Imperator enim intra ecclesiam non supra ecclesiam est ; bonus enim imperator quaerit auxilium ecclesiae, non refutat. Haec ut humiliter dicimus ita constanter exponimus.}~». \cite[\nopp 75A, 36]{ambroise_lettres}.}~»
\end{quote}

\bigskip

De nouveau, Ambroise ne se prive pas de rappeler l'interdiction des ingérences impériales dans les sphères d'action de l'Église par cette injonction « L'empereur, en effet, est au-dedans de l'Église, il n'est pas au-dessus de l'Église. » Cette façon de présenter le pouvoir impérial est d'ailleurs une définition récurrente de l'empereur chrétien selon Ambroise. Dans sa traduction de l'\latin{Epistolae}, Gérard Nauroy fait d'ailleurs référence à l'une des premières rencontres entre Ambroise et Théodose, rapportée par Sozomène, où « l'évêque lui interdit de prendre place dans le chœur de l'Église, lui assignant une place au premier rang des fidèles.\footnote{Gérard Nauroy, note 1, page 432. \cite[\nopp 75A, 36]{ambroise_lettres}.}~». Il y a toujours, au sujet de la foi, cette délimitation complexe du pouvoir de chacun. L'empereur ne peut pas user de sa \latin{potestas} sur la religion, mais les évêques ne sont pas pour autant supérieurs au pouvoir impérial. Mais plus important encore, cette citation met en avant deux aspects cruciaux dans la relation d'Ambroise au pouvoir impérial. Premièrement la mise en avant constante d'une collaboration entre les deux institutions : l'idée que l'empereur doive s'appuyer sur l'autorité et la connaissance de l'Église pour consolider son pouvoir et sa gestion de l'Empire, et inversement que l'Église a besoin d'une autorité armée pour faciliter le contrôle de la foi chrétienne. Deuxièmement l'importance de la loyauté des évêques envers les empereurs chrétiens. Ce principe évite toute possibilité d'opposition brutale de la part du monde épiscopal : limiter le pouvoir de l'empereur dans le sacré n'est pas symbole d'une attaque envers son rôle politique. Finalement, il est possible de voir la création d'une \latin{auctoritas} par Ambroise comme la continuité de la politique de Gratien, qui décide en 382 de renoncer au titre de \latin{Pontifex Maximus}, et par là même à bon nombre de ses droits sur la sphère religieuse. La restriction des compétences des empereurs concernant les décisions religieuses, que nous avons pu longuement constater, n'est pas une soumission juridique à l'Église ni à Ambroise, mais une forme de désacralisation de la fonction impériale, qui se voit limitée par les règles de sa propre foi.

\bigskip

\subsection*{Conclusion}

Au terme de cette analyse de la séparation des institutions, il apparaît assez clairement que la « nouvelle doctrine » évoquée par Giuseppe Visonà n'est pas une simple rhétorique liée à un besoin d'action dans des circonstances précises, mais bien une véritable définition politique et théologique de la société romaine, et notamment de son gouvernement. Ambroise réussit à théoriser puis faire appliquer ses réflexions, principalement par le biais de ses communications aux empereurs, et réussit ainsi à placer l'\latin{auctoritas} impériale et sénatoriale entre les mains de l'Église chrétienne. Le prestige moral et juridique de l'\latin{auctoritas} permet à l'Église de consolider son influence sur le pouvoir sans pour autant développer de véritables compétences politiques.

\bigskip

L'apport principal d'Ambroise réside dans la création d'une expertise exclusive aux évêques sur les questions de foi. La \latin{potestas} impériale se place, dans ce cadre, derrière une frontière clairement définie empêchant l'empereur d'intervenir sans tomber dans la tyrannie ou le sacrilège. En tant qu'« homme de l'Église », le souverain se retrouve dans l'Église chrétienne, sans privilège lié à son statut civil. Il n'est alors plus en légitimité d'intervenir dans la sphère religieuse sur sa propre initiative, plaçant ainsi le monde épiscopal dans une véritable position autonome. Le pouvoir impérial n'est donc pas véritablement réduit, mais sa souveraineté est limitée par les exigences de la foi. Une logique de collaboration s'installe entre les protagonistes, mais toujours sous le regard de l'Église. Ambroise est bien évidemment conscient du rôle majeur joué par les empereurs dans le maintien et le renforcement de la foi chrétienne, mais il cherche à valoriser cette \latin{auctoritas} et vise ainsi à contrôler les interventions impériales.

\bigskip

La dynamique entre l'Église et le pouvoir impérial théorisée par Ambroise s'inscrit, en plus de l'aspect institutionnel, dans les relations plus personnelles entre les évêques et les empereurs. La figure d'Ambroise, en tant qu'évêque de la capitale occidentale de l'Empire, prend donc une place toujours plus importante auprès des empereurs, mettant au premier plan la question de l'\latin{auctoritas} personnelle. Ce qui nous amène à observer la mise en œuvre concrète du rôle politique que nous avons développé, à travers l'action des évêques, et plus particulièrement la relation personnelle que développe Ambroise avec les souverains contemporains.
