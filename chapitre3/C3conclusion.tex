\section*{Conclusion}

Au terme de cette analyse de l'arrivée d'une nouvelle dynamique entre l'autorité épiscopale et le pouvoir impérial, il apparaît assez clairement qu'Ambroise de Milan a joué un rôle majeur dans la redéfinition des rapports de force à la fin du IV\textsuperscript{e} siècle. En articulant la revendication d'une autonomie institutionnelle de l'Église avec la pratique personnelle de l'influence des évêques sur les empereurs, il fait intervenir les fondements du modèle politique des territoires chrétiens, qui donne une place politique de premier plan aux autorités spirituelles. L'empereur, autrefois maître incontesté du domaine du sacré, se retrouve soumis à la juridiction morale de l'Église, de la même manière que n'importe quel autre fidèle, voire même plus à cause de ses responsabilités.

\bigskip

Ce chapitre nous permet d'explorer une autre facette des écrits d'Ambroise. En effet, la pensée politique d'Ambroise ne se présente pas comme une simple réflexion théorique parfaitement cohérente ; elle doit au contraire composer avec l'instantanéité des actions et événements, ce qui amène une grande partie de ses pensées à être exprimées par les correspondances entretenues avec les évêques, les empereurs ou sa sœur. Qu'il s'agisse de l'affaire de l'autel de la Victoire, des missions diplomatiques à Trèves ou du massacre de Thessalonique, la doctrine ambrosienne se démarque par une réponse réaliste à un besoin concret. Sa théorie lui permet de suivre une ligne directrice claire, mais qui demande une constante adaptation. Cette dimension explique les nombreuses variations d'approche et de stratégie que l'on observe à travers ses écrits : il adapte son discours à la relation qu'il entretient avec son interlocuteur, qu'il soit dans une phase d'éducation morale et politique ou au contraire simplement à la recherche d'une plus grande légitimité et popularité par la religion. Le fait qu'une grande partie de la relation qu'Ambroise établit et entretient entre l'Église et le pouvoir nous soit principalement connue par des sources du cœur de l'action, et non pas extérieures à celle-ci, nous permet de comparer avec précision l'idéal d'Ambroise, dans sa construction d'un gouvernement du peuple et d'un empereur chrétien modèle, avec la réalité qu'il impose aux souverains contemporains. S'il n'essaie pas de s'interposer comme un opposant à la politique impériale, il n'hésite pas pour autant à faire connaître sa position pour imposer des façons d'agir afin de guider le pouvoir politique dans la direction qu'il souhaite. La récupération du concept d'\latin{auctoritas} pour l'Église et surtout pour lui-même se fait ressentir régulièrement dans ses lettres, lorsqu'il insiste sur l'indépendance des évêques, la séparation des domaines d'action et surtout le positionnement de l'empereur dans l'Église et non pas au-dessus de celle-ci.

\bigskip

Il est toutefois nécessaire de poser les limites de cette « victoire » ambrosienne en rappelant les contestations impériales contre son \latin{auctoritas} : on est encore loin d'une évidence pour les souverains. Les lettres, ou plutôt dans ce cadre l'absence de lettre, nous démontrent des résistances régulières contre lesquelles l'ensemble des stratégies d'influence d'Ambroise ne peuvent rien. C'est notamment le cas dans la relation avec Théodose entre les deux affaires de Callinicum et de Thessalonique : bien que pleinement chrétien, l'empereur n'accepte pas l'intervention d'Ambroise en 388 et choisit de l'écarter des informations et décisions politiques sur les deux années suivantes. Cette tension rappelle que l'autonomie et l'autorité de l'Église restent, à cette époque, dépendantes du bon vouloir impérial qui peut être rapidement et facilement remis en cause. L'exemple précédemment évoqué des dernières années de vie de l'évêque de Milan reste la démonstration la plus marquante de cette limite de l'autorité. Avec le refus total de collaboration du général romain Stilicon, tuteur d'Honorius, Ambroise est relégué à son statut d'évêque. Bien qu'Ambroise s'impose sans aucun doute durant son épiscopat comme la figure spirituelle la plus influente de son temps, capable d'intervenir de façon régulière et respectée dans les affaires impériales, il ne met pas pour autant en place un statut officiel des évêques auprès des empereurs romains.

\bigskip

En définitive, ce troisième chapitre nous permet une meilleure compréhension de l'impact d'Ambroise dans l'évolution des rapports entre l'Église et les empereurs. L'évêque de Milan est bien loin d'instaurer une théocratie ni même de contester l'autonomie du pouvoir politique, mais il parvient à travers ses réflexions théoriques et ses communications avec les détenteurs du pouvoir à développer une autorité épiscopale suffisante pour obliger les empereurs à se référer aux évêques lors des questions de foi, en leur interdisant notamment de prendre des initiatives. L'\latin{auctoritas} de l'Église n'est donc pas encore une institution politique officielle, mais commence à apparaître comme une réalité incontournable de la gestion de l'Empire. Bien qu'il s'adresse au nom de l'ensemble des représentants chrétiens, c'est principalement par ses propres relations, connaissances diplomatiques et popularité qu'Ambroise s'impose en tant que conseiller impérial influent, représentant définitivement cette autorité épiscopale qu'il recherche et expose.
